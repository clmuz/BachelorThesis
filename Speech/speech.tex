\documentclass{article}

\usepackage[no-math]{fontspec}		% XeLaTeX fonts
\usepackage{polyglossia}        	% Babel for XeLaTeX
\usepackage{hyperref}				% Ссылки внутри текста
\usepackage{url}					% URL ссылки
\usepackage{geometry}				% Margins
\usepackage{mathspec}				% Math fonts
\usepackage{amssymb}				% For Real symbol
\usepackage{pgfplots}       		% Графики
\usepackage{graphicx}       		% Работа с картинками
\usepackage{float}          		% Необходимо для указания места картинки на странице
\usepackage{marginnote}				% Margin notes
\usepackage{color}

\geometry{							% Margins
	a4paper,
	left =	 30mm,
	right =	 30mm,
	top =	 25mm,
	bottom = 25mm,}

\setdefaultlanguage{russian}		% Polyglossia Default language
\setotherlanguage{english}			% Polyglossia Second language

\setmainfont[Ligatures=TeX]{CMU Serif}
\setsansfont[Ligatures=TeX]{CMU Sans Serif}
\setmonofont{CMU Typewriter Text}
\setmathfont{Latin Modern Math}

\newcommand{\redtext}[1]{{\color{red} #1}}
\newcommand{\note}[1]{\noindent\rule{\textwidth}{.2pt}\marginpar{\redtext{#1}}

}

\title{Адаптивный рандомизированный алгоритм выделения сообществ в графах}
\author{Тимофей Проданов}
\date{}

\begin{document}

\maketitle

\note{Слайд 1}
Добрый день, я Тимофей Проданов и я буду рассказывать об адаптивном выделении сообществ в графах.

\note{Слайд 2:\\Выделение\\ сообществ}
Последние двадцать лет развивается изучение сложных сетей, то есть графов с неправильной, сложной структурой. Сложные сети были с успехом применены в различных областях, например в биоинформатике и социологии. Сложные сети, построенные по реальным системам, часто разбиваются на группы тесно связанных узлов графа, сообщества. Способность находить и анализировать подобные группы узлов даёт большие возможности в изучении реальных систем, например, сообщества в интернете представляют распространённые темы. Поиск таких групп узлов в называют выделением сообществ или кластеризацией.

В 2004 году Марк Ньюман представил целевую функцию модулярность, показывающую качество разбиения графа на сообщества, определение модулярности изображено на слайде. Модулярность удобна в использовании, так как подсчёт выигрыша от объединения двух сообществ занимает одну операцию, а объединение двух сообществ --- $О$ от количества соседей одного из сообществ. Таким образом, задачу выделения сообществ рассматривают как задачу максимизации модулярности.

\note{Слайд 3:\\Алгоритмы выделения сообществ}
Эффективными оказались рандомизированные алгоритмы выделения сообществ в графах. В 2010 году Овельгённе и Гейер-Шульц предложили рандомизированный жадный алгоритм выделения сообществ с параметром $k$. На каждой итерации рассматривается $k$ случайных сообществ, у каждого сообщества исследуются соседи, и затем соединяется лучшая пара.

В 2012 году те же учёные представили схему кластеризации основных группа графа. Сначала $s$ начальных алгоритмов выделяют сообщества, а те узлы, которые начальные алгоритмы назначили в разные сообщества, распределяет по сообществам финальный алгоритм. В итеративной схеме узлы, относительно которых начальные алгоритмы не сошлись во мнении, снова распределяют по сообществам начальные алгоритмы.

В качестве начальных и финального алгоритмов можно использовать рандомизированный жадный алгоритм с разными параметрами.

\note{Слайд 4:\\\emph{SPSA}.\\ Постановка задачи}
Качество работы этих алгоритмов критично зависит от их параметров, и остаётся открытым вопрос об адаптивных версиях алгоритмов, которые были бы работоспособны на большем количестве задач.

Для создания таких модификаций используется стохастическая аппроксимация. Алгоритм одновременно возмущаемой стохастической аппроксимации, или по другому \emph{SPSA}, предполагает разбиение улучшаемого алгоритма на два $n$ шагов. В течении нечётного шага улучшаемый алгоритм использует один параметр, а во время чётного --- другой, после чего вычисляются следующие два значения параметра.

В работе рассматривается применение алгоритма \emph{SPSA} к двум рандомизированным алгоритмам выделения сообществ для получения модификаций, способных хорошо работать на большем количестве входных графов.

\note{Слайд 5:\\Применимость \emph{SPSA}.\\ Функция качества}
Применимость \emph{SPSA} обоснована теоретически для выпуклой в среднем функции качества.

На слайде изображена медианная модулярность разбиений алгоритмом $RG$ двух графов при разных значениях параметра $k$. На каждом графе модулярность либо достигает максимума при небольшом $k$, либо постоянно увеличивается при росте $k$. Однако время работы линейно зависит от $k$, поэтому предлагается функция качества, зависящая и от модулярности, и от параметра $k$, она показана на слайде. Такая функция в среднем выпуклая и подходит для работы \emph{SPSA}.

\note{Слайд 6:\\Адаптивный рандомизированный жадный алгоритм}
Предлагается адаптиный рандомизированный жадный алгоритм или $ARG$, построенный на основе $RG$. Алгоритм идентичен рандомизированному жадному алгоритму с параметром в $k$. Однако действие $ARG$ разбито на шаги длиной в $\sigma$ итераций, после окончания шага параметр $k$ меняется.

В функции качества используется медиана прироста модулярности за последние $\sigma$ шагов. Использовать прирост модулярности опасно, так как алгоритм рандомизированный и даже при плохом $k$ иногда модулярность будет сильно возрастать.

Для получения следующих двух параметров $k$ некоторая текущая оценка параметра возмущается в обе стороны. Следующая оценка будет ближе к тому возмущению, которое дало меньшее значение функции качества, формула вычисления изображена на слайде.

\end{document}