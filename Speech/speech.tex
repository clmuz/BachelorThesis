\documentclass{article}

\usepackage[no-math]{fontspec}		% XeLaTeX fonts
\usepackage{polyglossia}        	% Babel for XeLaTeX
\usepackage{hyperref}				% Ссылки внутри текста
\usepackage{url}					% URL ссылки
\usepackage{geometry}				% Margins
\usepackage{mathspec}				% Math fonts
\usepackage{amssymb}				% For Real symbol
\usepackage{pgfplots}       		% Графики
\usepackage{graphicx}       		% Работа с картинками
\usepackage{float}          		% Необходимо для указания места картинки на странице
\usepackage{marginnote}				% Margin notes
\usepackage{color}

\geometry{							% Margins
	a4paper,
	left =	 30mm,
	right =	 30mm,
	top =	 25mm,
	bottom = 25mm,}

\setdefaultlanguage{russian}		% Polyglossia Default language
\setotherlanguage{english}			% Polyglossia Second language

\setmainfont[Ligatures=TeX]{CMU Serif}
\setsansfont[Ligatures=TeX]{CMU Sans Serif}
\setmonofont{CMU Typewriter Text}
\setmathfont{Latin Modern Math}

\newcounter{FrameNum}
\newcommand{\redtext}[1]{{\color{red} #1}}
\newcommand{\note}[1]{\noindent\rule{\textwidth}{.2pt}\marginpar{\redtext{Слайд \stepcounter{FrameNum}\arabic{FrameNum}\\#1}}

}

\title{Адаптивный рандомизированный алгоритм выделения сообществ в графах}
\author{Тимофей Проданов}
\date{}

\begin{document}

\maketitle

\note{}
Доброе утро, я Тимофей Проданов и я буду рассказывать об адаптивном выделении сообществ в графах.

\note{Сложные сети}
Последние двадцать лет развивается изучение сложных сетей, то есть графов с неправильной, сложной структурой. Сложные сети были с успехом применены в различных областях, например в биоинформатике и социологии. Было показано, что такие сети, построенные по реальным системам часто имеют характеристики, усложняющие анализ классической теорией графов или статистический анализ.

Сложные сети часто разбиваются на группы тесно связанных узлов графа, сообщества. Способность находить и анализировать подобные группы узлов даёт большие возможности в изучении реальных систем, например, сообщества в интернете представляют распространённые темы.

\note{Блондель}
Винсент Блондель в 2008 году разбил на сообщества два миллиона абонентов бельгийской телефонной компании. На рисунке размер круга отвечает за размер сообщества, а цвет за основной язык --- французский или голландский. Заметно, как зелёные и красные сообщества распределелись по разным сторонам сети, а в центральных сообществах цвета оказались смешанными.

\note{Алгоритмы}
В 2004 году Марк Ньюман представил целевую функцию модулярность, показывающую качество разбиения графа на сообщества.

Эффективными оказались рандомизированные алгоритмы максимизации модулярности. В 2010 году Овельгённе и Гейер-Шульц предложили рандомизированный жадный алгоритм выделения сообществ с параметром $k$. На каждой итерации рассматривается $k$ случайных сообществи их соседей, и затем соединяется лучшая пара.

В 2012 году те же учёные победили на конкурсе DIMACS со схемой кластеризации основных групп графа, которая заключается в следующем: Сначала $s$ начальных алгоритмов выделяют сообщества, а те узлы, которые начальные алгоритмы поместили в разные сообщества, затем разбивает по сообществам финальный алгоритм. В итеративной схеме узлы, относительно которых начальные алгоритмы разошлись во мнении, снова распределяют по сообществам начальные алгоритмы.

\note{Постановка задачи}
Однако не существует набора параметров, при которых эти алгоритмы хорошо выделяют сообщества в каждом графе.

Для решения этой проблемы в работе рассматривается создание адаптивных версий алгоритмов, то есть приспосабливающихся к входным данным. И целью является создание алгоритма, дающего хорошие результаты на большем количестве графов.

\note{\emph{SPSA}}
Для создания адаптивных алгоритмов применяется стохастический градиентный спуск к рандомизированному жадному алгоритму и к схеме кластеризации основных групп графа.

Алгоритм $SPSA$ предполагает разбиение улучшаемого алгоритма на два $n$ шагов. В течении нечётного шага алгоритм использует один параметр, а во время чётного --- другой, после чего подбираются следующие два значения параметра.

\newpage
\note{$ARG$}
Предлагается адаптиный рандомизированный жадный алгоритм, почти идентичный рандомизированному жадному алгоритму с параметром $k$. Однако действие $ARG$ разбито на шаги длиной в $\sigma$ итераций, после окончания шага параметр $k$ подстраивается.

Для получения следующих двух параметров $k$ некоторая текущая оценка параметра возмущается в обе стороны. Следующая оценка будет ближе к тому возмущению, которое дало лучший результат. Оценка результата зависит от $k$ и от медианы прироста модулярности за $\sigma$ шагов.

\note{Параметры $ARG$}
В отличии от $RG$, имевшего один параметр, $ARG$ имеет пять параметров, однако от чувствительности и количества итераций в шаге результат зависит слабо, а размер возмущения и начальная оценка имеет значения, дающие хороший результат на всех графах. Существует параметр, отвечающий за значимость времени, при его увеличении время работы уменьшается, однако при уменьшении модулярности.

\note{Сравнение $RG$ и $ARG$}
Для сравнения качества рандомизированного жадного алгоритма и его адаптивной версии сопоставлялась медианная модулярность рзбиений графов с конкурса DIMACS. Левее пунктирной линии указаны $RG$ с разными параметрами, правее --- $ARG$ с разными параметрами. Чем более зелёный цвет --- тем разбиение лучше, и чем более красный --- тем хуже. Видно, что $RG$ даёт плохие результаты чаще, чем $ARG$. [$RG_{10}$]

% \note{Применение $ARG$ в $CGGC$}
% Первым этапом $CGGC$ $s$ начальных алгоритмов выделяют сообщества. Если начальный алгоритм даёт плохие разбиения, то и результат всей схемы будет плохим. Однако по значению модулярности невозможно сказать, относительно хорошее это разбиение или относительно плохое.

% Поэтому имеет смысл использовать $ARG$ вместо $RG$ в качестве начального алгоритма, ведь нестабильные начальные алгоритмы портят финальное разбиение.

% На таблице изображены результаты $CGGC$, в верхней строке указаны начальные алгоритмы, во второй --- финальные, а слева --- три тестовых графа. Золотым цветом отмечены лучшие результаты, серебрянным --- вторые по модулярности, а красным --- очень плохие. Видно, что в последних двух столбцах с начальным алгоритмом $ARG$ показаны более хорошие значения, и ни разу не появляются очень плохие.

\note{$ACGGC$}
Представляется адаптивная схема кластеризации основных групп графа, схожая с неадаптивной схемой, но на первом этапе в качестве начальных алгоритмов используется $RG$ с подстраивающимися параметрами.

Затем в создании промежуточного разбиения участвуют не все начальные разбиения, но только некоторое количество лучших.

\note{Параметры $ACGGC$}
$ACGGC$ имеет несколько параметров, однако можно подобрать набор параметров, достаточно хорошо работающих на всех тестовых графах.

\note{Снижение времени работы}
Предложено два механизма снижения времени работы. Первый из них заключается в возможности увеличения параметра значимости времени, уменьшая время и модулярность.

Второй механизм устанавливает ограничение на максимальную оценку параметра $k$, этот вариант более стабилен и как можно увидеть из тепловых карт --- время всегда уменьшается при уменьшении максимальной оценки, и часто модулярности становятся выше при уменьшении максимальной оценки. Красный цвет тут обозначает не такие плохие разбиения, как на предыдущем рисунке.

% \note{Сравнение $ACGGC$ и $CGGC$}
% Слева на таблице указаны тестовые графы с конкурса DIMACS, от 34 до полутора миллионов узлов. Два первых столбца представляют результаты адаптивную схему с разными параметрами, а следующие три --- результаты неадаптивную схему с разными параметрами. Золотым отмечены лучшие результаты, серебрянным --- вторые по модулярности и красным --- очень плохие результаты.

% $ACGGC$ даёт лучше и более стабильные значения.

\note{Сравнение $ACGGCi$ и $CGGCi$}
Как и $CGGC$, $ACGGC$ можно итерировать. На таблице изображены сравнение итерационных схем кластеризации основных групп графа. В первых двух столбцах изображена адаптивная схема с разными параметрами, в третьей --- неадаптивная и в последней --- комбинация этих двух итеративных схем. Как можно увидеть, неадаптивная итерационная схема редко давала лучшее разбиение, а также в отличии от остальных один раз приняла очень плохое значение.

\note{Результаты}
В рамках работы были исследованы методы современные методы выделения сообществ в сложных сетях, предложен новый адаптивный рандомизированный жадный алгоритм, исследованы его параметры и произведено сравнение результатов с неадаптивным рандомизированным жадным алгоритмом. Адаптивный алгоритм даёт более стабильные результаты и хорошо работает на большем количестве входных графов.

Также представлена адаптивная схема кластеризации основных групп графа, проанализированы её параметры, предложены механизмы снижения времени работы и результаты сопоставлены с результатами неадаптивной схемы. На тестовых графах в большинстве случаев адаптивная схема дала лучшие результаты и ни разу не дала очень плохого разбиения на сообщества.

Были рассмотрены адаптивная итеративная схема кластеризации основных групп графа и комбинирование адаптивной и неадаптивной схемы кластеризации основных групп графа.

\end{document}