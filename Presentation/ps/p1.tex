\begin{frame}
	\frametitle{Сложные сети}

	1998 г. Воттс, Строгатц --- начало изучения сложных сетей%
	\footnote{Watts and Strogatz. \bluetext{Collective dynamics of ``small-world'' networks.}} \vspace{.5em}

	Графы с неправильной, сложной структурой \vspace{.5em}

	Применение:
	\begin{itemize}
		\item эпидемиология
		\item биоинформатика
		\item поиск преступников
		\item социология
		\item изучение структуры и топологии интернета
	\end{itemize} \vspace{.5em}

	Усложнён анализ теорией графов или статистический анализ \vspace{.5em}

	\bluetext{Сообщества} --- тесно связанные группы узлов \vspace{.5em}

	Поиск таких групп --- выделение сообществ, кластеризация
\end{frame}

\begin{frame}
	\frametitle{Пример разбиения на сообщества}
	\begin{center}
		\includegraphics[height = 0.8\textheight]{Blondel.png}
	\end{center}
	\scriptsize Blondel et al. \bluetext{Fast unfolding of communities in large networks.}
\end{frame}

\begin{frame}
	\frametitle{Алгоритмы выделения сообществ}

	2004 г. Модулярность%
	\footnote{\scriptsize Newman and Girvan. \bluetext{Finding and evaluating community structure in networks.}}
	 $Q$
	\vspace{.5em}

	2010 г. Рандомизированный жадный алгоритм%
	\footnote{\scriptsize Ovelg{\"o}nne and Geyer-Schulz. \bluetext{Cluster cores and modularity maximization.}} $RG$:
	\begin{itemize}
		\item $k$ случайных сообществ и их соседи
		\item Лучшая пара соседей
	\end{itemize}\vspace{.5em}

	2012 г. Схема кластеризации основных групп графа%
	\footnote{\scriptsize Ovelg{\"o}nne and Geyer-Schulz. \bluetext{A comparison of agglomerative hierarchical algorithms for modularity clustering.}} $CGGC$:
	\begin{enumerate}
		\item $s$ \bluetext{начальных алгоритмов}
		\item \bluetext{Финальный алгоритм} распределяет неопределившиеся узлы
	\end{enumerate}\vspace{.5em}

	% Итеративная схема $CGGCi$ \vspace{.5em}

	$RG$ как начальный и финальный алгоритм
\end{frame}


\begin{frame}
	\frametitle{Постановка задачи адаптации параметров}
	Качество работы $RG$ и $CGGC$ зависит от их параметров\\
	Для каждого графа --- свои хорошие параметры \vspace{2em}

	Адаптивные алгоритмы, приспосабливающиеся к входным данным \vspace{2em}

	Цель:\\
	Хорошие результаты на большем количестве графов
\end{frame}

\begin{frame}
	\frametitle{Стохастический градиентный спуск}
	Применение стохастического градиентного спуска $SPSA$ к $RG$ и $CGGC$ \vspace{1em}

	Стохастический градиентный спуск $SPSA$:
	\begin{itemize}
		\item Разбиение алгоритма на $2n$ шагов
		\item Каждый шаг новый параметр
		\item Новые значения параметра подбираются в зависимости от предыдущих шагов
	\end{itemize}

\end{frame}

