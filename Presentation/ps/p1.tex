\begin{frame}
	\frametitle{Пример разбиения на сообщества}
	\begin{center}
		\includegraphics[height = 0.86\textheight]{Blondel.png}
	\end{center}
\end{frame}

\begin{frame}
	\frametitle{Выделение сообществ в графах}
	\bluetext{Сообщества} --- тесно связанные группы узлов в графах

	2004 г. \bluetext{Модулярность}:
	$$Q(G, P) = \sum_{i \in 1..K}{\left(e_{ii} - a_i^2\right)},$$
	где
	\begin{itemize} 
		\item $G$ --- граф
		\item $P$ --- разбиение на сообщества
		\item $K$ --- количество сообществ
		\item $e$ --- нормированная матрица смежности сообществ
		\item $a$ --- вектор нормированных степеней сообществ
	\end{itemize}
\end{frame}


\begin{frame}
	\frametitle{Алгоритмы выделения сообществ}


	2010 г. Рандомизированный жадный алгоритм $RG$%
	\footnote{\scriptsize Ovelg{\"o}nne and Geyer-Schulz. \bluetext{Cluster cores and modularity maximization.} ICDMW'10. IEEE International Conference on Data Mining Workshops. pp. 1204–1213 (2010)}:
	\begin{itemize}
		\item $k$ случайных сообществ и их соседи
		\item Лучшая пара соседей
	\end{itemize}\vspace{.5em}

	2012 г. Схема кластеризации основных групп графа $CGGC$%
	\footnote{\scriptsize Ovelg{\"o}nne and Geyer-Schulz. \bluetext{A comparison of agglomerative hierarchical algorithms for modularity clustering.} Graph Partitioning and Graph Clustering. 588:187, 2012}:
	\begin{enumerate}
		\item $s$ \bluetext{начальных алгоритмов}
		\item \bluetext{Финальный алгоритм} распределяет неопределившиеся узлы
	\end{enumerate}\vspace{.5em}

	Итеративная схема $CGGCi$ \vspace{.5em}

	$RG$ как начальный и финальный алгоритм
\end{frame}


\begin{frame}
	\frametitle{Стохастическая аппроксимация. Постановка задачи}
	Качество работы $RG$ и $CGGC$ зависит от их параметров \vspace{.5em}

	Одновременно возмущаемая стохастическая аппроксимация $SPSA$:
	\begin{itemize}
		\item Разбиение алгоритма на $2n$ шагов
		\item Каждый шаг новый параметр
		\item Каждые два шага вычисляются следующие два значения параметра
	\end{itemize} \vspace{.5em}

	Хорошие результаты на большем количестве входных данных \vspace{.5em}

	Применение $SPSA$ к $RG$ и $CGGC$
\end{frame}


\begin{frame}
	\frametitle{Применимось \emph{SPSA}. Функция качества}
	Выпуклая усреднённая функция качества\vspace{.5em}
	
	Рандомизированный жадный алгоритм $RG$:\vspace{-1em}
	\begin{figure}[H]
		\begin{tikzpicture}
			\begin{axis}[
			    table/col sep = semicolon,
			    height = 0.45\paperheight, 
			    width = 0.49\columnwidth,
			    xlabel = {$k$},
			    ylabel = {$Q$},
			    legend pos = north east,
			    no marks,
			    title = {karate},
			    y label style={at={(axis description cs:-.2,.5)}, anchor=south},
			]
			\addplot table [x={k}, y={q}]{../Text/data/arg/validity/karate.csv};
			\end{axis}
		\end{tikzpicture}
		\hskip 0.01\columnwidth
		\begin{tikzpicture}
			\begin{axis}[
			    table/col sep = semicolon,
			    height = 0.45\paperheight, 
			    width = 0.49\columnwidth,
			    xlabel = {$k$},
			    ylabel = {$Q$},
			    legend pos = south east,
			    no marks,
			    title = {netscience},
			    y label style={at={(axis description cs:-.18,.5)}, anchor=south},
			]
			\addplot table [x={k}, y={q}]{../Text/data/arg/validity/netscience.csv};
			\end{axis}
		\end{tikzpicture}
	\end{figure}~\vspace{-2.5em}\\

	Время растёт линейно от $k$\vspace{.5em}

	Функция качества:\vspace{-.5em} $$f(Q, k) = -\alpha (\ln Q - \beta \ln k),\ \alpha > 0,\ \beta \ge 0$$
\end{frame}


