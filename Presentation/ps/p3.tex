

% \begin{frame}
% 	\frametitle{\emph{ARG} как начальный алгоритм \emph{CGGC}}
% 	Если начальные алгоритмы дают плохой результат в схеме $CGGC$, то и окончательный результат будет плохим. По модулярности одного разбиения нельзя сказать, хорошее разбиение или плохое. Запускать $RG_k$ с разными параметрами для проверки качества разбиения не выгодно. Поэтому имеет смысл запускать более стабильный $ARG$.

% 	\small
% 	\begin{table}[H]
% 		\caption{Модулярность разбиений, полученных в результате работы $CGGC$ с начальным алгоритмом $A_{init}$ и финальным алгоритмом $A_{final}$ на трёх графах}
% 		\begin{tabularx}{\textwidth}{Xrrrrrr} \hline
% 		$A_{init}$	& \multicolumn{2}{c}{$RG_3$} & \multicolumn{2}{c}{$RG_{10}$} & \multicolumn{2}{c}{$ARG$} \\
% 		$A_{final}$ & \multicolumn{1}{c}{$RG_3$} & \multicolumn{1}{c}{$RG_{10}$} & \multicolumn{1}{c}{$RG_3$} & \multicolumn{1}{c}{$RG_{10}$} & \multicolumn{1}{c}{$RG_3$} & \multicolumn{1}{c}{$RG_{10}$} \\\hline
% 		$G_1$ 		& \cred 0.16840	& 0.71155	& \cred 0.44934	& \csilver 0.74794	& 0.42708	& \cgold 0.74872	\\
% 		$G_2$				& 0.80628	& \csilver 0.80645	& 0.80633	& \csilver 0.80645	& 0.80628	& \cgold 0.80647	\\
% 		$G_3$	& 0.84078	& \cgold 0.85372	& 0.84031	& 0.84448	& 0.83671	& \csilver 0.85279	\\\hline
% 		\end{tabularx}
% 	\end{table}

% \end{frame}

\begin{frame}
	\frametitle{Адаптивная схема кластеризации основных групп графа}

	$ACGGC$ схожа с $CGGC$:
	\begin{enumerate}
		\item $l$ начальных алгоритмов:
		\begin{itemize}
			\item $RG_k$ с разными $k$
			\item $k$ подстраиваются
		\end{itemize}
		\item Из $l$ начальных разбиений выбирается несколько лучших
		\item Неопределившиеся узлы разбивает по сообществам финальный алгоритм
	\end{enumerate}
\end{frame}


\begin{frame}
	\frametitle{Параметры \emph{ACGGC}}

	\begin{multicols}{2}

	\bluetext{\textbullet} Размер возмущения $d$,\\
	~~~Чувствительность $\alpha$
	\begin{figure}[H]
		\begin{tikzpicture}
			\begin{semilogxaxis}[
			    table/col sep = semicolon,
			    height = 0.35\paperheight, 
			    width = 0.99\columnwidth,
			    xlabel = {$\alpha$},
			    ylabel = {$Q$},
			    legend columns = -1,
			    legend style={
					at={(1,-.4)},
					anchor=north east},
			    no marks,
			    ymin = 0.665,
			    yticklabel style={/pgf/number format/fixed,
	                  /pgf/number format/precision=3},
	            x label style={at={(axis description cs:.28,-.35)}, anchor=south},
	            y label style={at={(axis description cs:-.15,.35)}, anchor=south},
			]
			\addplot table [x={alpha}, y={q}]{../Text/data/es/alphad/july_1.csv};
			\addplot table [x={alpha}, y={q}]{../Text/data/es/alphad/july_2.csv};
			\legend{{\scriptsize $d = 1$}, {\scriptsize $d = 2$}};
			\end{semilogxaxis}
		\end{tikzpicture}
	\end{figure}

	\vspace{-.4em} \bluetext{\textbullet} Начальная оценка $\hat{k}_0$
	\begin{figure}[H]
		\begin{tikzpicture}
			\begin{axis}[
			    table/col sep = semicolon,
			    height = 0.35\paperheight,
			    width = 0.99\columnwidth,
			    xlabel = {$\hat{k}_0$},
			    ylabel = {$Q$},
			    no marks,
			    y label style={at={(axis description cs:-0.2,.5)}, anchor=south},
	            x label style={at={(axis description cs:.28,-.38)}, anchor=south},
			    yticklabel style={/pgf/number format/fixed,
	                  /pgf/number format/precision=3},
			]
			\addplot table [x = {k0}, y={q}]{../Text/data/es/k0/july_k0.csv};
			\end{axis}
		\end{tikzpicture}
	\end{figure}

	\newpage \bluetext{\textbullet} Доля хороших начальных разбиений $r$ \vspace{-1em}
	\begin{figure}[H]
		\begin{tikzpicture}
			\begin{axis}[
			    table/col sep = semicolon,
			    height = 0.4\paperheight,
			    width = 0.99\columnwidth,
			    xlabel = {$r$},
			    ylabel = {$Q$},
			    no marks,
			    legend pos = south east,
			    y label style={at={(axis description cs:-0.2,.5)}, anchor=south},
	            x label style={at={(axis description cs:.28,-.3)}, anchor=south},
			    yticklabel style={/pgf/number format/fixed,
	                  /pgf/number format/precision=3},
			]
			\addplot table [x = {r}, y={q}]{../Text/data/es/r-r1/july_r.csv};
			\addplot table [x expr = {1 - \thisrowno{0}}, y={q}]{../Text/data/es/r-r1/july_r1.csv};
			\legend{{$r_1$}, {$r_2$}};
			\end{axis}
		\end{tikzpicture}
	\end{figure}

	\vspace{-1em} \bluetext{\textbullet} Количество шагов $l$ \vspace{-1em}
	\begin{figure}[H]
		\begin{tikzpicture}
			\begin{axis}[
			    table/col sep = semicolon,
			    height = 0.4\paperheight,
			    width = 0.99\columnwidth,
			    xlabel = {$l$},
			    ylabel = {$Q$},
			    legend pos = south east,
			    no marks,
			    y label style={at={(axis description cs:-0.23,.5)}, anchor=south},
	            x label style={at={(axis description cs:.28,-.3)}, anchor=south},
			    yticklabel style={/pgf/number format/fixed,
	                  /pgf/number format/precision=3},
			]
			\addplot table [x = {l}, y={q}]{../Text/data/es/l/july_l_0.05.csv};
			\addplot table [x = {l}, y={q}]{../Text/data/es/l/july_l_1.csv};
			\legend{{\scriptsize $r = 0.05$}, {\scriptsize $r = 1$}};
			\end{axis}
		\end{tikzpicture}
	\end{figure} \vspace{-2.1em}~

	\end{multicols}
\end{frame}


\begin{frame}
	\frametitle{Механизмы снижения времени работы}
	\bluetext{\textbullet} Увеличение значимости времени: \vspace{-0.5em}
	\begin{figure}[H]
		\begin{tikzpicture}
			\begin{semilogxaxis}[
			    table/col sep = semicolon,
			    height = 0.4\paperheight,
			    width = 0.49\columnwidth,
			    xlabel = {$\beta$},
			    ylabel = {$Q$},
			    no marks,
			    title = {Модулярность},
			    title style={at={(0.5,0.9)}},
			    y label style={at={(axis description cs:-0.25,.5)}, anchor=south},
			    yticklabel style={/pgf/number format/fixed,
	                  /pgf/number format/precision=3},
	            x label style={at={(axis description cs:.5,-.3)}, anchor=south},
			]
			\addplot table [x = {beta}, y = {q}]{../Text/data/es/beta/jazz_beta.csv};
			\end{semilogxaxis}
		\end{tikzpicture}
		\hskip 0.01\columnwidth
		\begin{tikzpicture}
			\begin{semilogxaxis}[
			    table/col sep = semicolon,
			    height = 0.4\paperheight,
			    width = 0.49\columnwidth,
			    xlabel = {$\beta$},
			    title = {Время},
			    title style={at={(0.5,0.9)}},
			    ylabel = {$t$},
			    y label style={at={(axis description cs:-0.2,.5)}, anchor=south},
			    no marks,
			    x label style={at={(axis description cs:.5,-.3)}, anchor=south},
			]
			\addplot[red] table [x = {beta}, y = {time}]{../Text/data/es/beta/jazz_beta.csv};
			\end{semilogxaxis}
		\end{tikzpicture}
	\end{figure}\vspace{-1em}

	\bluetext{\textbullet} Установка ограничения на максимальную оценку $k_{max}$: \vspace{-0.5em}
	\begin{figure}[H]
	\hskip 0.02\columnwidth
		\begin{tikzpicture} %max(0, 1 + (rgs1[y, x] - max(rgs1[y, ]))/(.003 * max(rgs1[y, ])))
			\begin{axis}[
				height = 0.4\paperheight,
			    width = 0.49\columnwidth,
				view={0}{90},
				% grid=minor,
	 		    % grid style={dotted},
			    colormap = {redgreen}{rgb255(0cm)=(255,0,0); rgb255(1cm)=(255,255,255); rgb255(2cm)=(0,255,0)},
			    mesh/ordering=y varies,
			    mesh/cols=6,
			    mesh/rows=9,
			    xtick={1.5,2.5,3.5,4.5,5.5},
    			xticklabels={\scriptsize $+\infty$, \scriptsize $50$, \scriptsize $20$, \scriptsize $10$, \scriptsize $6$},
    			xlabel = {$k_{max}$},
    			ylabel = {Graphs},
			    title = {Модулярность},
			    title style={at={(0.5,0.9)}},
    			ytick={1.5,2.5,3.5,4.5,5.5,6.5,7.5,8.5},
	            x label style={at={(axis description cs:0.5,-0.35)}, anchor=south},
    			yticklabels = {\scriptsize $G_8$, \scriptsize $G_7$, \scriptsize $G_6$, \scriptsize $G_5$, \scriptsize $G_4$, \scriptsize $G_3$, \scriptsize $G_2$, \scriptsize $G_1$},
    			% x tick label style = {align = center, rotate = 50, anchor = north east},
			 ]
			\addplot3[surf,shader=flat corner] table{../Text/data/es/kmax/kmax1.csv};
			\end{axis}
		\end{tikzpicture}
		\hskip 0.01\columnwidth
		\begin{tikzpicture} %min(rgs1[y, ]) / rgs[y, x]
			\begin{axis}[
				height = 0.4\paperheight,
			    width = 0.49\columnwidth,
				view={0}{90},
				% grid=minor,
	 		    % grid style={dotted},
			    colormap = {redgreen}{rgb255(0cm)=(255,0,0); rgb255(1cm)=(255,255,255); rgb255(2cm)=(0,255,0)},
			    mesh/ordering=y varies,
			    mesh/cols=6,
			    mesh/rows=9,
			    xtick={1.5,2.5,3.5,4.5,5.5},
    			xticklabels={\scriptsize $+\infty$, \scriptsize $50$, \scriptsize $20$, \scriptsize $10$, \scriptsize $6$},
    			xlabel = {$k_{max}$},
    			ylabel = {Graphs},
			    title = {Время},
			    title style={at={(0.5,0.9)}},
    			ytick={1.5,2.5,3.5,4.5,5.5,6.5,7.5,8.5},
    			x label style={at={(axis description cs:0.5,-0.35)}, anchor=south},
    			yticklabels = {\scriptsize $G_8$, \scriptsize $G_7$, \scriptsize $G_6$, \scriptsize $G_5$, \scriptsize $G_4$, \scriptsize $G_3$, \scriptsize $G_2$, \scriptsize $G_1$},
    			% x tick label style = {align = center, rotate = 50, anchor = north east},
			 ]
			\addplot3[surf,shader=flat corner] table{../Text/data/es/kmax/kmax_t1.csv};
			\end{axis}
		\end{tikzpicture}
	\end{figure}

\end{frame}

\begin{frame}
	\frametitle{Сравнение \emph{CGGC} и \emph{ACGGC}}
	\scriptsize
	\begin{table}[H]
		\begin{tabularx}{\textwidth}{lrrrrr} \hline
					& $ACGGC^{I}$ & $ACGGC^{II}$ & $CGGC_{10}^{10}$	& $CGGC_{3}^{10}$ & $CGGC_{10}^{3}$ \\ \hline
		karate 			& \csilver 0.417242 	& \cgold 0.417406 	& 0.415598 	& 0.396532	& 0.405243	\\
		dolphins		& \cgold{0.524109}	& \csilver{0.523338}	& 0.521399	& 0.523338	& 0.522428	\\
		chesapeake		& \csilver{0.262439}	& \csilver{0.262439}	& \csilver{0.262439}	& \csilver{0.262439}	& 0.262370	\\
		adjnoun			& \cgold{0.299704}	& \csilver{0.299197}	& 0.295015	& 0.292703	& 0.290638	\\
		polbooks		& \csilver{0.527237}	& \csilver{0.527237}	& \csilver{0.527237}	& 0.526938	& 0.526784	\\
		football		& 0.603324	& \csilver{0.604266}	& \csilver{0.604266}	& 0.599537	& 0.599026	\\
		celegans 		& \cgold{0.439604}	& \csilver{0.438584} 	& 0.435819	& 0.436066	& 0.432261	\\
		jazz			& 0.444739	& \csilver{0.444848}	& \cgold{0.444871} 	& 0.444206 	& 0.444206	\\
		netscience		& \cgold{0.907229}	& \csilver{0.835267}	& \cred 0.724015 	& \cred 0.708812 	& \cred 0.331957	\\
		email			& \csilver{0.573333}	& \cgold{0.573409}	& 0.571018 	& 0.572667 	& 0.567423	\\
		polblogs		& \cgold{0.424107} 	& \csilver{0.423208}	& 0.422901 	& 0.421361 	& 0.390395	\\
		pgpGiantCompo	& \cgold{0.883115} 	& \csilver{0.883085}	& 0.882237 	& 0.882532	& 0.880340	\\
		as-22july06		& \cgold{0.671249}	& \csilver{0.670677}	& 0.666766	& 0.669847	& 0.665260	\\
		cond-mat-2003	& 0.744533 	& \csilver{0.750367}	& \cgold{0.751109} 	& \cred 0.708775	& \cred 0.413719	\\
		caidaRouterLevel& 0.846312	& \csilver{0.855651}	& 0.851622 	& \cgold{0.858955} 	& 0.843835	\\
		cnr-2000		& 0.912762 	& \cgold{0.912783}	& 0.912500 	& \csilver{0.912777}	& 0.912496	\\
		eu-2005			& \cgold{0.938292}  & \csilver{0.936984}	& 0.935510 	& 0.936515	& 0.936420	\\
		in-2004			& \csilver{0.979844}	& 0.979771	& \cgold{0.979883}	& 			& 			\\
		\hline
		\end{tabularx}
	\end{table}
\end{frame}

% \begin{frame}
% 	\frametitle{Сравнение \emph{CGGCi} и \emph{ACGGCi}}
% 	\begin{table}[H]
% 	\begin{tabularx}{\textwidth}{Xrrrr} \hline
% 					& $ACGGCi^{I}$ & $ACGGCi^{II}$ & $CGGCi$	& combined \\ \hline
% 	karate			& 0.417242	& \cgold{0.417406}	& 0.417242	& 0.417242	\\
% 	dolphins		& 0.525869	& 0.525869	& 0.525869	& 0.525869	\\
% 	chesapeake		& 0.262439	& 0.262439	& 0.262439	& 0.262439	\\
% 	adjnoun			& 0.303731	& 0.303504	& 0.303571	& \cgold{0.303970}	\\
% 	polbooks		& 0.527237	& 0.527237	& 0.527237	& 0.527237	\\
% 	football		& 0.604266	& 0.604407	& \cgold{0.604429}	& 0.604407	\\
% 	celegans 		& 0.446964	& 0.446836	& 0.445442	& \cgold{0.447234}	\\
% 	jazz			& 0.444871	& 0.444871	& 0.444871	& 0.444871	\\
% 	netscience		& \cgold{0.908845}	& 0.888422	& \cred 0.725781	& 0.907443	\\
% 	email			& 0.576778	& 0.577000	& 0.576749	& \cgold{0.577110}	\\
% 	polblogs		& \cgold{0.424025}	& 0.422920	& 0.423281	& 0.423996	\\
% 	\hline
% 	\end{tabularx}
% \end{table}
% \end{frame}


\begin{frame}
	\frametitle{Результаты}

	В рамках работы
	\begin{itemize}
		\item Проанализированы современные методы выделения сообществ
		\item Предложен $ARG$
		\begin{itemize}
			\item Исследованы параметры
			\item Более стабилен, чем $RG$
		\end{itemize}
		\item Представлена $ACGGC$
		\begin{itemize}
			\item Проанализированы параметры
			\item Описаны механизмы снижения времени работы
			\item Более стабилен, чем $CGGC$ и лучше результаты
		\end{itemize}
		\item Рассмотрена $ACGGCi$
		\item Исследована комбинация из $ACGGCi$ и $CGGCi$
	\end{itemize}
\end{frame}