\section{Предварительные сведения}
\label{sec:intro}

%%%%%%%%%%%%%%%%%%%%%%%%%%%%%%%%%%%%%%%%%%%%%%%%%%%%%%%%%%%%%%%%%%%%%%%%%%%%%%%%%%%%%%%%%%%%%%%%%%%%%%%%%%%%%%%

\subsection{Определения и обозначения. Сообщества в графах}

Формально сложная сеть может быть представлена с помощью графа. В этой работе будут рассматриваться только невзвешенные неориентированные графы. Неориентированный невзвешенный граф $G = (\mathscr{N}, \mathscr{L})$ состоит из двух множеств --- множества $\mathscr{N} \ne \emptyset$, элементы которого называются \emph{узлами} или \emph{вершинами} графа, и множества $\mathscr{L}$ неупорядоченных пар из множества $\mathscr{N}$, элементы которого называются \emph{рёбрами} или \emph{связями}. Мощности множеств $\mathscr{N}$ и $\mathscr{L}$ равны $N$ и $L$ соответственно.

\emph{Подграфом} называется граф $G' = (\mathscr{N}', \mathscr{L}')$, где $\mathscr{N}' \subset \mathscr{N}$ и $\mathscr{L}' \subset \mathscr{L}$.

Узел обычно обозначают по его порядковому месту $i$ в множестве $\mathscr{N}$, а ребро, соединяющее пару узлов $i$ и $j$ обозначают $l_{ij}$. Узлы, между которыми есть ребро называются \emph{смежными}. Степенью узла назовают величину $s_i$, равную количеству рёбер, выходящих узла $i$.

\emph{Прогулка} из узла $i$ в узел $j$ --- это последовательность узлов, начинающаяся с узла $i$ и заканчивающаяся узлом $j$. \emph{Путь} --- это прогулка, в которой каждый узел встречается единожды. \emph{Геодезический путь} --- это кратчайший путь, а количество узлов в нём на один больше \emph{геодезического расстояния}.

Пусть $G = (\mathscr{N}, \mathscr{L})$ --- граф, \emph{разбиением на сообщества} будем называть разбиение множества его вершин $P = \{C_1, \dots, C_K\}$, то есть $\bigcup_{i = 1}^K C_i = \mathscr{N}$ и $C_i \cap C_j = \emptyset \ \forall i \neq j \in 1..K$.

Множество сообществ $P = \{C_1, \dots, C_{K_1}\}$ будем называеть разбиением на сообщества \emph{на основе} разбиения $\widetilde{P} = \left\{\widetilde{C}_1, \dots \widetilde{C}_{K_2}\right\}$, если $\forall i \in 1..K_2\ \exists j \in 1..K_1: \widetilde{C}_i \subset C_j$.

Сообщество --- это такой подграф, чьи узлы плотно связаны, однако структурную сплочённость узлов можно определить по разному. Одно из определений вводит понятие \emph{клик}. Клик --- это максимальный подграф, состоящий из трёх и более вершин, каждая из которых связана с каждой другой вершиной из клика. $n$-клик --- это максимальный подграф, в котором самое большое геодезическое расстояние между любыми двумя вершинами не превосходит $n$. Сообщество определяют как $n$-клик \cite{Wasserman:1994}. Другое определение гласит, что подграф $G'$ является сообществом, если сумма всех степеней внутри $G'$ больше суммы всех степеней, направленных в остальную часть графа \cite{Wasserman:1994}.

Два сообщества называются \emph{смежными}, если существует ребро, направленное из вершины первого сообщества в вершину второго.


%%%%%%%%%%%%%%%%%%%%%%%%%%%%%%%%%%%%%%%%%%%%%%%%%%%%%%%%%%%%%%%%%%%%%%%%%%%%%%%%%%%%%%%%%%%%%%%%%%%%%%%%%%%%%%%

\subsection{Модулярность}
\label{subsec:modularity}

Однако описанными выше определениями сообществ пользоваться неудобно и их нахождение достаточно долгое с вычислительной точки зрения и NP-сложное \cite{Kvrivanek&Moravek:1986, Brucker:1978}. В 2004 году в \cite{Newman&Girvan:2004} было введено понятие \emph{модулярность} --- целевая функция, оценивающая неслучайность разбиения графа на сообщества.

Допустим, имеется $K$ сообществ, тогда \emph{нормированная матрица смежности сообществ} $\mathbf{e}$ определяется как симметричная матрица размером $K \times K$, где элементы $e_{ij}$ равны отношению количества рёбер, которые идут из сообщества $i$ в сообщество $j$, к полному количеству рёбер в графе (рёбра $l_{mn}$ и $l_{nm}$ считаются различными, где $m$, $n$ --- узлы). След этой матрицы $\mathrm{Tr} (\mathbf{e}) = \sum_{i \in 1..K}{e_{ii}}$ показывает отношение рёбер в сети, которые соединяют узлы одного и того же сообщества, к полному количеству рёбер в графе. Хорошее разбиение на сообщества должно иметь высокое значение следа.

Однако если поместить все вершины в одно сообщество --- след примет максимальное возможное значение, притом, что такое разбиение не будет сообщать ничего полезного о графе. Поэтому определяется вектор $\mathbf{a}$ длины $K$ с элементами $a_i = \sum_{j \in 1..K}{e_{ij}}$. Координата вектора $a_i$ является \emph{нормированной степенью сообщества} $i$ и обозначает долю количества рёбер, идущих к узлам, принадлежащим сообществу $i$, к полному количеству рёбер в графе. Если в графе рёбра проходят между вершинами независимо от сообществ --- $e_{ij}$ будет в среднем равно $a_i a_j$, поэтому модулярность можно определить следующим образом \cite{Newman&Girvan:2004}:
\begin{equation} \label{eq:q1}
Q(G, P) = \sum_{i \in 1..K}{\left(e_{ii} - a_i^2\right)} = \mathrm{Tr} \mathbf{e} - \|\mathbf{e}^2\|,
\end{equation}
где $\|\mathbf{x}\|$ является суммой элементов матрицы $\mathbf{x}$. Если сообщества распределены не лучше, чем в случайном разбиении --- модулярность будет примерно равна 0. Максимальным возможным значением функции будет 1, но на практике единица не достигается.

Было предложено несколько вариаций определений модулярности \cite{Muff&Rao&Caflisch:2005, Fortunato&Barthelemy:2007}. Так, для взвешенных графов существует следующее определение модулярности:
\begin{equation}\label{eq:modularity-weighted}
Q(G, P) = \frac{1}{2L} \sum_{x, y \in 1..N} \left(w_{xy} - \frac{s_x s_y}{2L}\right)\ \delta(c_P(x), c_P(y)),
\end{equation}

Во взвешенных графах каждое ребро $l_{xy}$ имеет характеристику \emph{вес} $w_{xy} \in \mathbb{R}$, в неориентированных графах $w_{xy} = w_{yx}$. В определении \eqref{eq:modularity-weighted} использовались следующие обозначения: $L$ --- мощность $\mathscr{L}$, $w_{xy}$ --- вес ребра между вершинами $x$ и $y$, $s_x$ и $s_y$ --- степени вершин $x$ и $y$ соответственно, $\delta$ --- символ Кронекера, а отображение $c_P(\cdot)$ указывает, в каком сообществе разбиения лежит узел графа. При условии, что все веса равны нулю или единице --- определение \ref{eq:modularity-weighted} эквивалентно определению \eqref{eq:q1}.

Теперь можно поставить задачу выделения сообществ следующим образом: \emph{требуется найти} такое разбиение графа, что модулярность примет максимальное значение. Можно заметить, что такая постановка задачи не использует какого-либо определения сообществ, и получившееся разбиение не проверяется на дополнительные свойства, кроме подсчёта модулярности.

Такая задача всё ещё будет NP-сложной \cite{Brandes&al:2008}. Однако преимущество модулярности состоит в том, что для того, чтобы посчитать, какой выигрыш будет извлечён из объединения двух сообществ, необходимо произвести только одну операцию. В рамках определения \eqref{eq:q1} такой выигрыш будет равен $\Delta Q = 2(e_{ij} - a_i a_j),$ где $i$ и $j$ --- потенциально объединяемые сообщества.

Для того, чтобы объединить два сообщества необходимо сделать $O(\min\{n_i, n_j\})$ операций, где $n_i$ и $n_j$ обозначают количество смежных к $i$ и $j$ сообществ. Не умоляя общности, $n_j \leq n_i$, тогда необходимо обновить столбец $i$-ый столбец и $i$-ую строку матрицы $\mathbf{e}$, а так же $i$-ый элемент вектора $\mathbf{a}$: $e_{ik} = e_{ki} = e_{ki} + e_{kj},$ где $k$ --- смежное к $j$ сообщество, и $a_{i} = a_{i} + a_{j}$. При этом сообщество $j$ следует удалить из дальнейшего рассмотрения.

Имея матрицу $\mathbf{e}$ и вектор $\mathbf{a}$ не очень важно, как устроен граф и сообщества, что позволяет искать сообщества, основываясь на некотором начальном разбиении, для которого уже построены $\mathbf{e}$ и $\mathbf{a}$.


%%%%%%%%%%%%%%%%%%%%%%%%%%%%%%%%%%%%%%%%%%%%%%%%%%%%%%%%%%%%%%%%%%%%%%%%%%%%%%%%%%%%%%%%%%%%%%%%%%%%%%%%%%%%%%%

\subsection{Рандомизированный жадный алгоритм (RG)}

Ньюман в 2004 году в \cite{Newman:2004} предложил алгоритм, максимизирующий модулярность \eqref{eq:q1}. Алгоритм начинается с разбиения графа на $K = N$ сообществ из одной вершины, а затем на каждой итерации просматривает все пары сообществ и соединяет ту пару, которая даст наибольший выигрыш модулярности.

Такой алгоритм достаточно долго работает и страдает от несбалансированного объединения сообществ --- сообщества растут с разной скоростью, большие кластеры соединяются со своими небольшими соседями независимо от того, выгодно это глобально или нет \cite{Ovelgoenne&Geyer-Schulz:2012a}. Поэтому был предложен рандомизированный жадный алгоритм (Randomized Greedy, \emph{RG}) в \cite{Ovelgoenne&Geyer-Schulz:2010}, который на каждой итерации рассматривал $k$ случайных сообществ и смежных к ним сообществ, а затем так же соединял пару, дающую наибольший выигрыш. Трудоёмкость такого алгоритма примерно равна $O(kL \ln N)$.

И первый алгоритм, и его рандомизированная вариация соединяют сообщества, записывая только номера соединений, до тех пор, пока не останется только одно сообщество, а затем создают разбиение из списка соединений до того момента, когда достигалась максимальная модулярность (так как в результате лучшего соединения модулярность может уменьшиться). 

Далее в работе рандомизированный жадный алгоритм с параметром $k$ будет обозначаться $RG_k$.
Можно отметить, что таким алгоритмом можно выделять сообщества на основе некоторого начального разбиения, при этом только немного поменяется начальный этап инициализации матрицы $\mathbf{e}$ и вектора $\mathbf{a}$ (как можно увидеть в алгоритме~\ref{alg:RG}).\\

\begin{algorithm}[H]
\setstretch{1}
\SetAlgoLined
\KwIn{Невзвешенный неориентированный граф $G = (\mathscr{N}, \mathscr{L})$, параметр $k$}
\KwOut{Разбиение на сообщества $P$}
\For{$i \in 1..N$}{
	\For{$j \in 1..N$}{
		\eIf{$i$ и $j$ смежные}{
			$e[i, j] = 1 / (2 * L)$\;
		}{
			$e[i, j] = 0$\;
		}
	}
	$a[i] = \sum_j e[i, j]$\;
}
\BlankLine
% \caption{Этап инициализации рандомизированного жадного алгоритма}
% \label{alg:RG-init}
$global\Delta Q \leftarrow 0$\;
$max\_global\Delta Q \leftarrow -\infty$\;
\For{$i \in 1..N$}{
	$max\Delta Q \leftarrow -\infty$\;
	\For{$j \in 1..k$}{
		$c1 \leftarrow$ случайное сообщество\;
		\ForAll{сообщества $c2$, смежные с $c1$}{
			$\Delta Q \leftarrow 2 * (e[i, j] - a[i] * a[j])$\;
			\If{$\Delta Q > max\Delta Q$}{
				$max\Delta Q \leftarrow \Delta Q$\;
				$next\_join \leftarrow (c1, c2)$\;
			}
		}
	}
	$joins\_list.push(next\_join)$\;
	$global\Delta Q \leftarrow global\Delta Q + max\Delta Q$\;
	\If{$global\Delta Q > max\_global\Delta Q$}{
		$max\_global\Delta Q \leftarrow global\Delta Q$\;
		$best\_step \leftarrow i$\;
	}
	\BlankLine
	$(c1, c2) \leftarrow next\_join$\;
	\If{количество соседей($c2$) > количество соседей($c1$)}{
		поменять местами $c1$ и $c2$\;
	}
	\ForAll{соседи $c3$ сообщества $c2$, где $c3 \neq c1, c2$}{
		$e[c3, c1] \leftarrow e[c3, c1] + e[c3, c2]$\;
		$e[c1, c3] \leftarrow e[c3, c1]$\;
	}
	$e[c1, c1] \leftarrow e[c1, c1] + e[c2, c2] + e[c1, c2] + e[c2, c1]$\;
	$a[c1] \leftarrow a[c1] + a[c2]$\;
}
\BlankLine
$P \leftarrow $ создать разбиение из $joins\_list[1..best\_step]$\;
\BlankLine
\caption{Рандомизированный жадный алгоритм}
\label{alg:RG}
\end{algorithm}

\vspace{0.1cm}

%%%%%%%%%%%%%%%%%%%%%%%%%%%%%%%%%%%%%%%%%%%%%%%%%%%%%%%%%%%%%%%%%%%%%%%%%%%%%%%%%%%%%%%%%%%%%%%%%%%%%%%%%%%%%%%

\subsection{Схема кластеризации основных групп графа (CGGC)}
\label{subsec:ens}

Овельгённе и Гейер-Шульц в 2012 году выиграли конкурс 10th DIMACS Implemen\-tation Challenge в категории \emph{кластеризация графа} со схемой кластеризации основных групп графа (Core Groups Graph Cluster, \emph{CGGC}). Схема заключается в том, что сначала $s$ \emph{начальных алгоритмов} разбивают граф на сообщества, и считается, что те вершины, в которых начальные алгоритмы сошлись во мнении, распределены по сообществам правильно, а те, которые остались, распределяет по сообществам \emph{финальный алгоритм} \cite{Ovelgoenne&Geyer-Schulz:2012b}.

Формализовать это можно следующим образом:
\begin{enumerate}
	\item Создаётся множество $S$ из $s$ разбиений $G$ с помощью начальных алгоритмов
	\item Создаётся разбиение $\hat{P}$, равное максимальному перекрытию разбиений из множества $S$
	\item Финальным алгоритмом создаётся разбиение $P$ графа $G$ на основе разбиения $\hat{P}$
\end{enumerate}

Необходимо определить понятие \emph{максимальное перекрытие}. Пусть существует множество разбиений $S = \{P_1, \dots, P_s\}$, отображение $c_P(v)$ указывает, в каком сообществе находится узел $v$ в разбиении $P$.
Тогда у максимального перекрытия $\hat{P}$ множества $S$ будут следующие свойства:
$$v, w \in \mathscr{N}, \forall i \in 1..s :\ c_{P_i}(v) = c_{P_i}(w) \Rightarrow c_{\hat{P}}(v) = c_{\hat{P}}(w)$$
$$v, w \in \mathscr{N}, \exists i \in 1..s :\ c_{P_i}(v) \ne c_{P_i}(w) \Rightarrow c_{\hat{P}}(v) \ne c_{\hat{P}}(w)$$

\emph{CGGC} можно итерировать, заставляя начальные алгоритмы разбивать максимальное перекрытие и получившееся максимальное перекрытие до тех пор, пока это будет увеличивать модулярность, такой алгоритм далее обозначается как \emph{CGGCi}. В таком случае схема будет выглядеть следующим образом:

\begin{enumerate}
	\item Инициализировать $\hat{P}$ разбиением из сообществ из одного узла
	\item Создать множество $S$ из $s$ разбиений графа $G$ на основе разбиения $\hat{P}$ с помощью начальных алгоритмов
	\item Записать в $\hat{P}$ максимальное перекрытие множества $S$
	\item Если разбиение $P_{best}$ не существует или оно хуже, чем $\hat{P}$, то присвоить $P_{best} \leftarrow \hat{P}$ и вернуться на второй шаг
	\item Финальным алгоритмом создать разбиение $P$ графа $G$ на основе разбиения $P_{best}$ 
\end{enumerate}

В качестве начальных и финального алгоритма можно брать $RG_k$.


%%%%%%%%%%%%%%%%%%%%%%%%%%%%%%%%%%%%%%%%%%%%%%%%%%%%%%%%%%%%%%%%%%%%%%%%%%%%%%%%%%%%%%%%%%%%%%%%%%%%%%%%%%%%%%%

\subsection[Одновременно возмущаемая стохастическая аппроксимация (SPSA)]{Одновременно возмущаемая стохастическая\\ аппроксимация (SPSA)}
Стохастическая аппроксимация была введена Роббинсом и Монро в 1951 году \cite{Robbins&Monro:1951} и затем была использована для решения оптимизационный задач Кифером и Вольфовицем \cite{Kiefer&Wolfowitz:1952}. В \cite{Blum:1954} алгоритм стохастической аппроксимации был расширен до многомерного случая. В $m$-мерном пространстве обычная KW-процедура, основанная на конечно-разностной аппроксимации градиента, использовала $2m$ измерений на каждой итерации (по два измерения на каждую координату градиента). Последовательно Граничин \cite{Граничин:1989}, Поляк и Цыбаков \cite{Поляк&Цыбаков:1990} и Спалл \cite{Spall:1992} предложили алгоритм \emph{одновременно возмущаемой стохастической аппроксимации} (\emph{SPSA}), который на каждой итерации использует всего два измерения. Алгоритм \emph{SPSA} имеет такую же скорость сходимости, несмотря на то, что в многомерном случае (даже при $m \to \infty$) в нём используется заметно меньше измерений \cite{Spall:2005}.

Стохастическая аппроксимация первоначально использовалась как инструмент для статистических вычислений и в дальнейшем разрабатывалась в рамках отдельной ветки теории управления. На сегодняшний день стохастическая аппроксимация имеет большое разнообразие применений в таких областях, как адаптивная обработка сигналов, адаптивное выделение ресурсов, адаптивное управление.

Алгоритмы стохастической аппроксимации показали свою эффективность в решении задач минимизации стационарных функционалов. В \cite{Polyak:1987} для функционалов, меняющихся со временем были применены метод Ньютона и градиентный метод, но они применимы только в случае дважды дифференцируемых функционалов и в случае известных ограничений на Гессиан функционала. Так же оба метода требуют возможности вычисления градиента в произвольной точке.

Общую схему одновременно возмущаемой стохастической аппроксимации можно представить следующим образом:

\begin{enumerate}
	\item Выбор начального приближения $\hat{\theta}_0 \in \mathbb{R}^m$, счётчик $n \leftarrow 0$, выбор параметров алгоритма $d \in \mathbb{R} \setminus \{0\}$, $\{\alpha_n\} \subset \mathbb{R}^m$
	\item Увеличение счётчика $n \leftarrow n + 1$
	\item Выбор вектора возмущения $\Delta_n \in \mathbb{R}^m$, чьи координаты независимо генерируются по распределению Бернулли, дающему $\pm1$ с вероятностью $\frac{1}{2}$ для каждого значения
	\item Определение новых аргументов функционала $\theta_{n}^{-} \leftarrow \hat{\theta}_{n - 1} - d\Delta_{n}$ и $\theta_{n}^{+} \leftarrow \hat{\theta}_{n - 1} + d\Delta_{n}$
	\item Вычисление значений функционала $y_n^{-} \leftarrow f(\theta_{n}^{-}), y_n^{+} \leftarrow f(\theta_{n}^{+})$
	\item Вычисление следующей оценки
	\begin{equation} \label{eq:spsa-central}
		\hat{\theta}_n \leftarrow \hat{\theta}_{n - 1} - \alpha_n \Delta_{n} \frac{y_n^{+} - y_n^{-}}{2d}
	\end{equation}
	\item Далее происходит либо остановка алгоритма, либо переход на второй пункт
\end{enumerate}

В \cite{Kushner&Yin:2003, Borkar:2008, Granichin&Amelina:2015} рассматривается метод стохастической аппроксимации с постоянным размером шага, в таком случае вместо последовательности $\{\alpha_n\}$ используется единственный параметр $\alpha \in \mathbb{R}^m$, и следующая оценка вычисляется по следующей формуле вместо \eqref{eq:spsa-central}:
\begin{equation}
	\hat{\theta}_n \leftarrow \hat{\theta}_{n - 1} - \alpha \Delta \frac{y_n^{+} - y_n^{-}}{2d}
\end{equation}

Это позволяет эффективно решать проблемы очень плохого начального приближения $\hat{\theta}_0$ и дрейфующей оптимальной точки, когда аргумент, при котором функционал принимает лучшие значения, меняется со временем.


%%%%%%%%%%%%%%%%%%%%%%%%%%%%%%%%%%%%%%%%%%%%%%%%%%%%%%%%%%%%%%%%%%%%%%%%%%%%%%%%%%%%%%%%%%%%%%%%%%%%%%%%%%%%%%%

\subsection{Постановка задачи адаптации параметров алгоритма}
\label{subsec:task}

Рандомизированный жадный алгоритм имеет один параметр $k$, в то время как схема кластеризации основных групп графа один параметр $s$ и различные начальные и финальные алгоритмы. В случае использования $RG_k$ в качестве начального алгоритма --- \emph{CGGC} можно рассматривать как алгоритм с параметрами $s,\;k$ и некоторым финальным алгоритмом.

Часто алгоритмы на разных входных данных имеют разные оптимальные параметры, то есть нет одного набора параметров, решающих каждую задачу наилучшим образом. Алгоритм \emph{SPSA} показал хорошие результаты в создании адаптивных модификаций алгоритмов, то есть модификаций, подстраивающихся под любые входные данные.

В выпускной квалификационной работе будет рассматриваться применение алгоритма \emph{SPSA} к двум алгоритмам выделения сообществ в графах, а так же сравнение модулярностей этих алгоритмов и полученных с помощью \emph{SPSA} модификаций.


%%%%%%%%%%%%%%%%%%%%%%%%%%%%%%%%%%%%%%%%%%%%%%%%%%%%%%%%%%%%%%%%%%%%%%%%%%%%%%%%%%%%%%%%%%%%%%%%%%%%%%%%%%%%%%%

\subsection{Тестовые графы}

В качестве тестовых графов были взяты графы, используемые для оценки алгоритмов на конкурсе 10th DIMACS Implementation Challenge. Далее представлены используемые графы с кратким описанием, в порядке увеличения количества узлов. Графы можно найти по адресу \url{http://www.cc.gatech.edu/dimacs10/archive/clustering.shtml}

\begin{itemize}
	
	\item \textbf{karate}, $N = 34,\;L = 78$: социальная сеть между 34 членами карате клуба с 1970 по 1972 год \cite{Zachary:1977}
	\item \textbf{chesapeake}, $N = 39,\;L = 170$: сеть мезогалинных вод Чесапикского залива \cite{Baird&Ulanowicz:1989}
	\item \textbf{dolphins}, $N = 62,\;L = 159$: социальная сеть частых общений между 62 дельфинами \cite{Lusseau&al:2003}
	\item \textbf{polbooks}, $N = 105,\;L = 441$: сеть книг о политике США, изданных во время президентских выборов 2004 года. Рёбра между книгами означают частые покупки одними и теми же покупателями. Сеть скомпилирована Валдисом Кребсом, однако не была опубликована
	\item \textbf{adjnoun}, $N = 112,\;L = 425$: сеть смежности частых прилагательных и существительных в романе <<Давид Копперфильд>> Чарльза Диккенса \cite{Newman:2006}
	\item \textbf{football}, $N = 115,\;L = 613$: сеть игр в американский футбол между колледжами из дивизиона IA во время регулярного сезона осенью 2000 года \cite{Girvan&Newman:2002}
	\item \textbf{jazz}, $N = 198,\;L = 2742$: сеть джазовых музыкантов \cite{Gleiser&Danon:2003}
	\item \textbf{\celegans}, $N = 453,\;L = 2025$: метаболическая сеть Caenorhabditis elegans \cite{Duch&Arenas:2005}
	\item \textbf{email}, $N = 1133,\;L = 5451$: сеть связей по электронной почте между членами Университета Ровира и Вирхилий \cite{Guimera&al:2003}
	\item \textbf{polblogs}, $N = 1490,\;L = 16715$: сеть ссылок между интернет блогами о политике США в 2005 году \cite{Adamic&Glance:2005}
	\item \textbf{netscience}, $N = 1589,\;L = 2742$: сеть соавторства среди учёных, работающих над сложными сетями \cite{Newman:2006}
	\item \textbf{PGPgiantcompo}, $N = 10680,\;L = 24316$: список узлов гигантской компоненты сети пользователей алгоритма Pretty-Good-Privacy для защищенного обмена информацией \cite{Boguna&al:2004}
	\item \textbf{as-22july06}, $N = 22963,\;L = 48436$: структура интернета на уровне автономных систем на 22 июля 2006 года. Сеть создана Марком Ньюманом и не опубликована
	\item \textbf{cond-mat-2003}, $N = 31163,\;L = 120029$: сеть соавторства среди учёных, публиковавших препринты в определённые архивы между 1995 и 2003 годами \cite{Newman:2001}
	\item \textbf{caidaRouterLevel}, $N = 192244,\;L =  609066$: граф структуры интернета на уровне роутера, собранный ассоциацией CAIDA в апреле и мае 2003 года
	\item \textbf{cnr-2000}, $N = 325557,\;L = 2738969$: небольшая часть обхода итальянского домена .cnr \cite{Boldi&Vigna:2004, Boldi&al:2011, Boldi&al:2004}
	\item \textbf{eu-2005}, $N = 862664,\;L = 16138468$: небольшая часть обхода домена Европейского союза .eu \cite{Boldi&Vigna:2004, Boldi&al:2011, Boldi&al:2004}
	\item \textbf{in-2004}, $N = 1382908,\;L = 13591473$: небольшая часть обхода индийского домена .in \cite{Boldi&Vigna:2004, Boldi&al:2011, Boldi&al:2004}

\end{itemize}

При разработке алгоритмов использовались синтетические наборы данных, которые представляют собой автоматически сгенерированные графы с заранее известным количеством сообществ. Такой граф имеет четыре параметра --- количество узлов $N$, количество сообществ $K$ (все сообщества одинаковых размеров), вероятность появления ребра между узлами одного сообщества $p_1$ и вероятность появления ребра между вершинами разных сообществ $p_2$. Преимущество автоматически сгенерированных графов заключается в проверке работы алгоритма на разных по размеру графах с разными по плотности сообществами. Так же для тестирования важно, что по построению известны реальные сообщества. Однако графы, построенные по реальным системам, могут сильно отличаться от подобных графов.

Так, далее в работе используется автоматически сгенерированный граф под названием \emph{auto40}, со следующими параметрами: $N = 40,000,\ K = 40,\ p_1 = 0.1,\ p_2 = 5\cdot 10^{-4}$.