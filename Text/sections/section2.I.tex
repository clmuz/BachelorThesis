%%%%%%%%%%%%%%%%%%%%%%%%%%%%%%%%%%%%%%%%%%%%%%%%%%%%%%%%%%%%%%%%%%%%%%%%%%%%%%%%%%%%%%%%%%%%%%%%%%%%%%%%%%%%%%%

\section{Адаптивный рандомизированный жадный алгоритм}
\label{sec:ARG}


%%%%%%%%%%%%%%%%%%%%%%%%%%%%%%%%%%%%%%%%%%%%%%%%%%%%%%%%%%%%%%%%%%%%%%%%%%%%%%%%%%%%%%%%%%%%%%%%%%%%%%%%%%%%%%%

\subsection{Применимость алгоритма SPSA}

Для того, чтобы алгоритм SPSA был применим --- необходимо иметь выпуклую функцию качества, которую необходимо минимизировать. В большинстве случаев модулярность результатов работы \emph{RG} на разных графах с разными значениями параметра $k$ будет выглядеть следующим образом:

\begin{figure}[H]
	\begin{tikzpicture}
		\begin{axis}[
		    table/col sep = semicolon,
		    height = 0.18\paperheight, 
		    width = 0.48\columnwidth,
		    xlabel = {$k$},
		    ylabel = {$Q$},
		    legend pos = north east,
		    no marks,
		    title = {karate},
		    cycle list name = diploma list
		]
		\addplot table [x={k}, y={q}]{data/arg/validity/karate.csv};
		\end{axis}
	\end{tikzpicture}
	\hskip 0.01\columnwidth
	\begin{tikzpicture}
		\begin{axis}[
		    table/col sep = semicolon,
		    height = 0.18\paperheight, 
		    width = 0.48\columnwidth,
		    xlabel = {$k$},
		    ylabel = {$Q$},
		    legend pos = south east,
		    no marks,
		    title = {chesapeake},
		    cycle list name = diploma list
		]
		\addplot table [x={k}, y={q}]{data/arg/validity/chesapeake.csv};
		\end{axis}
	\end{tikzpicture} % 2nd row
	\begin{tikzpicture}
		\begin{axis}[
		    table/col sep = semicolon,
		    height = 0.18\paperheight, 
		    width = 0.48\columnwidth,
		    xlabel = {$k$},
		    ylabel = {$Q$},
		    legend pos = south east,
		    no marks,
		    title = {polbooks},
		    cycle list name = diploma list
		]
		\addplot table [x={k}, y={q}]{data/arg/validity/polbooks.csv};
		\end{axis}
	\end{tikzpicture}
	\hskip 0.01\columnwidth
	\begin{tikzpicture}
		\begin{axis}[
		    table/col sep = semicolon,
		    height = 0.18\paperheight, 
		    width = 0.48\columnwidth,
		    xlabel = {$k$},
		    ylabel = {$Q$},
		    legend pos = south east,
		    no marks,
		    title = {netscience},
		    cycle list name = diploma list
		]
		\addplot table [x={k}, y={q}]{data/arg/validity/netscience.csv};
		\end{axis}
	\end{tikzpicture} % 3rd row
	\begin{tikzpicture}
		\begin{axis}[
		    table/col sep = semicolon,
		    height = 0.18\paperheight, 
		    width = 0.48\columnwidth,
		    xlabel = {$k$},
		    ylabel = {$Q$},
		    legend pos = north east,
		    no marks,
		    title = {email},
		    cycle list name = diploma list
		]
		\addplot table [x={k}, y={q}]{data/arg/validity/email.csv};
		\end{axis}
	\end{tikzpicture}
	\hskip 0.01\columnwidth
	\begin{tikzpicture}
		\begin{axis}[
		    table/col sep = semicolon,
		    height = 0.18\paperheight, 
		    width = 0.48\columnwidth,
		    xlabel = {$k$},
		    ylabel = {$Q$},
		    legend pos = south east,
		    no marks,
		    title = {cond-mat-2003},
		    cycle list name = diploma list
		]
		\addplot table [x={k}, y={q}]{data/arg/validity/cond-mat-2003.csv};
		\end{axis}
	\end{tikzpicture}
	\caption{Зависимость модулярности от $k$ в результатах работы $RG_k$ на шести тестовых графах}
	\label{fig:rg-k}
\end{figure}
\begin{figure}[H] % 4rd row
	\begin{tikzpicture}
		\begin{axis}[
		    table/col sep = semicolon,
		    height = 0.18\paperheight, 
		    width = 0.48\columnwidth,
		    xlabel = {$k$},
		    ylabel = {$Q$},
		    legend pos = south east,
		    no marks,
		    title = {in-2004},
		    cycle list name = diploma list
		]
		\addplot table [x={k}, y={q}]{data/arg/validity/in-2004.csv};
		\end{axis}
	\end{tikzpicture}
	\begin{tikzpicture}
		\begin{axis}[
		    table/col sep = semicolon,
		    height = 0.18\paperheight, 
		    width = 0.48\columnwidth,
		    xlabel = {$k$},
		    ylabel = {$Q$},
		    legend pos = north east,
		    no marks,
		    title = {eu-2005},
		    cycle list name = diploma list
		]
		\addplot table [x={k}, y={q}]{data/arg/validity/eu-2005.csv};
		\end{axis}
	\end{tikzpicture}
	\caption{Продолжение рисунка \ref{fig:rg-k}. Зависимость модулярности от $k$ в результатах работы $RG_k$ на графах \emph{in-2004} и \emph{eu-2005}}
	\label{fig:rg-k2}
\end{figure}

Значение $k$, при которым $RG_k$ будет на графе принимать наилучшее значение, далее в работе называется \emph{оптимальным} $k$ или $k_{opt}$.

Рисунки \ref{fig:rg-k} и \ref{fig:rg-k2} показывают разделение поведения \emph{RG} на разных графах на два возможных случая: в первом алгоритм принимает наилучший результат при некотором небольшом, но разным для разных графов $k_{opt}$ (например работа алгоритма на графе \emph{karate} и на графе \emph{eu-2005}). Во втором результаты алгоритма постепенно возрастают, приближаясь к некоторой асимптоте (хорошим примером будет работа алгоритма на графе \emph{netscience}). К первому случаю так же относится такое поведение, в котором алгоритм быстро (с ростом $k$) достигает своего лучшего значения, и затем его результаты очень несильно падают, и в дальнейшем держатся того же значения (такое выполняется, например, на графе \emph{in-2004}).

Похожее поведение алгоритм показывает на автоматически сгенерированных графах, но в таких графах можно получить менее предсказуемые значения $k_{opt}$ и можно предположить, что будет происходить с модулярностью при дальнейшем росте $k$.

\begin{figure}[H]
	\begin{tikzpicture}
		\begin{axis}[
		    table/col sep = semicolon,
		    height = 0.21\paperheight, 
		    width = 0.95\columnwidth,
		    xlabel = {$k$},
		    ylabel = {$Q$},
		    yticklabel style={/pgf/number format/fixed,
                  /pgf/number format/precision=3},
		    no marks,
		    cycle list name = diploma list
		]
		\addplot table [x={k}, y={q}]{data/arg/validity/auto13.csv};
		\end{axis}
	\end{tikzpicture}
	\caption{Зависимость модулярности от $k$ для разбиения на сообщества автоматически сгенерированного графа \emph{auto40} с параметрами $N = 40.000,\;K = 40,\;p_1 = 0.1,\;p_2 = 5\cdot 10^{-4}$ алгоритмом $RG_k$}
	\label{fig:q-auto13}
\end{figure}


%%%%%%%%%%%%%%%%%%%%%%%%%%%%%%%%%%%%%%%%%%%%%%%%%%%%%%%%%%%%%%%%%%%%%%%%%%%%%%%%%%%%%%%%%%%%%%%%%%%%%%%%%%%%%%%

\subsection{Функция качества}
\label{subsec:arg-f}

Таким образом, имеет смысл использовать в функции качества не только модулярность, но и время. Подсчёт времени сам по себе занимает время и в разных случаях может давать сильно отличающиеся результаты. Теоретическая трудоёмкость алгоритма линейно зависит от параметра $k$, на реальных графах зависимость тоже близка к линейной:

\begin{figure}[H]
	\begin{tikzpicture}
		\begin{axis}[
		    table/col sep = semicolon,
		    height = 0.18\paperheight, 
		    width = 0.495\columnwidth,
		    xlabel = {$k$},
		    ylabel = {$t$},
		    legend pos = south east,
		    no marks,
		    title = {polbooks},
		    cycle list name = diploma list
		]
		\addplot table [x={k}, y={time}]{data/arg/validity/polbooks.csv};
		\end{axis}
	\end{tikzpicture}
	\hskip 0.01\columnwidth
	\begin{tikzpicture}
		\begin{axis}[
		    table/col sep = semicolon,
		    height = 0.18\paperheight, 
		    width = 0.495\columnwidth,
		    xlabel = {$k$},
		    ylabel = {$t$},
		    legend pos = south east,
		    no marks,
		    title = {\celegans},
		    cycle list name = diploma list
		]
		\addplot table [x={k}, y={time}]{data/arg/validity/celegans_metabolic.csv};
		\end{axis}
	\end{tikzpicture}
	\caption{Зависимость времени $t$ в миллисекундах от $k$ для результатов работы $RG_k$ на графах \emph{polbooks} и \emph{\celegans}}
\end{figure}

Основываясь на этих данных вместо времени можно использовать значение $k$. Максимальное значение модулярности, которое может быть достигнуто на графе очень сильно отличается, поэтому нет смысла использовать абсолютное значение модулярности, но имеет смысл использовать относительные значения. При вычислении центральной точки \eqref{eq:spsa-central} в алгоритме одновременно возмущаемой стохастической аппроксимации использутся только разность функций качества. Поэтому если использовать в функции качества логарифмы от модулярности --- разность функций укажет, во сколько раз модулярность изменилась.

Так же функция качества должна принимать минимум, а не максимум, поэтому первой версией подобной функции может быть $f(Q) = -\alpha \ln Q,\;\alpha > 0$. Для того, чтобы принимать во внимание время работы, разумно добавить логарифм от $k$: 
\begin{equation} \label{eq:arg-f}
f(Q, k) = -\alpha (\ln Q - \beta \ln k),\;\alpha > 0, \beta \ge 0
\end{equation}
Коэффициент $\beta$ в таком случае можно рассматривать в следующем виде $\beta = \frac{\ln \gamma}{\ln 2}$, где коэффициент $\gamma$ указывает, во сколько раз необходимо увеличиться модулярности для того, чтобы покрыть увеличение времени (то есть $k$) вдвое. Если время не имеет значение коэффициент $\gamma$ принимает значение 1, а следовательно коэффициент $\beta$ принимает значение 0. Коэффициент $\alpha$ же играет роль размера шага при выборе следующей центральной точки.

\begin{figure}[H]
	\begin{tikzpicture}
		\begin{axis}[
		    table/col sep = semicolon,
		    height = 0.2\paperheight, 
		    width = 0.7\columnwidth,
		    xlabel = {$k$},
		    ylabel = {$F(Q, k)$},
		    legend style = {
		    	cells = {anchor = west},
		    	legend pos = outer north east
		    },
		    title = {karate},
		    cycle list name = diploma list
		 ]
		\addplot table [x={k}, y expr={-ln(\thisrowno{1})}]{data/arg/validity/karate.csv};
		\addplot table [x={k}, y expr={-ln(\thisrowno{1}) + 0.01 * ln(\thisrowno{0})}]{data/arg/validity/karate.csv};
		\addplot table [x={k}, y expr={-ln(\thisrowno{1}) + 0.02 * ln(\thisrowno{0})}]{data/arg/validity/karate.csv};
		\legend{{$-\ln Q$}, {$-(\ln Q - 0.01 \ln k)$}, {$-(\ln Q - 0.02 \ln k)$}}
		\end{axis}
	\end{tikzpicture}
	\begin{tikzpicture}
		\begin{axis}[
		    table/col sep = semicolon,
		    height = 0.2\paperheight, 
		    width = 0.7\columnwidth,
		    xlabel = {$k$},
		    ylabel = {$F(Q, k)$},
		    legend style = {
		    	cells = {anchor = west},
		    	legend pos = outer north east
		    },
		    title = {jazz},
		    no marks,
		    cycle list name = diploma list
		]
		\addplot table [x={k}, y expr={-ln(\thisrowno{1})}]{data/arg/validity/jazz.csv};
		\addplot table [x={k}, y expr={-ln(\thisrowno{1}) + 0.01 * ln(\thisrowno{0})}]{data/arg/validity/jazz.csv};
		\addplot table [x={k}, y expr={-ln(\thisrowno{1}) + 0.02 * ln(\thisrowno{0})}]{data/arg/validity/jazz.csv};
		\legend{{$-\ln Q$}, {$-(\ln Q - 0.01 \ln k)$}, {$-(\ln Q - 0.02 \ln k)$}}
		\end{axis}
	\end{tikzpicture}
	\begin{tikzpicture}
		\begin{axis}[
		    table/col sep = semicolon,
		    height = 0.2\paperheight, 
		    width = 0.7\columnwidth,
		    xlabel = {$k$},
		    ylabel = {$F(Q, k)$},
		    legend style = {
		    	cells = {anchor = west},
		    	legend pos = outer north east
		    },
		    title = {netscience},
		    no marks,
		    cycle list name = diploma list,
		    y label style={at={(axis description cs:-0.1,.5)}, anchor=south}
		]
		\addplot table [x={k}, y expr={-ln(\thisrowno{1})}]{data/arg/validity/netscience.csv};
		\addplot table [x={k}, y expr={-ln(\thisrowno{1}) + 0.1 * ln(\thisrowno{0})}]{data/arg/validity/netscience.csv};
		\addplot table [x={k}, y expr={-ln(\thisrowno{1}) + 0.2 * ln(\thisrowno{0})}]{data/arg/validity/netscience.csv};
		\legend{{$-\ln Q$}, {$-(\ln Q - 0.1 \ln k)$}, {$-(\ln Q - 0.2 \ln k)$}}
		\end{axis}
	\end{tikzpicture}
	\caption{Функции качества с разными коэффициентами $\beta$ для графов \emph{karate}, \emph{jazz} и \emph{netscience}}
	\label{fig:qua}
\end{figure}

Как видно из функций качества с разными коэффициентами $\beta$ для графа \emph{netscience} на рисунке \ref{fig:qua}, иногда для того, чтобы функция качества имела минимум на небольших $k$, надо задавать довольно большое значение коэффициента $\beta$. Однако это логично, на данном графе если нас устраивает время работы --- выгоднее всё время увеличивать значение $k$.

Так же заметно, что при хорошо подобранных $\beta$ функция качества выпукла и имеет минимум.


%%%%%%%%%%%%%%%%%%%%%%%%%%%%%%%%%%%%%%%%%%%%%%%%%%%%%%%%%%%%%%%%%%%%%%%%%%%%%%%%%%%%%%%%%%%%%%%%%%%%%%%%%%%%%%%

\subsection{Адаптивный алгоритм}

Для использования алгоритма \emph{SPSA} в рандомизированном жадном алгоритме предлагается разбить действие алгоритма на шаги длиной в $\sigma$ итераций. В течении каждого шага используется одно значение $k$. После каждого шага можно считать прирост модулярности, но вместо этого имеет смысл считать медиану прироста модулярности за $\sigma$ итераций --- так как алгоритм рандомизированный, время от времени будут появляться очень хорошие соединения сообществ, которые будут портить функцию качества, такой большой прирост может появиться даже при очень плохом $k$. В таком случае схема алгоритма будет выглядеть следующим образом:

\begin{enumerate}
	\item Выбор начальной центральной точки $\hat{k}_0 \in \mathbb{N}$, счётчик $n \leftarrow 0$, выбор размера возбуждения $d \in \mathbb{N}$, коэффициентов функции качества $\alpha, \beta \in \mathbb{R},\; \alpha > 0,\;\beta \ge 0$, и $\sigma \in \mathbb{N}$ --- количество итераций в одном шаге
	\item Увеличение счётчика $n \leftarrow n + 1$
	\item Определение новых аргументов функции $k_{n}^{-} \leftarrow \max\{\hat{k}_{n - 1} - d, 1\}$ и $k_{n}^{+} \leftarrow \hat{k}_{n - 1} + d$
	\item Выполнение $\sigma$ итераций с параметром $k_{n}^{-}$, вычисление медианы прироста модулярности $\mu_n^{-}$
	\item Выполнение $\sigma$ итераций с параметром $k_{n}^{+}$, вычисление медианы прироста модулярности $\mu_n^{+}$
	\item Вычисление функций качества $y_n^{-} \leftarrow -\alpha (\ln \mu_n^{-} - \beta \ln k_n^{-})$, $y_n^{+} \leftarrow -\alpha (\ln \mu_n^{+} - \beta \ln k_n^{+})$
	\item Вычисление следующей центральной точки
	\begin{equation} \label{eq:arg-centre}
		\hat{k}_n \leftarrow \max\left\{1, \left[\hat{k}_{n - 1} - \frac{y_n^{+} - y_n^{-}}{k_n^{+} - k_n^{-}}\right]\right\}
	\end{equation}
	\item Далее происходит переход на второй пункт
\end{enumerate}

Алгоритм заканчивает работу в тот момент, когда для рассмотрения осталось ровно одно сообщество. Далее в работе этот алгоритм называется \emph{адаптивным рандомизированным жадным алгоритмом} или \emph{ARG} (Adaptive Randomized Greedy).


%%%%%%%%%%%%%%%%%%%%%%%%%%%%%%%%%%%%%%%%%%%%%%%%%%%%%%%%%%%%%%%%%%%%%%%%%%%%%%%%%%%%%%%%%%%%%%%%%%%%%%%%%%%%%%%

\subsection{Чувствительность к перепадам функции качества}

Коэффициент $\alpha$ отвечает за то, насколько чувствителен алгоритм будет к перепадам функций качества --- чем больше $\alpha$, тем сильнее будет отличаться новая центральная точка от предыдущей в одной и той же ситуации. На трёх графах были измерены зависимости модулярности от параметра $\alpha$ при $d = 2,\ \sigma = 500,\ \hat{k}_0 = 10,\ \beta = 0$:

\begin{figure}[H]
	\begin{tikzpicture}
		\begin{axis}[
		    table/col sep = semicolon,
		    height = 0.14\paperheight, 
		    width = 0.95\columnwidth,
		    xlabel = {$\alpha$},
		    ylabel = {$Q$},
		    legend style = {
		    	cells = {anchor = west},
		    	legend pos = outer north east
		    },
		    title = {cond-mat-2003},
		    no marks,
		    cycle list name = diploma list
		]
		\addplot table [x={alpha}, y={q}]{data/arg/alpha/cond/cond_alpha.csv};
		\end{axis}
	\end{tikzpicture}
	\begin{tikzpicture}
		\begin{axis}[
		    table/col sep = semicolon,
		    height = 0.14\paperheight, 
		    width = 0.95\columnwidth,
		    xlabel = {$\alpha$},
		    ylabel = {$Q$},
		    legend style = {
		    	cells = {anchor = west},
		    	legend pos = outer north east
		    },
		    title = {caidaRouterLevel},
		    no marks,
		    cycle list name = diploma list
		]
		\addplot table [x={alpha}, y={q}]{data/arg/alpha/caida/caida_alpha.csv};
		\end{axis}
	\end{tikzpicture}
	\begin{tikzpicture}
		\begin{axis}[
		    table/col sep = semicolon,
		    height = 0.14\paperheight, 
		    width = 0.94\columnwidth,
		    xlabel = {$\alpha$},
		    ylabel = {$Q$},
		    legend style = {
		    	cells = {anchor = west},
		    	legend pos = outer north east
		    },
		    title = {cnr-2000},
		    no marks,
		    yticklabel style={/pgf/number format/fixed,
                  /pgf/number format/precision=3},
            y label style={at={(axis description cs:-0.08,.5)}, anchor=south},
		    cycle list name = diploma list
		]
		\addplot table [x={alpha}, y={q}]{data/arg/alpha/cnr/cnr_alpha.csv};
		\end{axis}
	\end{tikzpicture}
	\caption{Зависимость модулярности от параметра $\alpha$ в работе \emph{ARG} на графах \emph{cond-mat-2003}, \emph{caidaRouterLevel}, \emph{cnr-2000}}
	\label{fig:alpha}
\end{figure}

Из рисунка \ref{fig:alpha} видно, что результат не очень стабильный, хотя значение модулярности и колеблется на небольшом промежутке. Так же на графе \emph{cnr-2000} заметно, что увеличение параметра $\alpha$ влечёт за собой уменьшение модулярности в среднем.

\begin{figure}[H]
	\begin{tikzpicture}
		\begin{axis}[
		    table/col sep = semicolon,
		    height = 0.18\paperheight, 
		    width = 0.44\columnwidth,
		    xlabel = {$n$},
		    ylabel = {$k$},
		    y label style={at={(axis description cs:-0.1,.5)}, anchor=south},
		    title = {cond-mat-2003},
		    title style = {xshift = 3.5cm, yshift = 0.3cm},
		    no marks,
		    cycle list name = diploma list
		]
		\addplot table [x expr = {\thisrowno{0} / 1000}, y={k}]{data/arg/alpha/cond/cond10.csv};
		\addplot table [x expr = {\thisrowno{0} / 1000}, y={k}]{data/arg/alpha/cond/cond50.csv};
		\addplot table [x expr = {\thisrowno{0} / 1000}, y={k}]{data/arg/alpha/cond/cond100.csv};
		\end{axis}
	\end{tikzpicture}
	\hskip -2cm
	\begin{tikzpicture}
		\begin{axis}[
		    table/col sep = semicolon,
		    height = 0.18\paperheight, 
		    width = 0.44\columnwidth,
		    xlabel = {$n$},
		    ylabel = {$\mu$},
		    legend pos = outer north east,
		    y label style={at={(axis description cs:-0.1,.5)}, anchor=south},
		    no marks,
		    cycle list name = diploma list
		]
		\addplot table [x expr = {\thisrowno{0} / 1000}, y={mu}]{data/arg/alpha/cond/cond10.csv};
		\addplot table [x expr = {\thisrowno{0} / 1000}, y={mu}]{data/arg/alpha/cond/cond50.csv};
		\addplot table [x expr = {\thisrowno{0} / 1000}, y={mu}]{data/arg/alpha/cond/cond100.csv};
		\legend{{$\alpha = 10$}, {$\alpha = 50$}, {$\alpha = 100$}};
		\end{axis}
	\end{tikzpicture}
	\caption{Изменения параметра $k$ и медианы приросты модулярности со временем для разных параметров $\alpha$ на графе \emph{cond-mat-2003}}
	\label{fig:alpha1}
\end{figure}


\begin{figure}[H]
	\begin{tikzpicture}
		\begin{axis}[
		    table/col sep = semicolon,
		    height = 0.18\paperheight, 
		    width = 0.44\columnwidth,
		    xlabel = {$n$},
		    ylabel = {$k$},
		    y label style={at={(axis description cs:-0.13,.5)}, anchor=south},
		    title = {caidaRouterLevel},
		    title style = {xshift = 3.5cm, yshift = 0.3cm},
		    no marks,
		    cycle list name = diploma list
		]
		\addplot table [x expr = {\thisrowno{0} / 1000}, y={k}]{data/arg/alpha/caida/caida10s.csv};
		\addplot table [x expr = {\thisrowno{0} / 1000}, y={k}]{data/arg/alpha/caida/caida50s.csv};
		\addplot table [x expr = {\thisrowno{0} / 1000}, y={k}]{data/arg/alpha/caida/caida100s.csv};
		\end{axis}
	\end{tikzpicture}
	\hskip -2.5cm
	\begin{tikzpicture}
		\begin{axis}[
		    table/col sep = semicolon,
		    height = 0.18\paperheight, 
		    width = 0.44\columnwidth,
		    xlabel = {$n$},
		    ylabel = {$\mu$},
		    legend pos = outer north east,
		    y label style={at={(axis description cs:-0.13,.5)}, anchor=south},
		    no marks,
		    cycle list name = diploma list
		]
		\addplot table [x expr = {\thisrowno{0} / 1000}, y={mu}]{data/arg/alpha/caida/caida10s.csv};
		\addplot table [x expr = {\thisrowno{0} / 1000}, y={mu}]{data/arg/alpha/caida/caida50s.csv};
		\addplot table [x expr = {\thisrowno{0} / 1000}, y={mu}]{data/arg/alpha/caida/caida100s.csv};
		\legend{{$\alpha = 10$}, {$\alpha = 50$}, {$\alpha = 100$}};
		\end{axis}
	\end{tikzpicture}
	\begin{tikzpicture}
		\begin{axis}[
		    table/col sep = semicolon,
		    height = 0.18\paperheight, 
		    width = 0.44\columnwidth,
		    xlabel = {$n$},
		    ylabel = {$k$},
		    y label style={at={(axis description cs:-0.13,.5)}, anchor=south},
		    title = {cnr-2000},
		    title style = {xshift = 3.5cm, yshift = 0.3cm},
		    no marks,
		    cycle list name = diploma list
		]
		\addplot table [x expr = {\thisrowno{0} / 1000}, y={k}]{data/arg/alpha/cnr/cnr5s.csv};
		\addplot table [x expr = {\thisrowno{0} / 1000}, y={k}]{data/arg/alpha/cnr/cnr50s.csv};
		\addplot table [x expr = {\thisrowno{0} / 1000}, y={k}]{data/arg/alpha/cnr/cnr100s.csv};
		\end{axis}
	\end{tikzpicture}
	\hskip -1.7cm
	\begin{tikzpicture}
		\begin{axis}[
		    table/col sep = semicolon,
		    height = 0.18\paperheight, 
		    width = 0.44\columnwidth,
		    xlabel = {$n$},
		    ylabel = {$\mu$},
		    legend pos = outer north east,
		    y label style={at={(axis description cs:-0.12,.5)}, anchor=south},
		    no marks,
		    cycle list name = diploma list
		]
		\addplot table [x expr = {\thisrowno{0} / 1000}, y={mu}]{data/arg/alpha/cnr/cnr5s.csv};
		\addplot table [x expr = {\thisrowno{0} / 1000}, y={mu}]{data/arg/alpha/cnr/cnr50s.csv};
		\addplot table [x expr = {\thisrowno{0} / 1000}, y={mu}]{data/arg/alpha/cnr/cnr100s.csv};
		\legend{{$\alpha = 5$}, {$\alpha = 50$}, {$\alpha = 100$}};
		\end{axis}
	\end{tikzpicture}
	\caption{Продолжение рисунка \ref{fig:alpha1}. Изменения параметра $k$ и медианы приросты модулярности со временем для разных параметров $\alpha$ на графах \emph{caidaRouterLevel} и \emph{cnr-2000}. Графики смазаны для повышения читаемости}
	\label{fig:alpha2}
\end{figure}

Из рисунков \ref{fig:alpha1} и \ref{fig:alpha2} видно, что в зависимости от $\alpha$ параметр $k$, с которым прогоняются шаги по $\sigma$ шагов, и который поочередно принимает значения $k_n^{-}$ и $k_n^{+}$, меняется с разной интенсивностью, но прирост модулярности при этом изменяется примерно одинаково.
