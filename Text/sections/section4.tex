%%%%%%%%%%%%%%%%%%%%%%%%%%%%%%%%%%%%%%%%%%%%%%%%%%%%%%%%%%%%%%%%%%%%%%%%%%%%%%%%%%%%%%%%%%%%%%%%%%%%%%%%%%%%%%%

\section*{Заключение}

Были проанализированы современные методы выделения сообществ в графах и создано две модификации рандомизированных алгоритмов выделения сообществ в графах с помощью одновременно возмущаемой стохастической аппроксимации: адаптивный рандомизированный жадный алгоритм, описанный в разделе \ref{sec:ARG}, а также адаптивная схема кластеризации основных групп графа, описанная в разделе \ref{sec:ACGGC}. Для каждого алгоритма приведён анализ его параметров и результатов.

При этом адаптивный рандомизированный жадный алгоритм показал более стабильные результаты, чем его неадаптивный вариант (подраздел \ref{subsec:arg-res}), что может в том числе иметь применение, описанное в подразделе \ref{subsec:arg_in_cggc}.

Адаптивная схема кластеризации основных групп графа показала в среднем лучшие и более стабильные результаты, чем неадаптивная схема (подраздел \ref{subsec:acggc-res}). Также адаптивная схема имеет эффективный механизм снижения времени работы, описанный в подразделе \ref{subsec:acggc_time}.

В дальнейшних исследованиях области имеет смысл рассмотреть поведение параметров алгоритмов на больших графах, а так же использовать изменяющиеся со временем параметры, к примеру в итеративной адаптивной схеме кластеризации основных групп графа.