\section*{Введение}

Исторически, изучение сетей происходило в рамках теории графов, которая начала своё существование с решения Леонардом Эйлером задачи о кёнигсбергских мостах. В 1920-х взял своё начало анализ социальных сетей и лишь последние двадцать лет развивается изучение \emph{сложных сетей}, то есть сетей с неправильной, сложной структурой, в некоторых случаях рассматривают динамически меняющейся во времени сложные сети. От изучения маленьких сетей внимание переходит к сетям из тысяч или миллионов узлов.

В процессе изучения сложных систем, построенных по реальным системам, оказалось, что распределение степеней $P(s)$, определённое как доля узлов со степенью $s$ среди всех узлов графа, сильно отличается от распределения Пуассона, которое ожидается для случайных графов. Также сети, построенные по реальным системам характеризуются короткими путями между любыми двумя узлами и большим количеством маленьких циклов \cite{Boccaletti&al:2006}. Это показывает, что модели, предложенные теорией графов, не всегда будут хорошо работать для графов, построенных по реальным системам.

Современное изучение сложных сетей привнесло значительный вклад в понимание реальных систем. Сложные сети с успехом были применены в таких разных областях, как эпидемиология \cite{Moore&Newman:2000}, биоинформатика \cite{Zhao&al:2006}, поиск преступников \cite{Hong&al:2009}, социология \cite{Scott:2012}, изучение структуры и топологии интернета \cite{Faloutsos&al:1999, Broder&al:2000} и в многих других.

Свойством, присутствуещим почти у любой сети, является структура \emph{сообществ}, разделение узлов сети на разные группы узлов так, чтобы внутри каждой группы соединений между узлами много, а соединений между узлами разных групп мало. Плотно связанные группы узлов в социальных сетях представляют людей, принадлежащих социальным сообществам, плотно сплочённые группы узлов в интернете соответствуют страницам, посвящённым распространённым темам, а сообщества в генетических сетях связаны с функциональными модулями \cite{Boccaletti&al:2006}. Способность находить и анализировать подобные группы предоставляет большие возможности в изучении реальных систем, представленных с помощью сложных сетей.

В разделе \ref{sec:intro} будут рассмотрены существующие методы выделения сообществ в графах, а так же другая необходимая информация о графах и сложных сетях. Так же в разделе будет кратко описана одновременно возмущаемая стохастическая аппроксимация, а в подразделе \ref{subsec:task} будет поставлена задача. Затем в разделе \ref{sec:ARG} будет представлен адаптивный рандомизированный жадный алгоритм, его анализ и сравнение с существующими алгоритмами, возможное применение в подразделе \ref{subsec:arg_in_cggc}. В разделе \ref{sec:ACGGC} будет представлена адаптивная схема кластеризации основных групп графа, анализ и сравнение с существующими алгоритмами.