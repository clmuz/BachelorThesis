%%%%%%%%%%%%%%%%%%%%%%%%%%%%%%%%%%%%%%%%%%%%%%%%%%%%%%%%%%%%%%%%%%%%%%%%%%%%%%%%%%%%%%%%%%%%%%%%%%%%%%%%%%%%%%%

\section{Адаптивная схема кластеризации основных групп графа}
\label{sec:ACGGC}


%%%%%%%%%%%%%%%%%%%%%%%%%%%%%%%%%%%%%%%%%%%%%%%%%%%%%%%%%%%%%%%%%%%%%%%%%%%%%%%%%%%%%%%%%%%%%%%%%%%%%%%%%%%%%%%

\subsection{ARG в схеме кластеризации основных групп графа}
\label{subsec:arg_in_cggc}

Схема кластеризации основных групп графа (\emph{CGGC}) была более подробно описана в подразделе \ref{subsec:ens}, однако общая схема работы заключается в том, что сначала $s$ начальных алгоритмов выделяют сообщества, а те узлы, в которых они разошлись по мнению, затем разбивает на сообщества финальный алгоритм. Трудоёмкость таким образом складывается из $s$ трудоёмкостей начальных алгоритмов и трудоёмкости финального алгоритма. Поэтому имеет смысл в качестве начальных алгоритмов брать алгоритмы, работающие быстро. Однако в случае, если начальные алгоритмы очень плохие, то есть дают разбиения не лучше случайного --- модулярность разбиений, полученных \emph{CGGC} будет приблизительно равна модулярности разбиений, полученных финальным алгоритмом.

Далее разбиение, полученное начальными алгоритмами называется \emph{начальным} разбиением, полученное финальным алгоритмом --- \emph{финальным} разбиением. Также  разбиение, равное максимальному перекрытию начальных разбиений далее называется \emph{промежуточным}.

Если промежуточное разбиение состоит из достаточно большого количества сообществ (как минимум в несколько раз больше $2\sigma$) --- на нём разумно использовать в качестве финального алгоритма \emph{ARG}, чтобы получать стабильные результаты. Однако на небольшом количестве узлов или сообществ \emph{ARG} не имеет смысла.

Так же, если граф, подающийся на вход, достаточно большой --- в качестве начального алгоритма на нём имеет смысл использовать адаптивный рандомизированный жадный алгоритм.

В таблице \ref{tab:es1-q} сравниваются результаты работы \emph{CGGC} с разными начальными и финальными алгоритмами. В качестве начального алгоритма используюся два рандомизированных жадных алгоритма с параметрами $k = 3$ и $k = 10$ (обозначаются как и выше в работе $RG_k$, со значением параметра $k$ в индексе), и адаптивный рандомизированный жадный алгоритм с параметрами $d = 5,\ f(\mu, k) = -10(\ln \mu - 0.02 \ln k),\ \sigma = 1000,\ \hat{k}_0 = 8$ (обозначается $ARG$). А в качестве финального алгоритма используются только $RG_3$ и $RG_{10}$, так как на рассматриваемых графах промежуточное разбиение состоит из недостаточно большого количества сообществ, чтобы использовать \emph{ARG}.

\emph{CGGC} используется с параметром $s = 10$, начальный алгоритм обозначается как $A_{init}$, а финальный как $A_{final}$. Для повышения точности \emph{CGGC} с разными начальными и финальными алгоритмами запускались на графе \emph{cond-mat-2003} запускались 21 раз, на \emph{caidaRouterLevel} и \emph{auto40} --- три раза.

\begin{table}[H]
	\caption{Модулярность разбиений, полученных в результате работы \emph{CGGC} с начальным алгоритмом $A_{init}$ и финальным алгоритмом $A_{final}$ на трёх графах}
	\label{tab:es1-q}
	\begin{tabularx}{\textwidth}{XR{1.5cm}R{1.5cm}R{1.5cm}R{1.5cm}R{1.5cm}R{1.5cm}} \hline
	$A_{init}$	& \multicolumn{2}{c}{$RG_3$} & \multicolumn{2}{c}{$RG_{10}$} & \multicolumn{2}{c}{$ARG$} \\
	$A_{final}$ & \multicolumn{1}{c}{$RG_3$} & \multicolumn{1}{c}{$RG_{10}$} & \multicolumn{1}{c}{$RG_3$} & \multicolumn{1}{c}{$RG_{10}$} & \multicolumn{1}{c}{$RG_3$} & \multicolumn{1}{c}{$RG_{10}$} \\\hline
	cond-mat-2003 		& 0.16840	& 0.71155	& 0.44934	& 0.74794	& 0.42708	& 0.74872	\\
	auto40				& 0.80628	& 0.80645	& 0.80633	& 0.80645	& 0.80628	& 0.80647	\\
	caidaRouterLevel	& 0.84078	& 0.85372	& 0.84031	& 0.84448	& 0.83671	& 0.85279	\\\hline
	\end{tabularx}
\end{table}

\begin{table}[H]
	\caption{Время работы \emph{CGGC} с начальным алгоритмом $A_{init}$ и финальным алгоритмом $A_{final}$ на трёх графах}
	\label{tab:es1-t}
	\begin{tabularx}{\textwidth}{XR{1.5cm}R{1.5cm}R{1.5cm}R{1.5cm}R{1.5cm}R{1.5cm}} \hline
	$A_{init}$	& \multicolumn{2}{c}{$RG_3$} & \multicolumn{2}{c}{$RG_{10}$} & \multicolumn{2}{c}{$ARG$} \\
	$A_{final}$ & \multicolumn{1}{c}{$RG_3$} & \multicolumn{1}{c}{$RG_{10}$} & \multicolumn{1}{c}{$RG_3$} & \multicolumn{1}{c}{$RG_{10}$} & \multicolumn{1}{c}{$RG_3$} & \multicolumn{1}{c}{$RG_{10}$} \\\hline
	cond-mat-2003 		& 2.0	& 2.3	& 4.7	& 4.8	& 4.8	& 4.9	\\
	auto40				& 47.0	& 46.6	& 57.7	& 57.2	& 65.7	& 65.8	\\
	caidaRouterLevel	& 94.0	& 93.7	& 104.2	& 104.3	& 115.1	& 118.5	\\\hline
	\end{tabularx}
\end{table}

Граф \emph{cond-mat-2003} плохо разбивается случайными жадными алгоритмами с маленьким $k$, если обратить внимание на таблицу \ref{tab:es1-q} --- в случаях, когда $RG_3$ используется на этом графе в качестве начального алгоритма --- модулярность \emph{CGGC} сравнима с модулярностью финального алгоритма на этом графе (модулярности $RG_3$ и $RG_{10}$ на разных графах можно увидеть в таблице \ref{tab:arg-res-q}). В случаях, когда $RG_3$ используется в качестве финального алгоритма --- модулярность получается сравнительно плохой.

Заранее неизвестно, при каких $k$ алгоритм будет хорошо работать, а при каких плохо, кроме того неизвестно по результату, хорошее ли это разбиение для конкретного графа или нет. \emph{CGGC} работает достаточно долго, поэтому запускать \emph{CGGC} с разными начальными алгоритмами несколько раз для определения хороших параметров не выгодно. Таким образом, выгодно использовать $ARG$ и в качестве начального алгоритма, и в качестве финального, если промежуточное разбиение подходит по размерам.


%%%%%%%%%%%%%%%%%%%%%%%%%%%%%%%%%%%%%%%%%%%%%%%%%%%%%%%%%%%%%%%%%%%%%%%%%%%%%%%%%%%%%%%%%%%%%%%%%%%%%%%%%%%%%%%

\subsection{Адаптивное построение промежуточного разбиения}
\label{subsec:es-adaptive}

Так как адаптивный рандомизированный жадный алгоритм не может работать на графах с маленьким количеством узлов или на разбиениях с маленьким количеством сообществ --- была предложена другая схема адаптивной кластеризации основных групп графа:

\begin{enumerate}
	\item На вход подаётся граф $G$. Выбор параметров алгоритма: $d \in \mathbb{N}$ --- размер возмущения, $f(Q, k): \mathbb{R} \times \mathbb{N} \rightarrow \mathbb{R}$ --- функция качества, $l \in \mathbb{N}$ --- количество шагов, $r \in (0, 1]$ --- параметр, отвечающий за количество лучших начальных разбиений, учавствующих в создании промежуточного разбиения. И наконец $\hat{k}_0 \in \mathbb{N}$ --- начальная центральная точка. Инициализация счётчика $n \leftarrow 0$и списка начальных разбиений $S \leftarrow \emptyset$
	\item Увеличение счётчика $n \leftarrow n + 1$
	\item \label{item:es-k+} Вычисление следующих параметров рандомизированного жадного алгоритма $k_n^{-} \leftarrow \max\{1, \hat{k}_{n - 1} - d\}$ и $k_n^{+} \leftarrow \hat{k}_{n - 1} + d$
	\item Выделение сообществ в графе $G$ рандомизированным жадным алгоритмом с параметром $k = k_n^{-}$, результирующее разбиение $P_n^{-}$ записывается в список $S$
	\item Выделение сообществ в графе $G$ рандомизированным жадным алгоритмом с параметром $k = k_n^{+}$, результирующее разбиение $P_n^{+}$ записывается в список $S$
	\item Вычисление функций качества $y_n^{-} \leftarrow f(Q_n^{-}, k_n^{-}),\ y_n^{+} \leftarrow f(Q_n^{+}, k_n^{+})$
	\item Вычисление следующей центральной точка аналогично \eqref{eq:arg-centre}
	\begin{equation}\label{eq:aes-centre}
		\hat{k}_n \leftarrow \max\left\{1, \left[\hat{k}_{n - 1} - \frac{y_n^{+} - y_n^{-}}{k_n^{+} - k_n^{-}}\right]\right\}
	\end{equation}
	\item Если $n \ne l$ --- переход на второй пункт, иначе --- на следующий
	\item Существует два способа выбора лучшие разбиения: \emph{пропорциональный выбор}, по которому выбираются $\lceil r|S| \rceil = \lceil 2rl \rceil$ лучших разбиений (с наибольшей модулярностью).

	Второй, \emph{непропорциональный}, выбор заключается в отбрасывании всех разбиений на сообщества с модулярностью $Q \le (1 - r)Q_{best}$, где $Q_{best}$ --- модулярность лучшего разбиения в $S$. Из выбранных разбиений создается максимальное разбиение, которое затем передаётся в финальный алгоритм
\end{enumerate}

При этом граф $G$ может быть иметь сколь угодно маленькое количество узлов.

Далее в работе схема кластеризации основных групп графа с адаптивным построением промежуточного разбиения будет называться \emph{адаптивной схемой кластеризации основных групп графа} (\emph{ACGGC}).


%%%%%%%%%%%%%%%%%%%%%%%%%%%%%%%%%%%%%%%%%%%%%%%%%%%%%%%%%%%%%%%%%%%%%%%%%%%%%%%%%%%%%%%%%%%%%%%%%%%%%%%%%%%%%%%

\subsection{Размер возмущения и чувствительность к перепадам функции качества}

Функцией качества для построения промежуточного разбиения выбрана $f(Q, k) = -\alpha(\ln Q - \beta \ln k)$. Мотивация этого выбора аналогична выбору функции качества для $ARG$, описанная в подразделе \ref{subsec:arg-f}. Аналогично, параметр $\alpha$ указывает на чувствительность алгоритма к разностям значений функции качества, а параметр $\beta$ указывает на значимость времени работы алгоритма.

На рисунке \ref{fig:es-alpha-d} изображены результаты адаптивной стратегии с параметром $\beta = 0$ и остальными параметрами равными $f(Q, k) = -\alpha \ln Q,\ \hat{k}_0 = 5,\ l = 10,\ r = 1$. В качестве финального алгоритма используется $RG_{10}$.

\begin{figure}[H]
	\begin{tikzpicture}
		\begin{semilogxaxis}[
		    table/col sep = semicolon,
		    height = 0.15\paperheight, 
		    width = 0.83\columnwidth,
		    xlabel = {$\alpha$},
		    ylabel = {$Q$},
		    legend pos = outer north east,
		    no marks,
		    title = {\celegans},
		    yticklabel style={/pgf/number format/fixed,
                  /pgf/number format/precision=3},
            y label style={at={(axis description cs:-0.1,.5)}, anchor=south},
		    cycle list name = diploma list,
		]
		\addplot table [x={alpha}, y={q}]{data/es/alphad/celegans_1.csv};
		\addplot table [x={alpha}, y={q}]{data/es/alphad/celegans_2.csv};
		\legend{{$d = 1$}, {$d = 2$}};
		\end{semilogxaxis}
	\end{tikzpicture}
	\begin{tikzpicture}
		\begin{semilogxaxis}[
		    table/col sep = semicolon,
		    height = 0.15\paperheight, 
		    width = 0.83\columnwidth,
		    xlabel = {$\alpha$},
		    ylabel = {$Q$},
		    legend pos = outer north east,
		    no marks,
		    title = {jazz},
		    yticklabel style={/pgf/number format/fixed,
                  /pgf/number format/precision=4},
            y label style={at={(axis description cs:-0.1,.5)}, anchor=south},
		    cycle list name = diploma list,
		]
		\addplot table [x={alpha}, y={q}]{data/es/alphad/jazz_1.csv};
		\addplot table [x={alpha}, y={q}]{data/es/alphad/jazz_2.csv};
		\legend{{$d = 1$}, {$d = 2$}};
		\end{semilogxaxis}
	\end{tikzpicture}
	\begin{tikzpicture}
		\begin{semilogxaxis}[
		    table/col sep = semicolon,
		    height = 0.15\paperheight, 
		    width = 0.83\columnwidth,
		    xlabel = {$\alpha$},
		    ylabel = {$Q$},
		    legend pos = outer north east,
		    no marks,
		    title = {as-22july06},
		    yticklabel style={/pgf/number format/fixed,
                  /pgf/number format/precision=3},
            y label style={at={(axis description cs:-0.1,.5)}, anchor=south},
		    cycle list name = diploma list,
		]
		\addplot table [x={alpha}, y={q}]{data/es/alphad/july_1.csv};
		\addplot table [x={alpha}, y={q}]{data/es/alphad/july_2.csv};
		\legend{{$d = 1$}, {$d = 2$}};
		\end{semilogxaxis}
	\end{tikzpicture}
	\begin{tikzpicture}
		\begin{semilogxaxis}[
		    table/col sep = semicolon,
		    height = 0.15\paperheight, 
		    width = 0.83\columnwidth,
		    xlabel = {$\alpha$},
		    ylabel = {$Q$},
		    legend pos = outer north east,
		    no marks,
		    title = {netscience},
		    cycle list name = diploma list,
		]
		\addplot table [x={alpha}, y={q}]{data/es/alphad/netscience_1.csv};
		\addplot table [x={alpha}, y={q}]{data/es/alphad/netscience_2.csv};
		\legend{{$d = 1$}, {$d = 2$}};
		\end{semilogxaxis}
	\end{tikzpicture}
	\caption{Зависимость модулярности от параметра $\alpha$ при $d = 1$ и $d = 2$ в работе \emph{ACGGC} на четырёх графах}
	\label{fig:es-alpha-d}
\end{figure}

Значения модулярности колеблются в очень небольшом промежутке и на вышеприведённых четырёх графах значение $\alpha$, примерно равное $10^3$ даёт в среднем лучшие результаты. При этом на графах \emph{\celegans}, \emph{jazz} и \emph{netscience} зависимости модулярности от $\alpha$ выглядят одинакого для $d = 1$ и $d = 2$, а на графе \emph{as-22july06} зависимость при параметре $d = 2$ лежит в среднем выше, чем при параметре $d = 1$. Б\'{о}льшие значения параметра $d$ чреваты тем, что алгоритм будет слишком сильно отклоняться от оптимального значения параметра $k_{opt}$ (то есть при котором \emph{RG} в среднем будет получать лучшие результаты).

На рисунке \ref{fig:es-alpha-d3} представлено изменение $k^{-}_n$ и $k^{+}_n$ в ходе построения промежуточного разбиения от номера шага при $\alpha = 994$ на графе \emph{as-22july06} и $\alpha = 1020$ на графе \emph{netscience}, в обоих случаях $d = 2$, а остальные параметры те же, что и на рисунке \ref{fig:es-alpha-d}. Отдельно измерено, что наилучшее в среднем значение рандомизированный жадный алгоритм даст при $k = 1$ на графе \emph{as-22july06} и при очень больших $k$ на графе \emph{netscience}.

\begin{figure}[H]
	\begin{tikzpicture}
		\begin{axis}[
		    table/col sep = semicolon,
		    height = 0.16\paperheight, 
		    width = 0.49\columnwidth,
		    xlabel = {$n$},
		    ylabel = {$k^{\pm}_n$},
		    legend pos = north east,
		    no marks,
		    title = {as-22july06},
		    cycle list name = diploma list,
		]
		\addplot table [x expr = {\thisrowno{0} / 2}, y={k}]{data/es/alphad/_july994.csv};
		\end{axis}
	\end{tikzpicture}
	\hskip 0.01\columnwidth
	\begin{tikzpicture}
		\begin{axis}[
		    table/col sep = semicolon,
		    height = 0.16\paperheight, 
		    width = 0.49\columnwidth,
		    xlabel = {$n$},
		    ylabel = {$k^{\pm}_n$},
		    legend pos = north east,
		    no marks,
		    title = {netscience},
		    cycle list name = diploma list,
		]
		\addplot table [x expr = {\thisrowno{0} / 2}, y={k}]{data/es/alphad/_netscience1020.csv};
		\end{axis}
	\end{tikzpicture}
	\caption{Изменение $k^{-}_n$ и $k^{+}_n$ с изменением номера шага в работе \emph{ACGGC} на графах \emph{as--22july06} и \emph{netscience}}
	\label{fig:es-alpha-d3}
\end{figure}

Заметно, как быстро алгоритмы находят близкие к $k_{opt}$ центральные точки $\hat{k}_n$.


%%%%%%%%%%%%%%%%%%%%%%%%%%%%%%%%%%%%%%%%%%%%%%%%%%%%%%%%%%%%%%%%%%%%%%%%%%%%%%%%%%%%%%%%%%%%%%%%%%%%%%%%%%%%%%%

\subsection{Начальная центральная точка}

Начальная центральная точка $\hat{k}_0$ указывает, рядом с какой точкой будут находиться параметры первых двух запусков рандомизированного жадного алгоритма. Очень большим делать начальную центральную точку не имеет смысла, так как довольно много графов имеет небольшой оптимальный параметр.

Однако, как показывает рисунок \ref{fig:es-k0}, на самом деле модулярность очень слабо зависит от начальной центральной точки. Для получения рисунка \emph{ACGGC} запускалась с параметрами $d = 2,\ f(Q, k) = -1000 \ln Q,\ l = 10,\ r = 1$, в качестве финального алгоритма использовался $RG_{10}$.

На графе \emph{as-22july06} при увеличении начальной центральной точки модулярность понижалась, однако изменения были достаточно небольшими, а так же на других графах такой зависимости выявлено не было.

\begin{figure}[H]
	\begin{tikzpicture}
		\begin{axis}[
		    table/col sep = semicolon,
		    height = 0.145\paperheight,
		    width = 0.465\columnwidth,
		    xlabel = {$\hat{k}_0$},
		    ylabel = {$Q$},
		    no marks,
		    title = {\celegans},
		    y label style={at={(axis description cs:-0.2,.5)}, anchor=south},
		    yticklabel style={/pgf/number format/fixed,
                  /pgf/number format/precision=4},
		    cycle list name = diploma list,
		]
		\addplot table [x = {k0}, y={q}]{data/es/k0/celegans_k0.csv};
		\end{axis}
	\end{tikzpicture}
	\begin{tikzpicture}
		\begin{axis}[
		    table/col sep = semicolon,
		    height = 0.145\paperheight,
		    width = 0.465\columnwidth,
		    xlabel = {$\hat{k}_0$},
		    ylabel = {$Q$},
		    no marks,
		    title = {jazz},
		    y label style={at={(axis description cs:-0.22,.5)}, anchor=south},
		    yticklabel style={/pgf/number format/fixed,
                  /pgf/number format/precision=4},
		    cycle list name = diploma list,
		]
		\addplot table [x = {k0}, y={q}]{data/es/k0/jazz_k0.csv};
		\end{axis}
	\end{tikzpicture}\\ %	\hskip 10pt
	\begin{tikzpicture}
		\begin{axis}[
		    table/col sep = semicolon,
		    height = 0.145\paperheight,
		    width = 0.465\columnwidth,
		    xlabel = {$\hat{k}_0$},
		    ylabel = {$Q$},
		    no marks,
		    title = {as-22july06},
		    y label style={at={(axis description cs:-0.2,.5)}, anchor=south},
		    yticklabel style={/pgf/number format/fixed,
                  /pgf/number format/precision=3},
		    cycle list name = diploma list,
		]
		\addplot table [x = {k0}, y={q}]{data/es/k0/july_k0.csv};
		\end{axis}
	\end{tikzpicture}
	\begin{tikzpicture}
		\begin{axis}[
		    table/col sep = semicolon,
		    height = 0.145\paperheight,
		    width = 0.465\columnwidth,
		    xlabel = {$\hat{k}_0$},
		    ylabel = {$Q$},
		    no marks,
		    title = {netscience},
		    y label style={at={(axis description cs:-0.22,.5)}, anchor=south},
		    yticklabel style={/pgf/number format/fixed,
                  /pgf/number format/precision=4},
		    cycle list name = diploma list,
		]
		\addplot table [x = {k0}, y={q}]{data/es/k0/netscience_k0.csv};
		\end{axis}
	\end{tikzpicture}
	\caption{Зависимость модулярности от начальной центральной точки $\hat{k}_0$ в работе \emph{ACGGC} на четырёх графах}
	\label{fig:es-k0}
\end{figure}


%%%%%%%%%%%%%%%%%%%%%%%%%%%%%%%%%%%%%%%%%%%%%%%%%%%%%%%%%%%%%%%%%%%%%%%%%%%%%%%%%%%%%%%%%%%%%%%%%%%%%%%%%%%%%%%

\subsection{Количество шагов и доля хороших разбиений}

Множество $S$ состоит из $2l$ разбиений, из которых затем выбирается $l_2$ лучших разбиений двумя способами в зависимости от параметра $r$. Пропорциональный выбор заключается в отборе ровно $\lceil 2rl \rceil$ лучших разбиений, а непропорциональный выбор заключается в отборе всех разбиений, у которых модулярность строго больше $(1 - r) Q_{best}$, где $Q_{best}$ --- модулярность лучшего разбиения в множестве $S$.

Затем $l_2$ лучших разбиений учавствуют в создании промежуточного разбиения. Предполагается, что при слишком маленьких $l$ алгоритм не успеет найти оптимальную точку $k$, в то время как при $r \rightarrow 1$ в создании промежуточного разбиения примут участие много плохих разбиений, полученных на первых шагах. Зависимость модулярности от доли хороших разбиений при $d = 2,\ f(Q, k) = -1000 \ln Q,\ \hat{k}_0 = 5,\ l = 10$ и финальном алгоритме $RG_{10}$ изображена на рисунке \ref{fig:es-r}.

Из рисунка видно, что маленькие $r$ в пропорциональном выборе дают плохие разбиения, что логично, так как в этом случае выбирается очень мало разбиений. В то время как в непропорциональном выборе даже при маленьких $r$ (на рисунке \ref{fig:es-r} наименьшее значение $r = 0.05$) значения достаточно хорошие. Это обусловлено тем, что есть достаточно много разбиений, близких к лучшему. Затем на разных графах алгоритмы ведут себя по разному, хотя и разница между модулярностями при разных значениях $r$ небольшия. Однако непропорциональный выбор ведёт себя более предсказумо на разных графах, поэтому такой выбор более рекомендуем и далее в работе будет рассматриваться только непропорциональный выбор лучших разбиений.

\begin{figure}[H]
	\begin{tikzpicture}
		\begin{axis}[
		    table/col sep = semicolon,
		    height = 0.16\paperheight,
		    width = 0.45\columnwidth,
		    xlabel = {$r$},
		    ylabel = {$Q$},
		    no marks,
		    title = {\celegans},
		    y label style={at={(axis description cs:-0.2,.5)}, anchor=south},
		    yticklabel style={/pgf/number format/fixed,
                  /pgf/number format/precision=3},
		    cycle list name = diploma list,
		]
		\addplot table [x = {r}, y={q}]{data/es/r-r1/celegans_r.csv};
		\addplot table [x expr = {1 - \thisrowno{0}}, y={q}]{data/es/r-r1/celegans_r1.csv};
		\end{axis}
	\end{tikzpicture}
	\hskip 5pt
	\begin{tikzpicture}
		\begin{axis}[
		    table/col sep = semicolon,
		    height = 0.16\paperheight,
		    width = 0.45\columnwidth,
		    xlabel = {$r$},
		    ylabel = {$Q$},
		    no marks,
		    title = {jazz},
		    y label style={at={(axis description cs:-0.2,.5)}, anchor=south},
		    yticklabel style={/pgf/number format/fixed,
                  /pgf/number format/precision=3},
		    cycle list name = diploma list,
		]
		\addplot table [x = {r}, y={q}]{data/es/r-r1/jazz_r.csv};
		\addplot table [x expr = {1 - \thisrowno{0}}, y={q}]{data/es/r-r1/jazz_r1.csv};
		\end{axis}
	\end{tikzpicture}\\
	\begin{tikzpicture}
		\begin{axis}[
		    table/col sep = semicolon,
		    height = 0.16\paperheight,
		    width = 0.45\columnwidth,
		    xlabel = {$r$},
		    ylabel = {$Q$},
		    no marks,
		    title = {as-22july06},
		    y label style={at={(axis description cs:-0.2,.5)}, anchor=south},
		    yticklabel style={/pgf/number format/fixed,
                  /pgf/number format/precision=3},
		    cycle list name = diploma list,
		]
		\addplot table [x = {r}, y={q}]{data/es/r-r1/july_r.csv};
		\addplot table [x expr = {1 - \thisrowno{0}}, y={q}]{data/es/r-r1/july_r1.csv};
		\end{axis}
	\end{tikzpicture}
	\hskip 17pt
	\begin{tikzpicture}
		\begin{axis}[
		    table/col sep = semicolon,
		    height = 0.16\paperheight,
		    width = 0.45\columnwidth,
		    xlabel = {$r$},
		    ylabel = {$Q$},
		    no marks,
		    title = {netscience},
		    y label style={at={(axis description cs:-0.1,.5)}, anchor=south},
		    yticklabel style={/pgf/number format/fixed,
                  /pgf/number format/precision=3},
		    cycle list name = diploma list,
		]
		\addplot table [x = {r}, y={q}]{data/es/r-r1/netscience_r.csv};
		\addplot table [x expr = {1 - \thisrowno{0}}, y={q}]{data/es/r-r1/netscience_r1.csv};
		\end{axis}
	\end{tikzpicture}
	\caption{Зависимость модулярности от доли хороших разбиений, учавствующих в создании промежуточного разбиения, в работе \emph{ACGGC} на четырёх графах. Сплошной линией отмечен пропорциональный выбор, а прерывистой --- непропорциональный}		% Изменять при изменении цветовой схемы
	\label{fig:es-r}
\end{figure}


На рисунках \ref{fig:es-l-t} и \ref{fig:es-l-q} изображены зависимость времени работы и модулярности от количества шагов $l$ при $r = 0.05$ и $r = 1$. Остальные параметры при этом были равны $d = 2,\;f(Q, k) = -1000 \ln Q,\;\hat{k}_0 = 5$. В качестве финального алгоритма, как и в предыдущих примерах, использовался $RG_{10}$.

\begin{figure}[H]
	\begin{tikzpicture}
		\begin{axis}[
		    table/col sep = semicolon,
		    height = 0.16\paperheight,
		    width = 0.43\columnwidth,
		    xlabel = {$l$},
		    ylabel = {$t$},
		    legend pos = north east,
		    no marks,
		    title = {\celegans},
		    y label style={at={(axis description cs:-0.12,.5)}, anchor=south},
		    yticklabel style={/pgf/number format/fixed,
                  /pgf/number format/precision=4},
            cycle list name = diploma list,
		]
		\addplot table [x = {l}, y={time}]{data/es/l/celegans_l_0.05.csv};
		\addplot table [x = {l}, y={time}]{data/es/l/celegans_l_1.csv};
		\end{axis}
	\end{tikzpicture}
	\hskip 3pt
	\begin{tikzpicture}
		\begin{axis}[
		    table/col sep = semicolon,
		    height = 0.16\paperheight,
		    width = 0.43\columnwidth,
		    xlabel = {$l$},
		    ylabel = {$t$},
		    legend style = {
		    	cells = {anchor = west},
		    	legend pos = outer north east
		    },
		    no marks,
		    title = {jazz},
		    y label style={at={(axis description cs:-0.12,.5)}, anchor=south},
		    yticklabel style={/pgf/number format/fixed,
                  /pgf/number format/precision=4},
            cycle list name = diploma list,
        ]
        \addplot table [x = {l}, y={time}]{data/es/l/jazz_l_0.05.csv};
		\addplot table [x = {l}, y={time}]{data/es/l/jazz_l_1.csv};
		\legend{{$r = 0.05$}, {$r = 1$}};
		\end{axis}
	\end{tikzpicture}
	\caption{Время работы от количества шагов $l$ в работе \emph{ACGGC} на четырёх графах при $r = 0.05$ и $r = 1$}
	\label{fig:es-l-t}
\end{figure}

Стоит отметить, что на графах \emph{jazz}, \emph{as-22july06} модулярности при разных $r$ практически не отличались, а также начиная с некоторого $l$ на этих двух графах и на графе \emph{netscience} модулярность не росла. На графе \emph{\celegans} модулярность достигала некоторого оптимального $l$ и затем падала. При этом время работы растёт практически линейно от $l$, как можно увидеть на рисунке \ref{fig:es-l-t}.

\begin{figure}[H]
	\begin{tikzpicture}
		\begin{axis}[
		    table/col sep = semicolon,
		    height = 0.16\paperheight,
		    width = 0.4\columnwidth,
		    xlabel = {$l$},
		    ylabel = {$Q$},
		    legend pos = north east,
		    no marks,
		    title = {\celegans},
		    y label style={at={(axis description cs:-0.23,.5)}, anchor=south},
		    yticklabel style={/pgf/number format/fixed,
                  /pgf/number format/precision=4},
            cycle list name = diploma list,
		]
		\addplot table [x = {l}, y={q}]{data/es/l/celegans_l_0.05.csv};
		\addplot table [x = {l}, y={q}]{data/es/l/celegans_l_1.csv};
		\end{axis}
	\end{tikzpicture}
	\begin{tikzpicture}
		\begin{axis}[
		    table/col sep = semicolon,
		    height = 0.16\paperheight,
		    width = 0.4\columnwidth,
		    xlabel = {$l$},
		    ylabel = {$Q$},
		    legend style = {
		    	cells = {anchor = west},
		    	legend pos = outer north east
		    },
		    no marks,
		    title = {jazz},
		    y label style={at={(axis description cs:-0.2,.5)}, anchor=south},
		    yticklabel style={/pgf/number format/fixed,
                  /pgf/number format/precision=4},
            cycle list name = diploma list,
		]	
		\addplot table [x = {l}, y={q}]{data/es/l/jazz_l_0.05.csv};
		\addplot table [x = {l}, y={q}]{data/es/l/jazz_l_1.csv};
		\legend{{$r = 0.05$}, {$r = 1$}};
		\end{axis}
	\end{tikzpicture}
	\begin{tikzpicture}
		\begin{axis}[
		    table/col sep = semicolon,
		    height = 0.16\paperheight,
		    width = 0.4\columnwidth,
		    xlabel = {$l$},
		    ylabel = {$Q$},
		    legend pos = north east,
		    no marks,
		    title = {as-22july06},
		    y label style={at={(axis description cs:-0.23,.5)}, anchor=south},
		    yticklabel style={/pgf/number format/fixed,
                  /pgf/number format/precision=3},
            cycle list name = diploma list,
		]
		\addplot table [x = {l}, y={q}]{data/es/l/july_l_0.05.csv};
		\addplot table [x = {l}, y={q}]{data/es/l/july_l_1.csv};
		\end{axis}
	\end{tikzpicture}
	\hskip 17pt
	\begin{tikzpicture}
		\begin{axis}[
		    table/col sep = semicolon,
		    height = 0.16\paperheight,
		    width = 0.4\columnwidth,
		    xlabel = {$l$},
		    ylabel = {$Q$},
		    legend style = {
		    	cells = {anchor = west},
		    	legend pos = outer north east
		    },
		    no marks,
		    title = {netscience},
		    y label style={at={(axis description cs:-0.15,.5)}, anchor=south},
		    cycle list name = diploma list,
		    yticklabel style={/pgf/number format/fixed,
                /pgf/number format/precision=4},
		]
		\addplot table [x = {l}, y={q}]{data/es/l/netscience_l_0.05.csv};
		\addplot table [x = {l}, y={q}]{data/es/l/netscience_l_1.csv};
		\legend{{$r = 0.05$}, {$r = 1$}};
		\end{axis}
	\end{tikzpicture}
	\caption{Зависимость модулярности от количества шагов $l$ в работе \emph{ACGGC} на четырёх графах при $r = 0.05$ и $r = 1$}
	\label{fig:es-l-q}
\end{figure}

Таким образом, нет смысла брать большое $l$, повышая время работы, но не гарантируя улучшения получившихся разбиений.

%%%%%%%%%%%%%%%%%%%%%%%%%%%%%%%%%%%%%%%%%%%%%%%%%%%%%%%%%%%%%%%%%%%%%%%%%%%%%%%%%%%%%%%%%%%%%%%%%%%%%%%%%%%%%%%

\subsection{Время работы}
\label{subsec:acggc_time}

Одним из способов влиять на время работы является увеличение коэффициента $\beta$ в функции качества $f(Q, k) = -\alpha(\ln Q - \beta \ln k)$. Осмысленно сокращать время работы, так как при наличии хорошего финального алгоритма, или при использовании итеративной схемы кластеризации основных групп графа, время работы начальных алгоритмом очень критично, в то время как увеличение модулярности начальных алгоритмов на небольшие значения может не принести большой выгоды глобально. Стоит отметить, что на графах, у которых существует ярко выраженная оптимальная точка $k$, увеличение $\beta$ не сильно повлияет на модулярность, но зато и не будет влиять на время работы. На графах, у которых с увеличением $k$ растёт модулярность, увеличение $\beta$ снижает модулярность и время работы.

Рисунок \ref{fig:es-beta1} был получен при применении \emph{ACGGC} с параметрами $d = 2,\ f(Q, k) = -1000(\ln Q - \beta \ln k),\ \hat{k}_0 = 5,\ l = 6,\ r = 0.05$.

\newpage~\vspace{-3cm}\\~
\begin{vwcol}[widths={0.99, 0.01}, rule=0pt]
\begin{figure}[H]
	\begin{tikzpicture}
		\begin{semilogxaxis}[
		    table/col sep = semicolon,
		    height = 0.16\paperheight,
		    width = 0.47\columnwidth,
		    xlabel = {$\beta$},
		    ylabel = {$Q$},
		    no marks,
		    title style = {xshift = 3.8cm, yshift = 0cm},
		    title = {\celegans},
		    y label style={at={(axis description cs:-.15,.5)}, anchor=south},
		    yticklabel style={/pgf/number format/fixed,
                  /pgf/number format/precision=3},
		    cycle list name = diploma list,
		]
		\addplot table [x = {beta}, y = {q}]{data/es/beta/celegans_beta.csv};
		\end{semilogxaxis}
	\end{tikzpicture}
	\hskip -2.3cm
	\begin{tikzpicture}
		\begin{semilogxaxis}[
		    table/col sep = semicolon,
		    height = 0.16\paperheight,
		    width = 0.47\columnwidth,
		    xlabel = {$\beta$},
		    ylabel = {$t$},
		    y label style={at={(axis description cs:-.1,.5)}, anchor=south},
		    no marks,
		    cycle list name = diploma list,
		]
		\addplot table [x = {beta}, y = {time}]{data/es/beta/celegans_beta.csv};
		\end{semilogxaxis}
	\end{tikzpicture}	%%%%%%%%%%%%%%%
	\begin{tikzpicture}
		\begin{semilogxaxis}[
		    table/col sep = semicolon,
		    height = 0.16\paperheight,
		    width = 0.47\columnwidth,
		    xlabel = {$\beta$},
		    ylabel = {$Q$},
		    no marks,
		    title style = {xshift = 3.8cm, yshift = 0cm},
		    title = {jazz},
		    y label style={at={(axis description cs:-0.15,.5)}, anchor=south},
		    yticklabel style={/pgf/number format/fixed,
                  /pgf/number format/precision=3},
		    cycle list name = diploma list,
		]
		\addplot table [x = {beta}, y = {q}]{data/es/beta/jazz_beta.csv};
		\end{semilogxaxis}
	\end{tikzpicture}
	\hskip -0.7cm
	\begin{tikzpicture}
		\begin{semilogxaxis}[
		    table/col sep = semicolon,
		    height = 0.16\paperheight,
		    width = 0.47\columnwidth,
		    xlabel = {$\beta$},
		    ylabel = {$t$},
		    y label style={at={(axis description cs:-0.1,.5)}, anchor=south},
		    no marks,
		    cycle list name = diploma list,
		]
		\addplot table [x = {beta}, y = {time}]{data/es/beta/jazz_beta.csv};
		\end{semilogxaxis}
	\end{tikzpicture}	%%%%%%%%%%%%%%%
	\begin{tikzpicture}
		\begin{semilogxaxis}[
		    table/col sep = semicolon,
		    height = 0.16\paperheight,
		    width = 0.47\columnwidth,
		    xlabel = {$\beta$},
		    ylabel = {$Q$},
		    no marks,
		    title style = {xshift = 3.8cm, yshift = -0.1cm},
		    title = {as-22july06},
		    y label style={at={(axis description cs:-.15,.5)}, anchor=south},
		    yticklabel style={/pgf/number format/fixed,
                  /pgf/number format/precision=3},
		    cycle list name = diploma list,
		]
		\addplot table [x = {beta}, y = {q}]{data/es/beta/july_beta.csv};
		\end{semilogxaxis}
	\end{tikzpicture}
	\hskip -1.8cm
	\begin{tikzpicture}
		\begin{semilogxaxis}[
		    table/col sep = semicolon,
		    height = 0.16\paperheight,
		    width = 0.47\columnwidth,
		    xlabel = {$\beta$},
		    ylabel = {$t$},
		    y label style={at={(axis description cs:-.18,.5)}, anchor=south},
		    no marks,
		    cycle list name = diploma list,
		]
		\addplot table [x = {beta}, y = {time}]{data/es/beta/july_beta.csv};
		\end{semilogxaxis}
	\end{tikzpicture}	%%%%%%%%%%%%%%%
	\begin{tikzpicture}
		\begin{semilogxaxis}[
		    table/col sep = semicolon,
		    height = 0.16\paperheight,
		    width = 0.47\columnwidth,
		    xlabel = {$\beta$},
		    ylabel = {$Q$},
		    no marks,
		    title style = {xshift = 3.8cm, yshift = 0cm},
		    title = {netscience},
		    y label style={at={(axis description cs:-.15,.5)}, anchor=south},
		    yticklabel style={/pgf/number format/fixed,
                  /pgf/number format/precision=3},
		    cycle list name = diploma list,
		]
		\addplot table [x = {beta}, y = {q}]{data/es/beta/netscience_beta.csv};
		\end{semilogxaxis}
	\end{tikzpicture}
	\hskip -1.3cm
	\begin{tikzpicture}
		\begin{semilogxaxis}[
		    table/col sep = semicolon,
		    height = 0.16\paperheight,
		    width = 0.47\columnwidth,
		    xlabel = {$\beta$},
		    ylabel = {$t$},
		    y label style={at={(axis description cs:-.1,.5)}, anchor=south},
		    no marks,
		    cycle list name = diploma list,
		]
		\addplot table [x = {beta}, y = {time}]{data/es/beta/netscience_beta.csv};
		\end{semilogxaxis}
	\end{tikzpicture}
	\caption{Зависимость модулярности и времени работы алгоритма от параметра $\beta$ в работе \emph{ACGGC} на четырёх графах}
	\label{fig:es-beta1}
\end{figure}
~\\~\\~\\~\\~\\~\\~\\~\\~\\~\\~\\~\\~\\~\\~\\~\\~\\~\\~\\~\\~\\~\\~\\~\\~\\~\\~\\~\\~\\~\\~\\~\\~\\~\\~\\~\\
\end{vwcol}

Видно, что на графах \emph{\celegans}, \emph{jazz} и \emph{as-22july06} увеличение $\beta$ довольно сильно влияло на время работы, при этом на графе \emph{\celegans} модулярность росла, на графе \emph{jazz} оставалась неизменной и лишь после некоторого переломного $\beta$ начинала падать в среднем, но сравнительно слабо. На графах \emph{as-22july06} и \emph{netscience} модулярность вела себя непредсказуемо, но принимала небольшой промежуток значений.

Другим подходом к ограничению времени является установка максимального возможного значения $k_n^{+}$ и $\hat{k}_n$ (как $k_n^{-}$ и $\hat{k}_n$ ограничены снизу единицей). Далее в работе ограничение сверху обозначается $k_{max} > 1 \in \mathbb{N}$.

Тогда шаг \ref{item:es-k+} в описании схемы \emph{ACGGC} в подразделе \ref{subsec:es-adaptive} преобразуется следующим образом: $k_n^{-} = \max\{1, \hat{k}_{n - 1} - d\},\ k_n^{+} = \min \{k_{max}, \hat{k}_{n + 1} + d\}$, а формула вычисления следующей центральной точки \eqref{eq:aes-centre} заменится следующей формулой:
\begin{equation}
	\hat{k}_n = \max \left\{1,\ \min \left\{k_{max},\ \left[\hat{k}_{n - 1} - \frac{y_n^{+} - y_n^{-}}{k_n^{+} - k_n^{-}}\right] \right\} \right\}
\end{equation}

Предполагается, что при таком подходе адаптивная стратегия будет давать несильно увеличивающиеся модулярности при уменьшении $k_{max}$ на графах с небольшой оптимальной точкой $k$, так как в таком случае будут отсекаться неправдоподобные большие $k$, при этом время будет так же несильно падать. На графах, у которых модулярность несильно растёт при увеличении $k$ время будет сильно снижаться, но при снижении модулярности.

В таблицах \ref{tab:es-kmax-q} и \ref{tab:es-kmax-t} показаны результаты применения адаптивной стратегии с параметрами $d = 2,\ f(Q, k) = -1000 \ln Q,\ \hat{k}_0 = 5,\ l = 6,\ r = 0.02$ и финальным алгоритмом $RG_{10}$ к четырём графам, поведение модулярности которых при разных $k$ можно увидеть на рисунке \ref{fig:es-kopt}.

\begin{figure}[H]
	\begin{tikzpicture}
		\begin{axis}[
			table/col sep = semicolon,
		    height = 0.16\paperheight,
		    width = 0.26\columnwidth,
		    xlabel = {$k$},
		    ylabel = {$Q$},
		    title = {jazz},
		    no marks,
		    cycle list name = diploma list,
		]
		\addplot table [x = {k}, y = {q}]{data/arg/validity/jazz.csv};
		\end{axis}
	\end{tikzpicture}
	\begin{tikzpicture}
		\begin{axis}[
			table/col sep = semicolon,
		    height = 0.16\paperheight,
		    width = 0.26\columnwidth,
		    xlabel = {$k$},
		    title = {\celegans},
		    no marks,
		    cycle list name = diploma list,
		]
		\addplot table [x = {k}, y = {q}]{data/arg/validity/celegans_metabolic.csv};
		\end{axis}
	\end{tikzpicture}
	\hskip -20pt
	\begin{tikzpicture}
		\begin{axis}[
			table/col sep = semicolon,
		    height = 0.16\paperheight,
		    width = 0.26\columnwidth,
		    xlabel = {$k$},
		    title = {netscience},
		    no marks,
		    cycle list name = diploma list,
		]
		\addplot table [x = {k}, y = {q}]{data/arg/validity/netscience.csv};
		\end{axis}
	\end{tikzpicture}
	\begin{tikzpicture}
		\begin{axis}[
			table/col sep = semicolon,
		    height = 0.16\paperheight,
		    width = 0.26\columnwidth,
		    xlabel = {$k$},
		    title = {cond-mat-2003},
		    no marks,
		    cycle list name = diploma list,
		]
		\addplot table [x = {k}, y = {q}]{data/arg/validity/cond-mat-2003.csv};
		\end{axis}
	\end{tikzpicture}
	\caption{Зависимость модулярности от параметра $k$ в работе $RG_k$ на четырёх графах}
	\label{fig:es-kopt}
\end{figure}

Из таблицы \ref{tab:es-kmax-q} можно увидеть, что время работы действительно заметно падает при уменьшении $k_{max}$, в то время как модулярность сильно падает только на графе \emph{netscience}. Более того, на некоторых графах (например на графах \emph{football}, \emph{cond-mat-2003}) модулярность при уменьшении $k_{max}$ даже растёт. Такое поведение может объясняться существованием на этих графах небольших параметров рандомизированного жадного алгоритма $k$, лучших или неотличимых по модулярности от больших $k$, но отличающихся по времени работы. И при установке на $k$ ограничения сверху такие небольшие оптимальные параметры $k$ будут чаще обнаруживаться.

\begin{table}[H]
 	\caption{Модулярность \emph{ACGGC} при разных $k_{max}$ на четырёх графах}
 	\label{tab:es-kmax-q}
 	\begin{tabularx}{\textwidth}{XR{1.5cm}R{1.5cm}R{1.5cm}R{1.5cm}R{1.5cm}}\hline
 		$k_{max}$		& $+\infty$	& 50		& 20		& 10		& 6 		\\\hline
 		polbooks		& 0.527237	& 0.527237	& 0.527237	& 0.527082	& 0.526985	\\
 		adjnoun			& 0.299720	& 0.299690	& 0.299859	& 0.300141	& 0.299676	\\
 		football		& 0.603324	& 0.603324	& 0.604184	& 0.604266	& 0.604266	\\
 		jazz			& 0.444739	& 0.444739	& 0.444739	& 0.444739	& 0.444739	\\
 		\celegans		& 0.439770	& 0.439368	& 0.439750	& 0.439460	& 0.439431	\\
 		email			& 0.573470	& 0.573416	& 0.573652	& 0.573756	& 0.573513	\\
 		netscience		& 0.953033	& 0.908130	& 0.842085	& 0.793289	& 0.768572	\\
 		cond-mat-2003	& 0.737611	& 0.743572	& 0.749595	& 0.749894	& 0.739200	\\\hline
 	\end{tabularx}
\end{table}

\begin{table}[H]
 	\caption{Время работы \emph{ACGGC} при разных $k_{max}$ на четырёх графах}
 	\label{tab:es-kmax-t}
 	\begin{tabularx}{\textwidth}{XR{1.5cm}R{1.5cm}R{1.5cm}R{1.5cm}R{1.5cm}}\hline
 		$k_{max}$		& $+\infty$	& 50		& 20		& 10		& 6 		\\\hline
 		polbooks		& 5.029		& 4.976		& 4.615		& 4.207		& 4.087		\\
 		adjnoun			& 6.115		& 6.099		& 5.481		& 4.952		& 4.744		\\
 		football		& 7.179		& 7.155		& 6.377		& 5.820		& 5.584		\\
 		jazz			& 23.66		& 23.25		& 20.92		& 19.12		& 18.59		\\
 		\celegans		& 23.85		& 23.49		& 22.48		& 20.92		& 20.01		\\
 		email			& 70.06		& 72.89		& 68.34		& 63.85		& 62.97		\\
 		netscience		& 477.97	& 85.40		& 46.08		& 38.26		& 30.89		\\
 		cond-mat-2003	& 41,950	& 9,596		& 6,075		& 5,092		& 4,166		\\\hline
 	\end{tabularx}
\end{table}


