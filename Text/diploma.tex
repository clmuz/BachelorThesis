\documentclass{matmex-diploma}

% \usepackage[no-math]{fontspec}		% XeLaTeX fonts
% \usepackage{polyglossia}        	% Babel for XeLaTeX
% \usepackage{hyperref}				% Ссылки внутри текста
% \usepackage{url}					% URL ссылки
% \usepackage{geometry}				% Margins
% \usepackage{mathspec}				% Math fonts
\usepackage{amssymb}				% For Real symbol
\usepackage{mathrsfs}				% Script Letters
\usepackage{pgfplots}       		% Графики
\usepackage{graphicx}       		% Работа с картинками
\usepackage{float}          		% Необходимо для указания места картинки на странице

\begin{document}
\filltitle{ru}{
    chair              = {Кафедра Информатики},
    title              = {Адаптивный рандомизированный алгоритм выделения сообществ в графах},
    % Здесь указывается тип работы. Возможные значения:
    %   coursework - Курсовая работа
    %   diploma - Диплом специалиста
    %   master - Диплом магистра
    %   bachelor - Диплом бакалавра
    type               = {bachelor},
    position           = {студента},
    group              = 442,
    author             = {Проданов Тимофей Петрович},
    supervisorPosition = {д. ф.-м. н., профессор},
    supervisor         = {О.Н. Граничин},
    reviewerPosition   = {},
    reviewer           = {В.А. Ерофеева},
    chairHeadPosition  = {},
    chairHead          = {},
%   university         = {Санкт-Петербургский Государственный Университет},
%   faculty            = {Математико-механический факультет},
%   city               = {Санкт-Петербург},
%   year               = {2013}
}
\filltitle{en}{
    chair              = {Department of Computer Science},
    title              = {Adaptive randomised algorithm for community detection in graphs},
    type 			   = {bachelor},
    author             = {Timofey Prodanov},
    supervisorPosition = {Professor},
    supervisor         = {Oleg Granichin},
    reviewerPosition   = {},
    reviewer           = {Victoria Erofeeva},
    chairHeadPosition  = {},
    chairHead          = {},
}

\maketitle
\tableofcontents

\section*{Введение}

\section{Предварительные сведения}

\subsection{Выделение сообществ в графах}
Исторически, изучение сетей происходило в рамках теории графов, которая начала своё существование с решения Леонардом Эйлером задачи о кёнигсбергских мостах. В 1920-х взял своё начало анализ социальных сетей и лишь последние двадцать лет развивается изучение \emph{сложных сетей}, то есть сетей с неправильной, сложной структурой, в некоторых случаях рассматривают динамически меняющейся во времени сложные сети. От изучения маленьких сетей внимание переходит к сетям из тысяч или миллионов узлов.

В процессе изучения сложных систем, построенных по реальным системам, оказалось, что распределение степеней $P(s)$, определённое как доля узлов со степенью $s$ среди всех узлов графа, сильно отличается от распределения Пуассона, которое ожидается для случайных графов. Также сети, построенные по реальным системам характеризуются короткими путями между любыми двумя узлами и большим количеством маленьких циклов\cite{Boccaletti&al:2006}. Это показывает, что модели, предложенные теорией графов, часто будут оказываться далеко от реальных потребностей.

Современное изучение сложных сетей привнесло значительный вклад в понимание реальных систем. Сложные сети с успехом были применены в таких разных областях, как изучение структуры и топологии интернета \cite{Faloutsos&al:1999, Broder&al:2000}, эпидемиологии \cite{Moore&Newman:2000}, биоинформатике \cite{Zhao&al:2006}, поиске преступников \cite{Hong&al:2009}, социологии \cite{Scott:2012} и многих других.

Свойством, присутствуещим почти у любой сети, является структура сообществ, разделение узлов сети на разные группы узлов так, чтобы внутри каждой группы соединений между узлами много, а соединений между узлами разных групп мало. Способность находить и анализировать подобные группы предоставляет большие возможности в изучении реальных систем, представленных с помощью сложных сетей. Плотно связанные группы узлов в социальных сетях представляют людей, принадлежащих социальным сообществам, плотно сплочённые группы узлов в интернете соответствуют страницам, посвящённым распространённым темам, а сообщества в генетических сетях связаны с функциональными модулями \cite{Boccaletti&al:2006}. Таким образом, выделение сообществ в сети является мощным инструментом для понимания функциональности сети.

\subsection{Определения и обозначения}

Формально, сложная система может быть представлена с помощью графа. В этой работе будут рассматриваться только невзвешенные неориентированные графы. Невзвешенный неориентированный граф $G = (\mathscr{N}, \mathscr{L})$ состоит из двух множеств --- множества $\mathscr{N} \ne \emptyset$, элементы которого называются \emph{узлами} или \emph{вершинами} графа, и множества $\mathscr{L}$ неупорядоченных пар из множества $\mathscr{N}$, элементы которого называются \emph{рёбрами} или \emph{связями}. Мощности множеств $\mathscr{N}$ и $\mathscr{L}$ равны $N$ и $K$ соответственно.

Подграфом называется граф $G' = (\mathscr{N}', \mathscr{L}')$, где $\mathscr{N}' \subset \mathscr{N}$ и $\mathscr{L}' \subset \mathscr{L}$.

Узел обычно обозначают по его порядковому месту $i$ в множестве $\mathscr{N}$, а ребро, соединяющее пару узлов $i$ и $j$ обозначается $l_{ij}$. Узлы, между которыми есть ребро называются \emph{смежными}. Граф часто представляют в матричном виде, задавая для него матрицу смежности $A$ размера $N \times N$, в которой элемент $a_{ij}$ равен единице, если ребро $l_{ij}$ существует, и 0, если не существует. В таком случае степенью узла называют величину $s_i = \sum_{j \in \mathscr{N}}a_{ij}$.

Прогулка из узла $i$ в узел $j$ --- это последовательность узлов, начинающаяся с узла $i$ и заканчивающаяся узлом $j$. Путь --- это прогулка, в которой каждый узел встречается единожды. Геодезический путь --- это кратчайший путь, а количество узлов в нём на один больше геодезического расстояния.

Сообщество --- это подграф, чьи узлы плотно связаны, однако структурная сплочённость узлов определялась по разному. Одно из определений вводит понятие \emph{клик}. Клик --- это максимальный такой подграф, состоящий из трёх и более вершин, каждая из которых связана с каждой другой вершиной из клика. $n$-клик --- это максимальный подграф, в котором самое большое геодезическое расстояние между любыми двумя вершинами не превосходит $n$. Другое определение гласит, что подграф $G'$ является сообществом, если сумма всех степеней внутри $G'$ больше суммы всех степеней, направленных в остальную часть графа \cite{Wasserman:1994}.

\subsection{Модулярность}
Однако подобными определениями пользоваться неудобно и их проверка достаточно долгая. В 2004 году была представлена \emph{модулярность} --- целевая функция, оценивающая неслучайность разбиения графа на сообщества \cite{Newman&Girvan:2004}. Допустим, у нас $\kappa$ сообществ, определим тогда симметричную матрицу $\mathbf{e}$ размером $\kappa \times \kappa$. Пусть $e_{ij}$ --- отношение количества рёбер, которые идут из сообщества $i$ в сообщество $j$, к полному количеству рёбер в графе (рёбра $l_{mn}$ и $l_{nm}$ считаются различными, $m$, $n$ --- узлы), $a_i = \sum_j{e_{ij}}$. След такой матрицы $\mathrm{Tr} \mathbf{e} = \sum_i{e_{ii}}$ показывает отношение рёбер в сети, которые соединяют узлы одного и того же сообщества, и хорошее разбиение на сообщества должно иметь высокое значение следа. Однако если поместить все вершины в одно сообщество --- след примет максимальное возможное значение, притом, что такое разбиение не будет сообщать ничего полезного о графе.

Поэтому далее определяется строка $a_i = \sum_j{e_{ij}}$, которая обозначает долю количества рёбер, идущих к узлам, принадлежащим сообществу $i$, к полному количеству рёбер в графе. Если в графе рёбра проходят между вершинами независимо от сообществ --- $e_{ij}$ будет в среднем равно $a_i a_j$, поэтому модулярность можно определить следующим образом:
$$Q = \sum_i{\left(e_{ii} - a_i^2\right)} = \mathrm{Tr} \mathbf{e} - \|\mathbf{e}^2\|,$$
где $\|\mathbf{x}\|$ является суммой элементов матрицы $\mathbf{x}$. Если количество рёбер внутри сообществ не будет отличаться от случайного взятого количества --- модулярность будет примерно равна 0. Максимальным возможным значением функции будет 1, но на практике модулярности графов лежат между 0.3 и 0.7.



\bibliographystyle{unsrt}
\bibliography{diploma}

\end{document}