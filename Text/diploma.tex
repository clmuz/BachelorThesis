\documentclass{matmex-diploma}

% \usepackage[no-math]{fontspec}		% XeLaTeX fonts
% \usepackage{polyglossia}        	% Babel for XeLaTeX
% \usepackage{hyperref}				% Ссылки внутри текста
% \usepackage{url}					% URL ссылки
% \usepackage{geometry}				% Margins
% \usepackage{mathspec}				% Math fonts
\usepackage{amssymb}				% For Real symbol
\usepackage{amsmath}				% Eqref
\usepackage{mathrsfs}				% Script Letters
\usepackage{pgfplots}       		% Графики
\usepackage{graphicx}       		% Работа с картинками
\usepackage{float}          		% Необходимо для указания места картинки на странице
\usepackage{algorithm2e}			% Algorithm
\usepackage{array}					% Tabular box size
\usepackage[font=small,skip=0pt]{caption} % Distance to caption
\usepackage{tabularx}


% ------------------- Русский для algorithm2e
\SetKwInput{KwData}{Исходные параметры}
\SetKwInput{KwResult}{Результат}
\SetKwInput{KwIn}{Входные данные}
\SetKwInput{KwOut}{Выходные данные}
\SetAlgorithmName{Алгоритм}{алгоритм}{Список алгоритмов}
\SetArgSty{textnormal}
\SetKwFor{ForAll}{forall}{do}{end}
% ------------------- Русский для algorithm2e

%\captionsetup[figure]{skip=2cm}

\begin{document}
\filltitle{ru}{
    chair              = {Кафедра Информатики},
    title              = {Адаптивный рандомизированный алгоритм выделения сообществ в графах},
    % Здесь указывается тип работы. Возможные значения:
    %   coursework - Курсовая работа
    %   diploma - Диплом специалиста
    %   master - Диплом магистра
    %   bachelor - Диплом бакалавра
    type               = {bachelor},
    position           = {студента},
    group              = 442,
    author             = {Проданов Тимофей Петрович},
    supervisorPosition = {д. ф.-м. н., профессор},
    supervisor         = {О.Н. Граничин},
    reviewerPosition   = {},
    reviewer           = {В.А. Ерофеева},
    chairHeadPosition  = {},
    chairHead          = {},
%   university         = {Санкт-Петербургский Государственный Университет},
%   faculty            = {Математико-механический факультет},
%   city               = {Санкт-Петербург},
%   year               = {2013}
}
\filltitle{en}{
    chair              = {Department of Computer Science},
    title              = {Adaptive randomised algorithm for community detection in graphs},
    type 			   = {bachelor},
    author             = {Timofey Prodanov},
    supervisorPosition = {Professor},
    supervisor         = {Oleg Granichin},
    reviewerPosition   = {},
    reviewer           = {Victoria Erofeeva},
    chairHeadPosition  = {},
    chairHead          = {},
}

\maketitle
\tableofcontents

\section*{Введение}

\section{Предварительные сведения}


%%%%%%%%%%%%%%%%%%%%%%%%%%%%%%%%%%%%%%%%%%%%%%%%%%%%%%%%%%%%%%%%%%%%%%%%%%%%%%%%%%%%%%%%%%%%%%%%%%%%%%%%%%%%%%%

\subsection{Выделение сообществ в графах}

Исторически, изучение сетей происходило в рамках теории графов, которая начала своё существование с решения Леонардом Эйлером задачи о кёнигсбергских мостах. В 1920-х взял своё начало анализ социальных сетей и лишь последние двадцать лет развивается изучение \emph{сложных сетей}, то есть сетей с неправильной, сложной структурой, в некоторых случаях рассматривают динамически меняющейся во времени сложные сети. От изучения маленьких сетей внимание переходит к сетям из тысяч или миллионов узлов.

В процессе изучения сложных систем, построенных по реальным системам, оказалось, что распределение степеней $P(s)$, определённое как доля узлов со степенью $s$ среди всех узлов графа, сильно отличается от распределения Пуассона, которое ожидается для случайных графов. Также сети, построенные по реальным системам характеризуются короткими путями между любыми двумя узлами и большим количеством маленьких циклов \cite{Boccaletti&al:2006}. Это показывает, что модели, предложенные теорией графов, часто будут оказываться далеко от реальных потребностей.

Современное изучение сложных сетей привнесло значительный вклад в понимание реальных систем. Сложные сети с успехом были применены в таких разных областях, как изучение структуры и топологии интернета \cite{Faloutsos&al:1999, Broder&al:2000}, эпидемиологии \cite{Moore&Newman:2000}, биоинформатике \cite{Zhao&al:2006}, поиске преступников \cite{Hong&al:2009}, социологии \cite{Scott:2012} и многих других.

Свойством, присутствуещим почти у любой сети, является структура сообществ, разделение узлов сети на разные группы узлов так, чтобы внутри каждой группы соединений между узлами много, а соединений между узлами разных групп мало. Способность находить и анализировать подобные группы предоставляет большие возможности в изучении реальных систем, представленных с помощью сложных сетей. Плотно связанные группы узлов в социальных сетях представляют людей, принадлежащих социальным сообществам, плотно сплочённые группы узлов в интернете соответствуют страницам, посвящённым распространённым темам, а сообщества в генетических сетях связаны с функциональными модулями \cite{Boccaletti&al:2006}. Таким образом, выделение сообществ в сети является мощным инструментом для понимания функциональности сети.


%%%%%%%%%%%%%%%%%%%%%%%%%%%%%%%%%%%%%%%%%%%%%%%%%%%%%%%%%%%%%%%%%%%%%%%%%%%%%%%%%%%%%%%%%%%%%%%%%%%%%%%%%%%%%%%

\subsection{Определения и обозначения}

Формально, сложная система может быть представлена с помощью графа. В этой работе будут рассматриваться только невзвешенные неориентированные графы. Неориентированный невзвешенный граф $G = (\mathscr{N}, \mathscr{L})$ состоит из двух множеств --- множества $\mathscr{N} \ne \emptyset$, элементы которого называются \emph{узлами} или \emph{вершинами} графа, и множества $\mathscr{L}$ неупорядоченных пар из множества $\mathscr{N}$, элементы которого называются \emph{рёбрами} или \emph{связями}. Мощности множеств $\mathscr{N}$ и $\mathscr{L}$ равны $N$ и $L$ соответственно.

Подграфом называется граф $G' = (\mathscr{N}', \mathscr{L}')$, где $\mathscr{N}' \subset \mathscr{N}$ и $\mathscr{L}' \subset \mathscr{L}$.

Узел обычно обозначают по его порядковому месту $i$ в множестве $\mathscr{N}$, а ребро, соединяющее пару узлов $i$ и $j$ обозначается $l_{ij}$. Узлы, между которыми есть ребро называются \emph{смежными}. Степенью узла назовём величину $s_i$, равную количеству рёбер, выходящих узла $i$.

Прогулка из узла $i$ в узел $j$ --- это последовательность узлов, начинающаяся с узла $i$ и заканчивающаяся узлом $j$. Путь --- это прогулка, в которой каждый узел встречается единожды. Геодезический путь --- это кратчайший путь, а количество узлов в нём на один больше геодезического расстояния.

До того, как мы определили понятие \emph{сообщество}, определим \emph{разбиение} на сообщества. Пусть $G = (\mathscr{N}, \mathscr{L})$ --- граф, тогда разбиением на сообщества будет называться разбиение множества его вершин $P = \{C_1, \dots, C_K\}$, то есть $\bigcup_{i = 1}^K C_i = \mathscr{N}$ и $C_i \cap C_j = \emptyset \ \forall i \neq j \in 1..K$.

Сообщество --- это такой подграф, чьи узлы плотно связаны, однако структурная сплочённость узлов можно определить по разному. Одно из определений вводит понятие \emph{клик}. Клик --- это максимальный такой подграф, состоящий из трёх и более вершин, каждая из которых связана с каждой другой вершиной из клика. $n$-клик --- это максимальный подграф, в котором самое большое геодезическое расстояние между любыми двумя вершинами не превосходит $n$. Другое определение гласит, что подграф $G'$ является сообществом, если сумма всех степеней внутри $G'$ больше суммы всех степеней, направленных в остальную часть графа \cite{Wasserman:1994}. Сообщества называются смежными, если существует ребро, направленное из вершины первого сообщества в вершину второго.


%%%%%%%%%%%%%%%%%%%%%%%%%%%%%%%%%%%%%%%%%%%%%%%%%%%%%%%%%%%%%%%%%%%%%%%%%%%%%%%%%%%%%%%%%%%%%%%%%%%%%%%%%%%%%%%

\subsection{Модулярность}

Однако подобными определениями сообществ пользоваться неудобно и их проверка достаточно долгая. В 2004 году была представлена \emph{модулярность} --- целевая функция, оценивающая неслучайность разбиения графа на сообщества \cite{Newman&Girvan:2004}. Допустим, у нас $K$ сообществ, определим тогда симметричную матрицу $\mathbf{e}$ размером $K \times K$. Пусть $e_{ij}$ --- отношение количества рёбер, которые идут из сообщества $i$ в сообщество $j$, к полному количеству рёбер в графе (рёбра $l_{mn}$ и $l_{nm}$ считаются различными, $m$, $n$ --- узлы). След такой матрицы $\mathrm{Tr} \mathbf{e} = \sum_{i \in 1..K}{e_{ii}}$ показывает отношение рёбер в сети, которые соединяют узлы одного и того же сообщества, и хорошее разбиение на сообщества должно иметь высокое значение следа. Однако если поместить все вершины в одно сообщество --- след примет максимальное возможное значение, притом, что такое разбиение не будет сообщать ничего полезного о графе.

Поэтому далее определяется вектор $\mathbf{a}$ длины K, элементы которой $a_i = \sum_{j \in 1..K}{e_{ij}}$, которая обозначает долю количества рёбер, идущих к узлам, принадлежащим сообществу $i$, к полному количеству рёбер в графе. Если в графе рёбра проходят между вершинами независимо от сообществ --- $e_{ij}$ будет в среднем равно $a_i a_j$, поэтому модулярность можно определить следующим образом:
\begin{equation} \label{eq:q1}
Q(G, P) = \sum_{i \in 1..K}{\left(e_{ii} - a_i^2\right)} = \mathrm{Tr} \mathbf{e} - \|\mathbf{e}^2\|,
\end{equation}
где $\|\mathbf{x}\|$ является суммой элементов матрицы $\mathbf{x}$. Если количество рёбер внутри сообществ не будет отличаться от случайного взятого количества --- модулярность будет примерно равна 0. Максимальным возможным значением функции будет 1, но на практике модулярности графов лежат между 0.3 и 0.7.

Было предложено несколько вариаций модулярности \cite{Muff&Rao&Caflisch:2005, Fortunato&Barthelemy:2007}. Так, эквивалентным приведённому выше определению будет
\begin{equation}
Q(G, P) = \frac{1}{2L} \sum_{x, y \in 1..N} \left(w_{xy} - \frac{s_x s_y}{2L}\right)\ \delta(c_P(x), c_P(y)),
\end{equation}
где $L$ --- мощность $\mathscr{L}$, $w_{xy}$ --- вес ребра между вершинами $x$ и $y$, $s_x$ и $s_y$ --- степени вершин $x$ и $y$ соответственно, $\delta$ --- символ Кронекера, а отображение $c_P(\cdot)$ указывает, в каком сообществе разбиения лежит узел графа.

Теперь можно поставить задачу выделения сообществ следующим образом: требуется найти такое разбиение графа, что модулярность примет максимальное значение. Можно заметить, что такая постановка не использует какого-либо определения сообществ, и получившиеся разбиение не проверяется на дополнительные свойства, кроме подсчёта модулярности. Однако такая задача всё ещё будет NP-сложной \cite{Brandes&al:2008}.

Преимущество модулярности состоит в том, что для того, чтобы посчитать, какой выигрыш мы извлечем из объединения двух сообществ, необходимо произвести только одну операцию. В рамках определения \eqref{eq:q1} такой выигрыш будет равен $\Delta Q = 2(e_{ij} - a_i a_j),$ где $i$ и $j$ --- потенциально объединяемые сообщества.

Для того, чтобы объединить два сообщества необходимо сделать $O(\min\{n_i, n_j\})$ операций, где $n_i$ и $n_j$ обозначают количество смежных к $i$ и $j$ сообществ. Не умоляя общности, $n_j \leq n_i$, тогда необходимо обновить столбец $i$-ый столбец и $i$-ую строку матрицы $\mathbf{e}$, а так же $i$-ый элемент вектора $\mathbf{a}$: $e_{ik} = e_{ki} = e_{ki} + e_{kj},$ где $k$ --- смежное к $j$ сообщество, и $a_{i} = a_{i} + a_{j}$. При этом сообщество $j$ следует удалить из дальнейшего рассмотрения.

Имея матрицу $\mathbf{e}$ и вектор $\mathbf{a}$ не очень важно, как устроен граф и сообщества, что позволяет искать сообщества, основываясь на некотором начальном разбиении, для которого построены $\mathbf{e}$ и $\mathbf{a}$.


%%%%%%%%%%%%%%%%%%%%%%%%%%%%%%%%%%%%%%%%%%%%%%%%%%%%%%%%%%%%%%%%%%%%%%%%%%%%%%%%%%%%%%%%%%%%%%%%%%%%%%%%%%%%%%%

\subsection{Рандомизированный жадный алгоритм}

Ньюман в 2004 году предложил алгоритм, максимизирующий модулярность \cite{Newman:2004}. Алгоритм начинается с разбиения графа на $N$ сообществ из одной вершины, а затем на каждой итерации просматривает все пары сообществ и соединяет ту пару, которая даст наибольший выигрыш модулярности. Такой алгоритм достаточно долго работает и страдает от несбалансированного объединения сообществ --- сообщества растут с разной скоростью, большие кластеры соединяются со своими небольшими соседями независимо от того, выгодно это глобально или нет \cite{Ovelgoenne&Geyer-Schulz:2012a}.

Поэтому был предложен рандомизированный жадный алгоритм (RG) \cite{Ovelgoenne&Geyer-Schulz:2010}, который на каждой итерации рассматривал $k$ случайных сообществ и смежных к ним сообществ, а затем так же соединял пару, дающую наибольший выигрыш. Трудоёмкость такого алгоритма примерно равна $O(kL \ln N)$. И первый алгоритм, и его рандомизированная вариация соединяют сообщества, записывая только номера соединений, до тех пор, пока не останется только одно сообщество, а затем создают разбиение из списка соединений до того момента, когда достигалась максимальная модулярность (так как в результате лучшего соединения модулярность может уменьшиться).

Можно отметить, что таким алгоритмом можно кластеризовать не только граф, но и граф с некоторым начальным разбиением, в котором можно сообщества разбивать дальше, но нельзя их соединять. При этом только немного поменяется начальный этап инициализации матрицы $\mathbf{e}$ и вектора $\mathbf{a}$ (смотри Алгоритм \ref{alg:RG}).

\begin{algorithm}[p]
\SetAlgoLined
\KwIn{Невзвешенный неориентированный граф $G = (\mathscr{N}, \mathscr{L})$, параметр $k$}
\KwOut{Разбиение на сообщества $P$}
\BlankLine
\For{$i \in 1..N$}{
	\For{$j \in 1..N$}{
		\eIf{$i$ и $j$ смежные}{
			$e[i, j] = 1 / (2 * L)$\;
		}{
			$e[i, j] = 0$\;
		}
	}
	$a[i] = \sum_j e[i, j]$\;
}
$global\Delta Q \leftarrow 0$\;
$max\_global\Delta Q \leftarrow -\infty$\;
\BlankLine
\For{$i \in 1..N$}{
	$max\Delta Q \leftarrow -\infty$\;
	\For{$j \in 1..k$}{
		$c1 \leftarrow$ случайное сообщество\;
		\ForAll{сообщества $c2$, смежные с $c1$}{
			$\Delta Q \leftarrow 2 * (e[i, j] - a[i] * a[j])$\;
			\If{$\Delta Q > max\Delta Q$}{
				$max\Delta Q \leftarrow \Delta Q$\;
				$next\_join \leftarrow (c1, c2)$\;
			}
		}
	}
	$joins\_list.push(next\_join)$\;
	$global\Delta Q \leftarrow global\Delta Q + max\Delta Q$\;
	\If{$global\Delta Q > max\_global\Delta Q$}{
		$max\_global\Delta Q \leftarrow global\Delta Q$\;
		$best\_step \leftarrow i$\;
	}
	\BlankLine
	$(c1, c2) \leftarrow next\_join$\;
	\If{количество соседей($c2$) > количество соседей($c1$)}{
		поменять местами $c1$ и $c2$\;
	}
	\ForAll{соседи $c3$ сообщества $c2$, где $c3 \neq c1, c2$}{
		$e[c3, c1] \leftarrow e[c3, c1] + e[c3, c2]$\;
		$e[c1, c3] \leftarrow e[c3, c1]$\;
	}
	$e[c1, c1] \leftarrow e[c1, c1] + e[c2, c2] + e[c1, c2] + e[c2, c1]$\;
	$a[c1] \leftarrow a[c1] + a[c2]$\;
}
\BlankLine
$P \leftarrow $ создать разбиение из $joins\_list[1..best\_step]$\;
\BlankLine
\caption{Рандомизированный жадный алгоритм}
\label{alg:RG}
\end{algorithm}


%%%%%%%%%%%%%%%%%%%%%%%%%%%%%%%%%%%%%%%%%%%%%%%%%%%%%%%%%%%%%%%%%%%%%%%%%%%%%%%%%%%%%%%%%%%%%%%%%%%%%%%%%%%%%%%

\subsection{Ансамблевая стратегия}
\label{subsec:ens}

Овельгённе и Гейер-Шульц в 2012 году выиграли 10th DIMACS Implementation Challenge с ансамблевой стратегией выделения сообществ (ES). Ансамблевая стратегия заключается в том, что сначала $s$ начальных алгоритмов разбивают граф на сообщества, и считается, что те вершины, в которых начальные алгоритмы сошлись во мнении определены по сообществам правильно, а те, которые остались, распределяет по сообществам финальный алгоритм \cite{Ovelgoenne&Geyer-Schulz:2012b}.

Формализовать это можно следующим образом:
\begin{enumerate}
	\item Создать множество $S$ из $s$ разбиений $G$ с помощью начальных алгоритмов
	\item Создать разбиение $\hat{P}$, равное максимальному перекрытию разбиений из множества $S$
	\item Финальным алгоритмом создать разбиение $\widetilde{P}$ графа $G$ на основе разбиения $\hat{P}$
\end{enumerate}

Необходимо определить понятие \emph{максимальное перекрытие}. Пусть у нас есть множество $S = \{P_1, \dots, P_s\}$, $c_P(v)$ указывает, в каком сообществе находится узел $v$ с разбиении $P$.
Тогда у максимального перекрытия $\hat{P}$ множества $S$ будут следующие свойства:
$$v, w \in \mathscr{N}, \forall i \in 1..s\ :\ c_{P_i}(v) = c_{P_i}(w) \Rightarrow c_{\hat{P}}(v) = c_{\hat{P}}(w)$$
$$v, w \in \mathscr{N}, \exists i \in 1..s\ :\ c_{P_i}(v) \ne c_{P_i}(w) \Rightarrow c_{\hat{P}}(v) \ne c_{\hat{P}}(w)$$

Ансамблевую стратегию можно итерировать, заставляя начальные алгоритмы разбивать максимальное перекрытие и получившееся максимальное перекрытие до тех пор, пока это будет увеличивать модулярность. В таком случае схема будет выглядеть следующим образом:

\begin{enumerate}
	\item Инициализировать $\hat{P}$ разбиением из сообществ из одного узла
	\item Создать множество $S$ из $s$ разбиений графа $G$ на основе разбиения $\hat{P}$ с помощью начальных алгоритмов
	\item Записать в $\hat{P}$ максимальное перекрытие множества $S$
	\item Если $P_{best}$ не существует или оно хуже, чем $\hat{P}$, то присвоить $P_{best} \leftarrow \hat{P}$ и вернуться на второй шаг
	\item Финальным алгоритмом создать разбиение $\widetilde{P}$ графа $G$ на основе разбиения $P_{best}$ 
\end{enumerate}

В качестве начальных и финального алгоритма можно брать рандомизированный жадный алгоритм.


%%%%%%%%%%%%%%%%%%%%%%%%%%%%%%%%%%%%%%%%%%%%%%%%%%%%%%%%%%%%%%%%%%%%%%%%%%%%%%%%%%%%%%%%%%%%%%%%%%%%%%%%%%%%%%%

\subsection{Одновременно возмущаемая стохастическая аппроксимация}
Стохастические аппроксимация была введена Роббинсом и Монро в 1951 году \cite{Robbins&Monro:1951} и затем была использована для решения оптимизационный задач Кифером и Вольфовицем (KW) \cite{Kiefer&Wolfowitz:1952}. В \cite{Blum:1954} алгоритм стохастической аппроксимации был расширен до многомерного случая. В $m$-мерном пространстве обычная KW-процедура, основанная на конечно-разностной аппроксимации градиента, использовала $2m$ измерений на каждой итерации (по два измерения на каждую координату градиента). Спалл предложил алгоритм \emph{одновременно возмущаемой стохастической аппроксимации} (SPSA) \cite{Spall:1992}, который на каждой итерации использует всего два измерения. Он показал, что SPSA алгоритм имеет такую же скорость сходимости, несмотря на то, что в многомерном случае (даже при $m \to \infty$), несмотря на то, что в нём используется заметно меньше измерений \cite{Spall:2005}.

Стохастическая аппроксимация первоначально использовалась как инструмент для статистических вычислений и в дальнейшем разрбатывалась в рамках отдельной ветки теории управления. На сегодняшний день стохастическая аппроксимация имеет большое разнообразие применений в таких областях, как адаптивная обработка сигналов, адаптивное выделение ресурсов, адаптивное управление.

Алгоритмы стохастической аппроксимации показали свою эффективность в решении задач минимизации стационарных функционалов. В \cite{Polyak:1987} для функционалов, меняющихся со временем были применены метод Ньютона и градиентный метод, но они применимы только в случае дважды дифференцируемых функционалов и в случае известных ограничений на Гессиан функционала. Так же оба метода требуют возможности вычисления градиента в произвольной точке.

Общую схему одновременно возмущаемой стохастической аппроксимации можно представить следующим образом:

\begin{enumerate}
	\item Выбор начальной центральной точки $\theta_0 \in \mathbb{R}^m$, счётчик $n = 0$, выбор параметров алгоритма $d \in \mathbb{R} \setminus \{0\}$, $\{\alpha_n\} \subset \mathbb{R}^m$
	\item Увеличение счётчика $n \rightarrow n + 1$
	\item Выбор вектора возмущения $\Delta_n \in \mathbb{R}^m$, чьи координаты независимо генерируются и в среднем дают ноль. Часто для генерации компонент вектора используют распределение Бернулли, дающее $\pm1$ с вероятностью $\frac{1}{2}$ для каждого значения
	\item Определение новых аргументов функции $\theta_{n}^{-}=\hat{\theta}_{n - 1} - d\Delta_{n}$ и $\theta_{n}^{+}=\hat{\theta}_{n - 1} + d\Delta_{n}$
	\item Вычисление значений функционала $y_n^{-} = f(\theta_{n}^{-}), y_n^{+} = f(\theta_{n}^{+})$
	\item Вычисление следующей центральной точки
	\begin{equation} \label{eq:spsa-central}
		\hat{\theta}_n = \hat{\theta}_{n - 1} - \alpha_n \frac{y_n^{+} - y_n^{-}}{|\theta_{n}^{+} - \theta_{n}^{-}|}
	\end{equation}
	\item Далее происходит либо остановка алгоритма, либо переход на второй пункт
\end{enumerate}

В \cite{Granichin&Amelina:2015} был представлен метод стохастической аппроксимации с константным размером шага, в таком случае вместо последовательности $\{\alpha_n\}$ используется единственный параметр $\alpha \in \mathbb{R}^m$, и следующая центральная точка вычисляется по следующей формуле: $\hat{\theta}_n = \hat{\theta}_{n - 1} - \alpha \frac{y_n^{+} - y_n^{-}}{|\theta_{n}^{+} - \theta_{n}^{-}|}$


%%%%%%%%%%%%%%%%%%%%%%%%%%%%%%%%%%%%%%%%%%%%%%%%%%%%%%%%%%%%%%%%%%%%%%%%%%%%%%%%%%%%%%%%%%%%%%%%%%%%%%%%%%%%%%%

\subsection{Постановка задачи}

Рандомизированный жадный алгоритм имеет один параметр $k$, в то время как ансамблевая имеет стратегия один параметр $s$ и, в случае использования рандомизированного жадного алгоритма в качестве начального и финального алгоритма, будет иметь два дополнительных параметра $k_1$ и $k_2$, то есть в целом три параметра. Если на параметры рандомизированного жадного алгоритма поставить ограничения, к примеру, $k_1 = k_2$, то то можно считать, что ансамблевая стратегия имеет два параметра, $s$ и $k$.

Часто алгоритмы на каждых входных данных имеют разные оптимальные параметры, то есть параметры, решающие задачу наилучшим образом. И алгоритм SPSA показал хорошие результаты в адаптировании параметров подобных алгоритмов, в ходе работы рассматриваемого алгоритма его параметры варьируются, довольно быстро достигая оптимальной точки.

В данной работе будет рассматриваться применение алгоритма SPSA к двум алгоритмам выделения сообществ в графах, а так же сравнение модулярностей этих алгоритмов и их адаптивных модификаций.


%%%%%%%%%%%%%%%%%%%%%%%%%%%%%%%%%%%%%%%%%%%%%%%%%%%%%%%%%%%%%%%%%%%%%%%%%%%%%%%%%%%%%%%%%%%%%%%%%%%%%%%%%%%%%%%

\subsection{Тестовые графы}

В качестве тестовых графов были взяты графы, используемые для оценки алгоритмов на 10th DIMACS Implementation Challenge. Далее представлены используемые графы с кратким описанием, в порядке увеличения веса файла в формате METIS Graph.

\begin{itemize}
	
	\item \textbf{karate}, $N = 34,\;L = 78$: социальная сеть между 34 членами карате клуба с 1970 по 1972 год \cite{Zachary:1977}
	\item \textbf{dolphins}, $N = 62,\;L = 159$: социальная сеть частых общений между 62 дельфинами \cite{Lusseau&al:2003}
	\item \textbf{chesapeake}, $N = 39,\;L = 170$: сеть мезогалинных вод Чесапикского залива \cite{Baird&Ulanowicz:1989}
	\item \textbf{adjnoun}, $N = 112,\;L = 425$: сеть смежности частых прилагательных и существительных в романе <<Давид Копперфильд>> Чарльза Диккенса \cite{Newman:2006}
	\item \textbf{polbooks}, $N = 105,\;L = 441$: сеть книг о политике США, изданных во время президентских выборов 2004 года. Рёбра между книгами означают частые покупки одними и теми же покупателями. Сеть скомпилирована Валдисом Кребсом, однако не была опубликована
	\item \textbf{football}, $N = 115,\;L = 613$: сеть игр в американский футбол между колледжами из дивизиона IA во время регулярного сезона осенью 2000 года \cite{Girvan&Newman:2002}
	\item \textbf{celegans\_metabolic}, $N = 453,\;L = 2025$: метаболическая сеть Caenorhabditis elegans \cite{Duch&Arenas:2005}
	\item \textbf{jazz}, $N = 198,\;L = 2742$: сеть джазовых музыкантов \cite{Gleiser&Danon:2003}
	\item \textbf{netscience}, $N = 1589,\;L = 2742$: сеть соавторства среди учёных, работающих над сложными сетями \cite{Newman:2006}
	\item \textbf{email}, $N = 1133,\;L = 5451$: сеть связей по электронной почте между членами Университета Ровира и Вирхилий \cite{Guimera&al:2003}
	\item \textbf{polblogs}, $N = 1490,\;L = 16715$: сеть ссылок между веб блогами о политике США в 2005 году \cite{Adamic&Glance:2005}
	\item \textbf{PGPgiantcompo}, $N = 10680,\;L = 24316$: список узлов гигантской компоненты сети пользователей алгоритма Pretty-Good-Privacy для защищенного обмена информацией \cite{Boguna&al:2004}
	\item \textbf{as-22july06}, $N = 22963,\;L = 48436$: структура интернета на уровне автономных систем на 22 июля 2006 года. Сеть создана Марком Ньюманом и не опубликована
	\item \textbf{cond-mat-2003}, $N = 31163,\;L = 120029$: сеть соавторства среди учёных, публиковавших препринты в определённые архивы между 1995 и 2003 годами \cite{Newman:2001}
	\item \textbf{caidaRouterLevel}, $N = 192244,\;L =  609066$: граф структуры интернета на уровне роутера, собранный ассоциацией CAIDA в апреле и мае 2003 года
	\item \textbf{cnr-2000}, $N = 325557,\;L = 2738969$: небольшая часть обхода итальянского домена .cnr \cite{Boldi&Vigna:2004, Boldi&al:2011, Boldi&al:2004}
	\item \textbf{in-2004}, $N = 1382908,\;L = 13591473$: небольшая часть обхода индийского домена .in \cite{Boldi&Vigna:2004, Boldi&al:2011, Boldi&al:2004}
	\item \textbf{eu-2005}, $N = 862664,\;L = 16138468$: небольшая часть обхода домена Европейского союза .eu \cite{Boldi&Vigna:2004, Boldi&al:2011, Boldi&al:2004}

\end{itemize}

Кроме того, некоторые тесты используют автоматически сгенерированные графы с заранее известным количеством сообществ. Такой граф имеет четыре параметра --- количество узлов $N$, количество сообществ $K$ (все сообщества одинаковых размеров), вероятность появления ребра между узлами одного сообщества $p_1$ и вероятность появления ребра между вершинами разных сообществ $p_2$. Преимущество автоматически сгенерированных графов заключается в проверке работы алгоритма на разных по размеру графах с разными по плотности сообществами. Так же из построения известны реальные сообщества. Однако графы, построенные по реальным системам могут сильно отличаться от подобных графов.

Так, далее в работе используется автоматически сгенерированный граф под названием \emph{auto40}, со следующими параметрами: $N = 40,000,\ K = 40,\ p_1 = 0.1,\ p_2 = 5\cdot 10^{-4}$.
%%%%%%%%%%%%%%%%%%%%%%%%%%%%%%%%%%%%%%%%%%%%%%%%%%%%%%%%%%%%%%%%%%%%%%%%%%%%%%%%%%%%%%%%%%%%%%%%%%%%%%%%%%%%%%%

\section{Адаптивный рандомизированный жадный алгоритм}


%%%%%%%%%%%%%%%%%%%%%%%%%%%%%%%%%%%%%%%%%%%%%%%%%%%%%%%%%%%%%%%%%%%%%%%%%%%%%%%%%%%%%%%%%%%%%%%%%%%%%%%%%%%%%%%

\subsection{Применимость алгоритма SPSA}

Для того, чтобы алгоритм SPSA был применим --- необходимо иметь выпуклую функцию качества, которую необходимо минимизировать. В большинстве модулярность результатов работы \emph{RG} на разных графах с разными значениями параметра $k$ будет выглядеть следующим образом:

\begin{figure}[H]
	\begin{tikzpicture}
		\begin{axis}[
		    table/col sep = semicolon,
		    height = 0.16\paperheight, 
		    width = 0.495\columnwidth,
		    xlabel = {$k$},
		    ylabel = {$Q$},
		    legend pos = north east,
		    no marks
		]
		\addplot table [x={k}, y={q}]{data/arg/validity/karate.csv};
		\legend{karate};
		\end{axis}
	\end{tikzpicture}
	\hskip 0.01\columnwidth
	\begin{tikzpicture}
		\begin{axis}[
		    table/col sep = semicolon,
		    height = 0.16\paperheight, 
		    width = 0.495\columnwidth,
		    xlabel = {$k$},
		    ylabel = {$Q$},
		    legend pos = south east,
		    no marks
		]
		\addplot table [x={k}, y={q}]{data/arg/validity/chesapeake.csv};
		\legend{chesapeake};
		\end{axis}
	\end{tikzpicture} % 2nd row
	\begin{tikzpicture}
		\begin{axis}[
		    table/col sep = semicolon,
		    height = 0.16\paperheight, 
		    width = 0.495\columnwidth,
		    xlabel = {$k$},
		    ylabel = {$Q$},
		    legend pos = south east,
		    no marks
		]
		\addplot table [x={k}, y={q}]{data/arg/validity/polbooks.csv};
		\legend{polbooks};
		\end{axis}
	\end{tikzpicture}
	\hskip 0.01\columnwidth
	\begin{tikzpicture}
		\begin{axis}[
		    table/col sep = semicolon,
		    height = 0.16\paperheight, 
		    width = 0.495\columnwidth,
		    xlabel = {$k$},
		    ylabel = {$Q$},
		    legend pos = south east,
		    no marks
		]
		\addplot table [x={k}, y={q}]{data/arg/validity/netscience.csv};
		\legend{netscience};
		\end{axis}
	\end{tikzpicture} % 3rd row
	\begin{tikzpicture}
		\begin{axis}[
		    table/col sep = semicolon,
		    height = 0.16\paperheight, 
		    width = 0.495\columnwidth,
		    xlabel = {$k$},
		    ylabel = {$Q$},
		    legend pos = north east,
		    no marks
		]
		\addplot table [x={k}, y={q}]{data/arg/validity/email.csv};
		\legend{email};
		\end{axis}
	\end{tikzpicture}
	\hskip 0.01\columnwidth
	\begin{tikzpicture}
		\begin{axis}[
		    table/col sep = semicolon,
		    height = 0.16\paperheight, 
		    width = 0.495\columnwidth,
		    xlabel = {$k$},
		    ylabel = {$Q$},
		    legend pos = south east,
		    no marks
		]
		\addplot table [x={k}, y={q}]{data/arg/validity/cond-mat-2003.csv};
		\legend{cond-mat-2003};
		\end{axis}
	\end{tikzpicture} % 4rd row
	\begin{tikzpicture}
		\begin{axis}[
		    table/col sep = semicolon,
		    height = 0.16\paperheight, 
		    width = 0.495\columnwidth,
		    xlabel = {$k$},
		    ylabel = {$Q$},
		    legend pos = south east,
		    no marks
		]
		\addplot table [x={k}, y={q}]{data/arg/validity/in-2004.csv};
		\legend{in-2004};
		\end{axis}
	\end{tikzpicture}
	\begin{tikzpicture}
		\begin{axis}[
		    table/col sep = semicolon,
		    height = 0.16\paperheight, 
		    width = 0.495\columnwidth,
		    xlabel = {$k$},
		    ylabel = {$Q$},
		    legend pos = north east,
		    no marks
		]
		\addplot table [x={k}, y={q}]{data/arg/validity/eu-2005.csv};
		\legend{eu-2005};
		\end{axis}
	\end{tikzpicture}
	\caption{Зависимость модулярности от $k$ в результатах работы $RG_k$ на разных графах}
\end{figure}

Результаты показывают разделение поведения алгоритма на разных графах на два возможных случая: в первом алгоритм принимает наилучший результат при некотором небольшом, но разном $k_{best}$ (например работа алгоритма на графе \emph{karate}). Во втором результаты алгоритма постепенно возрастают, приближаясь к некоторой асимптоте (хорошим примером будет работа алгоритма на графе \emph{netscience}. К первому случаю так же относится такое поведение, в котором алгоритм быстро (с ростом $k$) достигает своего лучшего значения, и затем его результаты очень несильно падают, и в дальнейшем держатся того же значения (такое выполняется, например, на графе \emph{in-2004}).

Похожее поведение алгоритм показывает на автоматически сгенерированных графах, но в таких графах можно получить более отличающиеся значения $k_{best}$ и предположить, что будет происходить с модулярностью при дальнейшем росте $k$.

\begin{figure}[H]
	\begin{tikzpicture}
		\begin{axis}[
		    table/col sep = semicolon,
		    height = 0.16\paperheight, 
		    width = \columnwidth,
		    xlabel = {$k$},
		    ylabel = {$Q$},
		    yticklabel style={/pgf/number format/fixed,
                  /pgf/number format/precision=3},
		    no marks
		]
		\addplot table [x={k}, y={q}]{data/arg/validity/auto13.csv};
		\end{axis}
	\end{tikzpicture}
	\caption{Зависимость модулярности от $k$ для разбиения на сообщества автоматически сгенерированного графа \emph{auto40} с параметрами $N = 40.000,\;K = 40,\;p_1 = 0.1,\;p_2 = 5\cdot 10^{-4}$ алгоритмом $RG_k$}
	\label{fig:q-auto13}
\end{figure}


%%%%%%%%%%%%%%%%%%%%%%%%%%%%%%%%%%%%%%%%%%%%%%%%%%%%%%%%%%%%%%%%%%%%%%%%%%%%%%%%%%%%%%%%%%%%%%%%%%%%%%%%%%%%%%%

\subsection{Функция качества}
\label{subsec:arg-f}

Таким образом, имеет смысл использовать в функции качества не только модулярность, но и время. Подсчёт времени сам по себе занимает время и в разных случаях может давать сильно отличающиеся результаты. Теоретическая трудоёмкость алгоритма линейно зависит от параметра $k$, на реальных графах зависимость тоже близка к линейной:

\begin{figure}[H]
	\begin{tikzpicture}
		\begin{axis}[
		    table/col sep = semicolon,
		    height = 0.16\paperheight, 
		    width = 0.495\columnwidth,
		    xlabel = {$k$},
		    ylabel = {$t$},
		    legend pos = south east,
		    no marks
		]
		\addplot[red] table [x={k}, y={time}]{data/arg/validity/polbooks.csv};
		\legend{polbooks};
		\end{axis}
	\end{tikzpicture}
	\hskip 0.01\columnwidth
	\begin{tikzpicture}
		\begin{axis}[
		    table/col sep = semicolon,
		    height = 0.16\paperheight, 
		    width = 0.495\columnwidth,
		    xlabel = {$k$},
		    ylabel = {$t$},
		    legend pos = south east,
		    no marks
		]
		\addplot[red] table [x={k}, y={time}]{data/arg/validity/celegans_metabolic.csv};
		\legend{\celegans};
		\end{axis}
	\end{tikzpicture}
	\caption{Зависимость времени $t$ в миллисекундах от $k$ для результатов работы $RG_k$ на графах \emph{polbooks} и \emph{\celegans}}
\end{figure}

Таким образом, вместо времени можно использовать значение $k$. Максимальное значение модулярности, которое может быть достигнуто на графе очень сильно отличается, поэтому нет смысла использовать абсолютное значение модулярности, но имеет смысл использовать относительные значения. При вычислении центрального значения \eqref{eq:spsa-central} в алгоритме одновременно возмущаемой стохастической аппроксимации использутся только разность функций качества. Таким образом, если использовать в функции качества логарифмы от модулярности --- разность функций укажет, во сколько раз модулярность изменилась.

Так же функция качества должна принимать минимум, а не максимум, поэтому первой версией подобной функции может быть $f(Q) = -\alpha \ln Q,\;\alpha > 0$. Для того, чтобы принимать во внимание время работы, разумно добавить логарифм от $k$: 
\begin{equation} \label{eq:arg-f}
f(Q, k) = -\alpha (\ln Q - \beta \ln k),\;\alpha > 0, \beta \ge 0
\end{equation}
Коэффициент $\beta$ в таком случае можно рассматривать в следующем виде $\beta = \frac{\ln \gamma}{\ln 2}$, где коэффициент $\beta$ указывает, во сколько раз необходимо увеличиться модулярности для того, чтобы покрыть увеличение времени (то есть $k$) вдвое. Если время не имеет значение коэффициент $\gamma$ принимает значение 1, а следовательно коэффициент $\beta$ принимает значение 0. Коэффициент $\alpha$ же играет роль размера шага при выборе следующей центральной точки.

\begin{figure}[H]
	\begin{tikzpicture}
		\begin{axis}[
		    table/col sep = semicolon,
		    height = 0.182\paperheight, 
		    width = 0.73\columnwidth,
		    xlabel = {$k$},
		    ylabel = {$F(Q, k)$},
		    legend style = {
		    	cells = {anchor = west},
		    	legend pos = outer north east
		    },
		    title = {karate},
		    no marks
		]
		\addplot table [x={k}, y expr={-ln(\thisrowno{1})}]{data/arg/validity/karate.csv};
		\addplot table [x={k}, y expr={-ln(\thisrowno{1}) + 0.01 * ln(\thisrowno{0})}]{data/arg/validity/karate.csv};
		\addplot table [x={k}, y expr={-ln(\thisrowno{1}) + 0.02 * ln(\thisrowno{0})}]{data/arg/validity/karate.csv};
		\legend{{$-\ln Q$}, {$-(\ln Q - 0.01 \ln k)$}, {$-(\ln Q - 0.02 \ln k)$}}
		\end{axis}
	\end{tikzpicture}
	\begin{tikzpicture}
		\begin{axis}[
		    table/col sep = semicolon,
		    height = 0.182\paperheight, 
		    width = 0.73\columnwidth,
		    xlabel = {$k$},
		    ylabel = {$F(Q, k)$},
		    legend style = {
		    	cells = {anchor = west},
		    	legend pos = outer north east
		    },
		    title = {jazz},
		    no marks
		]
		\addplot table [x={k}, y expr={-ln(\thisrowno{1})}]{data/arg/validity/jazz.csv};
		\addplot table [x={k}, y expr={-ln(\thisrowno{1}) + 0.01 * ln(\thisrowno{0})}]{data/arg/validity/jazz.csv};
		\addplot table [x={k}, y expr={-ln(\thisrowno{1}) + 0.02 * ln(\thisrowno{0})}]{data/arg/validity/jazz.csv};
		\legend{{$-\ln Q$}, {$-(\ln Q - 0.01 \ln k)$}, {$-(\ln Q - 0.02 \ln k)$}}
		\end{axis}
	\end{tikzpicture}
	\caption{Функции качества с разными коэффициентами $\beta$ для графов \emph{karate} и \emph{jazz}}
	\label{fig:qua1}
\end{figure}
\begin{figure}[H]
	\begin{tikzpicture}
		\begin{axis}[
		    table/col sep = semicolon,
		    height = 0.2\paperheight, 
		    width = 0.73\columnwidth,
		    xlabel = {$k$},
		    ylabel = {$F(Q, k)$},
		    legend style = {
		    	cells = {anchor = west},
		    	legend pos = outer north east
		    },
		    title = {netscience},
		    no marks
		]
		\addplot table [x={k}, y expr={-ln(\thisrowno{1})}]{data/arg/validity/netscience.csv};
		\addplot[violet] table [x={k}, y expr={-ln(\thisrowno{1}) + 0.1 * ln(\thisrowno{0})}]{data/arg/validity/netscience.csv};
		\addplot[purple] table [x={k}, y expr={-ln(\thisrowno{1}) + 0.2 * ln(\thisrowno{0})}]{data/arg/validity/netscience.csv};
		\legend{{$-\ln Q$}, {$-(\ln Q - 0.1 \ln k)$}, {$-(\ln Q - 0.2 \ln k)$}}
		\end{axis}
	\end{tikzpicture}
	\caption{Продолжение Рис. \ref{fig:qua1}. Функции качества с разными коэффициентами $\beta$ для графа \emph{netscience}}
	\label{fig:qua2}
\end{figure}

Как видно из рисунка \ref{fig:qua2}, иногда для того, чтобы функция качества имела минимум на небольших $k$, надо задавать довольно большое значение коэффициента $\beta$. Однако это логично, на данном графе если нас устраивает время работы --- выгоднее всё время увеличивать значение $k$.


%%%%%%%%%%%%%%%%%%%%%%%%%%%%%%%%%%%%%%%%%%%%%%%%%%%%%%%%%%%%%%%%%%%%%%%%%%%%%%%%%%%%%%%%%%%%%%%%%%%%%%%%%%%%%%%

\subsection{Адаптивный алгоритм}

При использовании алгоритма SPSA в рандомизированном жадном алгоритме предлагается разбить действие алгоритма на шаги длиной в $\sigma$ итераций. В течении каждого шага используется одно значение $k$. После каждого шага можно считать прирост модулярности, но вместо этого имеет смысл считать медиану прироста модулярности за $\sigma$ итераций --- так как алгоритм рандомизированный, время от времени будут появляться очень хорошие соединения сообществ, которые будут портить функцию качества, такой большой прирост может появиться даже при очень плохом $k$. В таком случае схема алгоритма будет выглядеть следующим образом:

\begin{enumerate}
	\item Выбор начальной центральной точки $\hat{k}_0 \in \mathbb{N}$, счётчик $n = 0$, выбор размера возбуждения $d \in \mathbb{N}$, коэффициентов функции качества $\alpha, \beta \in \mathbb{R},\; \alpha > 0,\;\beta \ge 0$, и $\sigma \in \mathbb{N}$ --- количество итераций в одном шаге
	\item Увеличение счётчика $n \rightarrow n + 1$
	\item Определение новых аргументов функции $k_{n}^{-} = \max\{\hat{k}_{n - 1} - d, 1\}$ и $k_{n}^{+}=\hat{k}_{n - 1} + d$
	\item Выполнение $\sigma$ итераций с параметром $k_{n}^{-}$, вычисление медианы прироста модулярности $\mu_n^{-}$
	\item Выполнение $\sigma$ итераций с параметром $k_{n}^{+}$, вычисление медианы прироста модулярности $\mu_n^{+}$
	\item Вычисление функций качества $y_n^{-} = -\alpha (\ln \mu_n^{-} - \beta \ln k_n^{-})$, $y_n^{+} = -\alpha (\ln \mu_n^{+} - \beta \ln k_n^{+})$
	\item Вычисление следующей центральной точки
	\begin{equation} \label{eq:arg-centre}
		\hat{k}_n = \max\left\{1, \left[\hat{k}_{n - 1} - \frac{y_n^{+} - y_n^{-}}{k_n^{+} - k_n^{-}}\right]\right\}
	\end{equation}
	\item Далее происходит переход на второй пункт
\end{enumerate}

Алгоритм заканчивает работу в тот момент, когда для рассмотрения осталось ровно одно сообщество. Далее в работе этот алгоритм называется \emph{адаптивным рандомизированным жадным алгоритмом} или \emph{ARG} (Adaptive Randomized Greedy).


%%%%%%%%%%%%%%%%%%%%%%%%%%%%%%%%%%%%%%%%%%%%%%%%%%%%%%%%%%%%%%%%%%%%%%%%%%%%%%%%%%%%%%%%%%%%%%%%%%%%%%%%%%%%%%%

\subsection{Чувствительность к перепадам функции качества}

Коэффициент $\alpha$ отвечает за то, насколько чувствителен алгоритм будет к перепадам функций качества --- чем больше $\alpha$, тем сильнее будет отличаться новая центральная точка от предыдущей в одной и той же ситуации. На трёх графах были измерены зависимости модулярности от параметра $\alpha$ при $d = 2,\ \sigma = 500,\ \hat{k}_0 = 10,\ \beta = 0$:

\begin{figure}[H]
	\begin{tikzpicture}
		\begin{axis}[
		    table/col sep = semicolon,
		    height = 0.14\paperheight, 
		    width = \columnwidth,
		    xlabel = {$\alpha$},
		    ylabel = {$Q$},
		    legend style = {
		    	cells = {anchor = west},
		    	legend pos = outer north east
		    },
		    title = {cond-mat-2003},
		    no marks
		]
		\addplot table [x={alpha}, y={q}]{data/arg/alpha/cond/cond_alpha.csv};
		\end{axis}
	\end{tikzpicture}
	\begin{tikzpicture}
		\begin{axis}[
		    table/col sep = semicolon,
		    height = 0.14\paperheight, 
		    width = \columnwidth,
		    xlabel = {$\alpha$},
		    ylabel = {$Q$},
		    legend style = {
		    	cells = {anchor = west},
		    	legend pos = outer north east
		    },
		    title = {caidaRouterLevel},
		    no marks
		]
		\addplot table [x={alpha}, y={q}]{data/arg/alpha/caida/caida_alpha.csv};
		\end{axis}
	\end{tikzpicture}
	\begin{tikzpicture}
		\begin{axis}[
		    table/col sep = semicolon,
		    height = 0.14\paperheight, 
		    width = 0.98\columnwidth,
		    xlabel = {$\alpha$},
		    ylabel = {$Q$},
		    legend style = {
		    	cells = {anchor = west},
		    	legend pos = outer north east
		    },
		    title = {cnr-2000},
		    no marks,
		    yticklabel style={/pgf/number format/fixed,
                  /pgf/number format/precision=3},
            y label style={at={(axis description cs:0,.5)}, anchor=south},
		]
		\addplot table [x={alpha}, y={q}]{data/arg/alpha/cnr/cnr_alpha.csv};
		\end{axis}
	\end{tikzpicture}
	\caption{Зависимость модулярности от параметра $\alpha$ в работе \emph{ARG} на графах \emph{cond-mat-2003}, \emph{caidaRouterLevel}, \emph{cnr-2000}}
	\label{fig:alpha}
\end{figure}

Из рисунка \ref{fig:alpha} видно, что результат не очень стабильный, хотя значение модулярности и колеблется на небольшом промежутке. Так же на третьем графе заметно, что увеличение параметра $\alpha$ влечёт за собой уменьшение модулярности в среднем.

\begin{figure}[H]
	\begin{tikzpicture}
		\begin{axis}[
		    table/col sep = semicolon,
		    height = 0.18\paperheight, 
		    width = 0.44\columnwidth,
		    xlabel = {$n$},
		    ylabel = {$k$},
		    y label style={at={(axis description cs:0.15,.5)}, anchor=south},
		    title = {cond-mat-2003},
		    title style = {xshift = 3.5cm, yshift = 0.3cm},
		    no marks
		]
		\addplot table [x expr = {\thisrowno{0} / 1000}, y={k}]{data/arg/alpha/cond/cond10.csv};
		\addplot table [x expr = {\thisrowno{0} / 1000}, y={k}]{data/arg/alpha/cond/cond50.csv};
		\addplot table [x expr = {\thisrowno{0} / 1000}, y={k}]{data/arg/alpha/cond/cond100.csv};
		\end{axis}
	\end{tikzpicture}
	\hskip -3cm
	\begin{tikzpicture}
		\begin{axis}[
		    table/col sep = semicolon,
		    height = 0.18\paperheight, 
		    width = 0.44\columnwidth,
		    xlabel = {$n$},
		    ylabel = {$\mu$},
		    legend pos = outer north east,
		    y label style={at={(axis description cs:0.15,.5)}, anchor=south},
		    no marks
		]
		\addplot table [x expr = {\thisrowno{0} / 1000}, y={mu}]{data/arg/alpha/cond/cond10.csv};
		\addplot table [x expr = {\thisrowno{0} / 1000}, y={mu}]{data/arg/alpha/cond/cond50.csv};
		\addplot table [x expr = {\thisrowno{0} / 1000}, y={mu}]{data/arg/alpha/cond/cond100.csv};
		\legend{{$\alpha = 10$}, {$\alpha = 50$}, {$\alpha = 100$}};
		\end{axis}
	\end{tikzpicture}
	\begin{tikzpicture}
		\begin{axis}[
		    table/col sep = semicolon,
		    height = 0.18\paperheight, 
		    width = 0.44\columnwidth,
		    xlabel = {$n$},
		    ylabel = {$k$},
		    y label style={at={(axis description cs:0.15,.5)}, anchor=south},
		    title = {caidaRouterLevel},
		    title style = {xshift = 3.5cm, yshift = 0.3cm},
		    no marks
		]
		\addplot table [x expr = {\thisrowno{0} / 1000}, y={k}]{data/arg/alpha/caida/caida10s.csv};
		\addplot table [x expr = {\thisrowno{0} / 1000}, y={k}]{data/arg/alpha/caida/caida50s.csv};
		\addplot table [x expr = {\thisrowno{0} / 1000}, y={k}]{data/arg/alpha/caida/caida100s.csv};
		\end{axis}
	\end{tikzpicture}
	\hskip -3cm
	\begin{tikzpicture}
		\begin{axis}[
		    table/col sep = semicolon,
		    height = 0.18\paperheight, 
		    width = 0.44\columnwidth,
		    xlabel = {$n$},
		    ylabel = {$\mu$},
		    legend pos = outer north east,
		    y label style={at={(axis description cs:0.1,.5)}, anchor=south},
		    no marks
		]
		\addplot table [x expr = {\thisrowno{0} / 1000}, y={mu}]{data/arg/alpha/caida/caida10s.csv};
		\addplot table [x expr = {\thisrowno{0} / 1000}, y={mu}]{data/arg/alpha/caida/caida50s.csv};
		\addplot table [x expr = {\thisrowno{0} / 1000}, y={mu}]{data/arg/alpha/caida/caida100s.csv};
		\legend{{$\alpha = 10$}, {$\alpha = 50$}, {$\alpha = 100$}};
		\end{axis}
	\end{tikzpicture}
	\begin{tikzpicture}
		\begin{axis}[
		    table/col sep = semicolon,
		    height = 0.18\paperheight, 
		    width = 0.44\columnwidth,
		    xlabel = {$n$},
		    ylabel = {$k$},
		    y label style={at={(axis description cs:0.1,.5)}, anchor=south},
		    title = {cnr-2000},
		    title style = {xshift = 3.5cm, yshift = 0.3cm},
		    no marks
		]
		\addplot table [x expr = {\thisrowno{0} / 1000}, y={k}]{data/arg/alpha/cnr/cnr5s.csv};
		\addplot table [x expr = {\thisrowno{0} / 1000}, y={k}]{data/arg/alpha/cnr/cnr50s.csv};
		\addplot table [x expr = {\thisrowno{0} / 1000}, y={k}]{data/arg/alpha/cnr/cnr100s.csv};
		\end{axis}
	\end{tikzpicture}
	\hskip -1.5cm
	\begin{tikzpicture}
		\begin{axis}[
		    table/col sep = semicolon,
		    height = 0.18\paperheight, 
		    width = 0.44\columnwidth,
		    xlabel = {$n$},
		    ylabel = {$\mu$},
		    legend pos = outer north east,
		    y label style={at={(axis description cs:0.1,.5)}, anchor=south},
		    no marks
		]
		\addplot table [x expr = {\thisrowno{0} / 1000}, y={mu}]{data/arg/alpha/cnr/cnr5s.csv};
		\addplot table [x expr = {\thisrowno{0} / 1000}, y={mu}]{data/arg/alpha/cnr/cnr50s.csv};
		\addplot table [x expr = {\thisrowno{0} / 1000}, y={mu}]{data/arg/alpha/cnr/cnr100s.csv};
		\legend{{$\alpha = 5$}, {$\alpha = 50$}, {$\alpha = 100$}};
		\end{axis}
	\end{tikzpicture}
	\caption{Изменения параметра $k$ и медианы приросты модулярности со временем для разных параметров $\alpha$ на трёх графах. Графики для второго и третьего графа смазаны для повышения читаемости}
\end{figure}

Из рисунков видно, что в зависимости от $\alpha$ параметр $k$, с которым прогоняются шаги по $\sigma$ шагов, и который поочередно принимает значения $k_n^{-}$ и $k_n^{+}$, меняется с разной интенсивностью, но результаты при этом получаются приблизительно одинаковые.

%%%%%%%%%%%%%%%%%%%%%%%%%%%%%%%%%%%%%%%%%%%%%%%%%%%%%%%%%%%%%%%%%%%%%%%%%%%%%%%%%%%%%%%%%%%%%%%%%%%%%%%%%%%%%%%

\subsection{Размер возмущения}

Коэффициент $d$ отвечает за то, насколько сильно будет возмущаться текущая оценка для получения следующих аргументов функционала. То есть, насколько $k_n^{+}$ и $k_n^{-}$ будут отличаться от $\hat{k}_{n - 1}$. На рисунке \ref{fig:arg-d} можно увидеть зависимость модулярности от размера возмущения при $f(\mu, k) = -10 \ln \mu,\ \sigma = 500,\ \hat{k}_0 = 10$.

На графах \emph{cnr-2000} и \emph{eu-2005} значения модулярности не очень сильно менялись в зависимости от параметра $d$, хотя некоторые значения $d$ и давали более большие значения. Однако на графах \emph{cond-mat-2003} и \emph{caidaRouterLevel} после некоторого порогового значения возмущения модулярность показывала, что получившееся разбиение не лучше случайного.

\newpage~\vspace{-2.4cm}\\~
\begin{vwcol}[widths={0.99, 0.01}, rule=0pt]
\begin{figure}[H]
	\begin{tikzpicture}
		\begin{axis}[
		    table/col sep = semicolon,
		    height = 0.16\paperheight, 
		    width = 0.45\columnwidth,
		    xlabel = {$d$},
		    ylabel = {$Q$},
		    legend pos = north east,
		    no marks,
		    title = {cond-mat-2003},
		    cycle list name = diploma list,
		    y label style={at={(axis description cs:-0.14,.5)}, anchor=south},
		]
		\addplot table [x={d}, y={q}]{data/arg/d/cond_d.csv};
		\end{axis}
	\end{tikzpicture}
	\hskip 10pt
	\begin{tikzpicture}
		\begin{axis}[
		    table/col sep = semicolon,
		    height = 0.16\paperheight, 
		    width = 0.45\columnwidth,
		    xlabel = {$d$},
		    ylabel = {$Q$},
		    legend pos = south east,
		    no marks,
		    title = {caidaRouterLevel},
		    cycle list name = diploma list,
		    y label style={at={(axis description cs:-0.13,.5)}, anchor=south},
		]
		\addplot table [x={d}, y={q}]{data/arg/d/caida_d.csv};
		\end{axis}
	\end{tikzpicture}
	\begin{tikzpicture}
		\begin{axis}[
		    table/col sep = semicolon,
		    height = 0.16\paperheight, 
		    width = 0.45\columnwidth,
		    xlabel = {$d$},
		    ylabel = {$Q$},
		    legend pos = south east,
		    no marks,
		    title = {cnr-2000},
		    yticklabel style={/pgf/number format/fixed,
                  /pgf/number format/precision=3},
		    cycle list name = diploma list,
		    y label style={at={(axis description cs:-0.14,.5)}, anchor=south},
		]
		\addplot table [x={d}, y={q}]{data/arg/d/cnr_d.csv};
		\end{axis}
	\end{tikzpicture}
	\hskip 7pt
	\begin{tikzpicture}
		\begin{axis}[
		    table/col sep = semicolon,
		    height = 0.16\paperheight, 
		    width = 0.45\columnwidth,
		    xlabel = {$d$},
		    ylabel = {$Q$},
		    legend pos = south east,
		    no marks,
		    title = {eu-2005},
		    cycle list name = diploma list,
		    y label style={at={(axis description cs:-0.15,.5)}, anchor=south},
		]
		\addplot table [x={d}, y={q}]{data/arg/d/eu_d.csv};
		\end{axis}
	\end{tikzpicture}
	\caption{Зависимость модулярности от размера возмущения на четырёх графах}
	\label{fig:arg-d}
\end{figure}
~\\~\\~\\~\\~\\~\\~\\~\\~\\~\\~\\
\end{vwcol}

%%%%%%%%%%%%%%%%%%%%%%%%%%%%%%%%%%%%%%%%%%%%%%%%%%%%%%%%%%%%%%%%%%%%%%%%%%%%%%%%%%%%%%%%%%%%%%%%%%%%%%%%%%%%%%%

\subsection{Количество итераций в одном шаге}

Параметр $\sigma$ указывает, как часто меняется $k$ в ходе работы \emph{ARG}. Так как в функции качества используется медиана прироста модулярности, а не прирост модулярности за все $\sigma$ итераций --- при изменении $\sigma$ нет необходимости менять функцию качества, модулярность прироста будет оставаться приблизительно такой же по величине, в то время как прирост модулярности линейно зависит от $\sigma$. Зависимость модулярности от количества итераций в одном шаге при $d = 5,\;f(\mu, k) = -10 \ln \mu,\;\hat{k}_0 = 10$ принимает такой вид:

\begin{figure}[H]
	\begin{tikzpicture}
		\begin{axis}[
		    table/col sep = semicolon,
		    height = 0.17\paperheight, 
		    width = 0.32\columnwidth,
		    xlabel = {$\sigma$},
		    ylabel = {$Q$},
		    legend pos = north east,
		    no marks,
		    title = {cond-mat-2003},
		    cycle list name = diploma list,
		]
		\addplot table [x={sigma}, y={q}]{data/arg/sigma/cond_sigma.csv};
		\end{axis}
	\end{tikzpicture}
	\hskip -0.01\columnwidth
	\begin{tikzpicture}
		\begin{axis}[
		    table/col sep = semicolon,
		    height = 0.17\paperheight, 
		    width = 0.32\columnwidth,
		    xlabel = {$\sigma$},
		    legend pos = south east,
		    no marks,
		    title = {caidaRouterLevel},
		    cycle list name = diploma list,
		]
		\addplot table [x={sigma}, y={q}]{data/arg/sigma/caida_sigma.csv};
		\end{axis}
	\end{tikzpicture}
	\hskip -0.01\columnwidth
	\begin{tikzpicture}
		\begin{axis}[
		    table/col sep = semicolon,
		    height = 0.17\paperheight, 
		    width = 0.32\columnwidth,
		    xlabel = {$\sigma$},
		    legend pos = south east,
		    no marks,
		    title = {cnr-2000},
		    yticklabel style={/pgf/number format/fixed,
                  /pgf/number format/precision=3},
		    cycle list name = diploma list,
		]
		\addplot table [x={sigma}, y={q}]{data/arg/sigma/cnr_sigma.csv};
		\end{axis}
	\end{tikzpicture}
	\caption{Зависимость модулярности от значения $\sigma$ на трёх графах. $\sigma$ принимает значения от 50 до 1400}
\end{figure}
\newpage
На всех трёх графах модулярность несильно, но непредсказуемо меняется при изменении $\sigma$. Во всех случаях есть хорошие и плохие значения $\sigma$, но в целом значения параметра достаточно равнозначны.


%%%%%%%%%%%%%%%%%%%%%%%%%%%%%%%%%%%%%%%%%%%%%%%%%%%%%%%%%%%%%%%%%%%%%%%%%%%%%%%%%%%%%%%%%%%%%%%%%%%%%%%%%%%%%%%

\subsection{Начальное приближение}

Начальное приближение указывает, от какой точки будут строиться $k_1^{-}$ и $k_1^{+}$, с которыми будут проходить первые и вторые $\sigma$ итераций, соответственно. Зависимость модулярности от начального приближения $\hat{k}_0$ при $d = 5,\ f(\mu, k) = -10 \ln \mu,\ \sigma = 1000$ будет выглядеть следующим образом:

\begin{figure}[H]
	\begin{tikzpicture}
		\begin{axis}[
		    table/col sep = semicolon,
		    height = 0.16\paperheight, 
		    width = 0.31\columnwidth,
		    xlabel = {$\hat{k}_0$},
		    ylabel = {$Q$},
		    legend pos = north east,
		    no marks,
		    title = {cond-mat-2003},
		    cycle list name = diploma list,
		]
		\addplot table [x={k0}, y={q}]{data/arg/k0/cond_k0.csv};
		\end{axis}
	\end{tikzpicture}
	\hskip 0.01\columnwidth
	\begin{tikzpicture}
		\begin{axis}[
		    table/col sep = semicolon,
		    height = 0.16\paperheight, 
		    width = 0.31\columnwidth,
		    xlabel = {$\hat{k}_0$},
		    legend pos = south east,
		    no marks,
		    title = {caidaRouterLevel},
		    cycle list name = diploma list,
		]
		\addplot table [x={k0}, y={q}]{data/arg/k0/caida_k0.csv};
		\end{axis}
	\end{tikzpicture}
	\hskip 0.01\columnwidth
	\begin{tikzpicture}
		\begin{axis}[
		    table/col sep = semicolon,
		    height = 0.16\paperheight, 
		    width = 0.31\columnwidth,
		    xlabel = {$\hat{k}_0$},
		    legend pos = south east,
		    no marks,
		    title = {cnr-2000},
		    yticklabel style={/pgf/number format/fixed,
                  /pgf/number format/precision=3},
		    cycle list name = diploma list,
		]
		\addplot table [x={k0}, y={q}]{data/arg/k0/cnr_k0.csv};
		\end{axis}
	\end{tikzpicture}
	\caption{Зависимость модулярности от значения $\hat{k}_0$ на трёх графах}
	\label{fig:arg-k0}
\end{figure}

Как можно заметить, на разных графах модулярность ведёт себя по разному в зависимости от $\hat{k}_0$. График для графа \emph{cond-mat-2003} на рисунке \ref{fig:arg-k0} начинается от первого вхождения $\hat{k}_0 = 7$ (остальные от $\hat{k}_0 = 5$), так как при меньших значениях модулярность очень сильно падает. Это связано с тем, что $d = 5$ и следовательно $k_1^{-} = 1$, в то время как $RG_1$ часто работает сильно хуже $RG_k$ с параметром $k > 1$. Однако, как видно на рисунке, выгоднее использовать небольшое значение $\hat{k}_0$, хоть и большее $d + 1$.


%%%%%%%%%%%%%%%%%%%%%%%%%%%%%%%%%%%%%%%%%%%%%%%%%%%%%%%%%%%%%%%%%%%%%%%%%%%%%%%%%%%%%%%%%%%%%%%%%%%%%%%%%%%%%%%

\subsection{Коэффициент $\beta$ функции качества}

Коэффициент $\beta$ функции качества \eqref{eq:arg-f} указывает на то, насколько много внимания алгоритм обращает на величину точек $k^{-}_n$ и $k^{+}_n$ при выборе следующей оценки. Предполагалось, что чем больше $\beta$, тем быстрее будет работать алгоритм, однако возможно будет находить более плохие разбиения. На рисунке \ref{fig:arg-beta} изображены результаты работы $ARG$ с параметрами $d = 5,\ \sigma = 1000,\ \hat{k}_0 = 8$, функция качества $f(\mu, k) = -10(\ln \mu - \beta \ln k)$ с разными значениями $\beta$ на графе \emph{cond-mat-2003}.

Время работы сначала увеличивается на небольших значениях параметра $\beta$, а затем постепенно падает, в то время как модулярность при уменьшении $\beta$ сначала постепенно падает, но после некоторого переломного значения $\beta$ сильно падает.

\begin{figure}[H]
	\begin{tikzpicture}
		\begin{axis}[
		    table/col sep = semicolon,
		    height = 0.16\paperheight, 
		    width = 0.95\columnwidth,
		    xlabel = {$\beta$},
		    ylabel = {$Q$},
		    legend pos = north east,
		    x tick label style = {align = center, rotate = 20, anchor = north east},
		    no marks,
		    cycle list name = diploma list,
		]
		\addplot table [x={beta}, y={q}]{data/arg/beta/cond_beta.csv};
		\end{axis}
	\end{tikzpicture}
	\begin{tikzpicture}
		\begin{axis}[
		    table/col sep = semicolon,
		    height = 0.16\paperheight, 
		    width = 0.95\columnwidth,
		    xlabel = {$\beta$},
		    ylabel = {$t$},
		    legend pos = north east,
		    no marks,
		    x tick label style = {align = center, rotate = 20, anchor = north east},
		    cycle list name = diploma list,
		]
		\addplot table [x={beta}, y={time}]{data/arg/beta/cond_beta.csv};
		\end{axis}
	\end{tikzpicture}
	\caption{Зависимость модулярности и времени от параметра функции качества $\beta$ на графе \emph{cond-mat-2003}}
	\label{fig:arg-beta}
\end{figure}


%%%%%%%%%%%%%%%%%%%%%%%%%%%%%%%%%%%%%%%%%%%%%%%%%%%%%%%%%%%%%%%%%%%%%%%%%%%%%%%%%%%%%%%%%%%%%%%%%%%%%%%%%%%%%%%

\subsection{Сравнение ARG и RG}
\label{subsec:arg-res}

На таблицах \ref{tab:arg-res-q} и \ref{tab:arg-res-t} показано сравнение результатов работы рандомизированного жадного алгоритма и адаптивного рандомизированного жадного алгоритма. В качестве входных данных для рассмотрения выбраны 7 тестовых графов.

Рандомизированный жадный алгоритм в таблицах обозначается как $RG_k$. Адаптивный рандомизированный жадный алгоритм обозначается как $ARG_{\beta}$ с параметром функции качества $\beta$ в индексе, остальные параметры при этом равны $d = 5,\;f(\mu, k) = -10 (\ln \mu - \beta \ln k),\;\sigma = 1000,\;\hat{k}_0 = 8$. Такой набор параметров не гарантирует наилучший результат на каждом графе или даже наилучший в среднем по всем графам результат для адаптивного алгоритма, однако показывает неплохие результаты.

Оба алгоритма --- рандомизированные, поэтому на каждом отдельном запуске результат может оказаться очень хорошим или сравнительно плохим, однако даст мало сведений о качестве алгоритма. Поэтому для каждого графа производилось определённое количество запусков, и в качестве модулярности бралась медиана значений, а в качестве времени --- среднее, для повышения точности измерения времени.

Для графа \emph{as-22july06} производился 101 запуск, для \emph{cond-mat-2003} --- 51 запуск, для \emph{caidaRouterLevel} и \emph{cnr-2000} --- 11 запусков, для \emph{in-2004} --- пять запусков, и для \emph{eu-2005} --- три запуска.

Кроме того, в таблицу вошли результаты работы над автоматически сгенерированным графом \emph{auto40}, который уже появлялся в работе, к примеру в рисунке \ref{fig:q-auto13}. Для этого графа для повышения точности производилось 11 запусков.

\begin{table}[H]
	\caption{Модулярности разбиений, полученных в результате работы рандомизированного жадного алгоритма и адаптивного рандомизированного жадного алгоритма с разными параметрами на тестовых графах}
 	\label{tab:arg-res-q}
 	{\scriptsize
 	\begin{tabularx}{\textwidth}{Xrrrrrrrrr}\hline
 						& $RG_1$	& $RG_3$	& $RG_{10}$	& $RG_{50}$	& $ARG_0$	&$ARG_{0.01}$	&$ARG_{0.05}$	& $ARG_{0.1}$	& $ARG_{0.2}$	\\\hline
 	as-22july06			& 0.65281	& 0.64658	& 0.64024	& 0.63479	& 0.64262	& 0.64041	& 0.64264	& 0.64134	& 0.64192	\\
 	cond-mat-2003		& 0.00012	& 0.19727	& 0.70738	& 0.69403	& 0.71129	& 0.71232	& 0.71193	& 0.69749	& 0.56631	\\
 	auto40			 	& 0.78944	& 0.79988	& 0.80417	& 0.80273	& 0.80136	& 0.80145	& 0.80174	& 0.80152	& 0.80102	\\
 	caidaRouterLevel 	& 0.01938	& 0.81101	& 0.79883	& 0.79300	& 0.79970	& 0.80114	& 0.80216	& 0.80059	& 0.80176	\\
 	cnr-2000			& 0.90237	& 0.91192	& 0.91144	& 0.90997	& 0.91028	& 0.91041	& 0.91039	& 0.91108	& 0.91075	\\
 	eu-2005				& 0.92765	& 0.92559	& 0.91780	& 0.90416	& 0.91062	& 0.91047	& 0.91048	& 0.91199	& 0.91242	\\
 	in-2004				& 0.00026	& 0.97836	& 0.97185	& 0.97596	& 0.97614	& 0.97615	& 0.97616	& 0.97618	& 0.97588	\\\hline
 	\end{tabularx}
 	}
\end{table}

\begin{table}[H]
	\caption{Время работы (в миллисекундах) рандомизированного жадного алгоритма и адаптивного рандомизированного жадного алгоритма с разными параметрами на тестовых графах}
 	\label{tab:arg-res-t}
 	{\scriptsize
 	\begin{tabularx}{\textwidth}{Xrrrrrrrrr}\hline
 						& $RG_1$	& $RG_3$	& $RG_{10}$	& $RG_{50}$	& $ARG_0$	&$ARG_{0.01}$	&$ARG_{0.05}$	& $ARG_{0.1}$	& $ARG_{0.2}$	\\\hline
 	as-22july06			& 177		& 189		& 231		& 464		& 238		& 241		& 238		& 233		& 222		\\
 	cond-mat-2003		& 58		& 184		& 463		& 931		& 477		& 477		& 474		& 476		& 351		\\
 	auto40			 	& 4,652		& 4,591		& 6,017		& 12,558	& 6,526		& 6,807		& 6,479		& 6,428		& 6,105		\\
 	caidaRouterLevel 	& 852		& 9,114		& 10,244	& 15,217	& 11,573	& 11,607	& 11,514	& 11,509	& 11,220	\\
 	cnr-2000			& 26,083	& 26,056	& 27,137	& 33,592	& 30,033	& 30,465	& 29,054	& 29,971	& 29,784	\\
 	eu-2005				& 202,188	& 200,686	& 207,689	& 246,170	& 233,761	& 226,869	& 225,748	& 226,427	& 266,038	\\
 	in-2004				& 9,208		& 487,953	& 553,196	& 607,408	& 622,813	& 625,124	& 617,345	& 640,454	& 616,187	\\\hline
 	\end{tabularx}
 	}
\end{table}

В большинстве случаев один или несколько параметров $k$ дают рандомизированному жадному алгоритму лучшие результаты, чем результаты адаптивного рандомизированного алгоритма, однако адаптивный вариант даёт более стабильные результаты. Можно заметить, что небольшие значения параметра $\beta$ дают лучшие результаты, чем нулевое значение, а при более б\'{о}льших значениях параметра время работы алгоритма уменьшается, но незначительно. В таблицу не попали результаты алгоритмов со значениями $\beta$ больше 0.2, где время работы действительно сильно снижалось, однако и значения модулярности получались слишком маленькими.

$ARG$ можно сравнить с $RG_{10}$, так как он тоже даёт стабильные результаты, в отличии от $RG_{1},\ RG_{3}$. Однако заметно, что \emph{ARG} в большинстве случаев даёт б\'{о}льшую модулярность.


\bibliographystyle{unsrt}
\bibliography{diploma}

\end{document}