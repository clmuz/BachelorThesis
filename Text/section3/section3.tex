%%%%%%%%%%%%%%%%%%%%%%%%%%%%%%%%%%%%%%%%%%%%%%%%%%%%%%%%%%%%%%%%%%%%%%%%%%%%%%%%%%%%%%%%%%%%%%%%%%%%%%%%%%%%%%%

\section{Ансамблевая стратегия}


%%%%%%%%%%%%%%%%%%%%%%%%%%%%%%%%%%%%%%%%%%%%%%%%%%%%%%%%%%%%%%%%%%%%%%%%%%%%%%%%%%%%%%%%%%%%%%%%%%%%%%%%%%%%%%%

\subsection{Адаптивный рандомизированный жадный алгоритм в ансамблевой стратегии}

Ансамблевая стратегия была более подробно описана в подразделе \ref{subsec:ens}, однако общая схема работы заключается в том, что сначала $s$ начальных алгоритмов выделяют  сообщества, а те узлы, в которых они разошлись по мнению, затем разбивает на сообщества финальный алгоритм. Трудоёмкость таким образом складывается из $s$ трудоёмкостей начальных алгоритмов и трудоёмкости финального алгоритма. Таким образом имеет смысл в качестве начальных алгоритмов брать алгоритмы, работающие быстро. Однако в случае, если начальные алгоритмы очень плохие, то есть дают разбиения не лучше случайного --- модулярность ансамблевой стратегии будет приблизительно равна модулярности финального алгоритма. 

Далее разбиение, полученное начальными алгоритмами называется \emph{начальным} разбиением, полученное финальным алгоритмом --- \emph{финальным} разбиением. Также  разбиение, равное максимальному перекрытию начальных разбиений далее называется \emph{промежуточным}.

Если промежуточное разбиение состоит из достаточно большого количества сообществ (как минимум в несколько раз больше $2\sigma$) --- на нём разумно использовать в качестве финального алгоритма адаптивную версию рандомизированного жадного алгоритма, чтобы получать стабильные результаты. Однако на небольших графах адаптивный алгоритм не имеет смысла.

Так же, если граф, подающийся на вход, достаточно большой --- в качестве начального алгоритма на нём имеет смысл использовать адаптивный алгоритм.

В таблице \ref{tab:es1-q} сравниваются результаты работы ансамблевой стратегии с разными начальными и финальными алгоритмами. В качестве начального алгоритма используюся два рандомизированных жадных алгоритма с параметрами $k = 3$ (обозначается $RG_3$) и $k = 10$ (обозначается $RG_{10}$), и адаптивный рандомизированный жадный алгоритм с параметрами $d = 5,\ f(\mu, k) = -10(\ln \mu - 0.02 \ln k),\ \sigma = 1000,\ \hat{k}_0 = 8$ (обозначается $ARG$). А в качестве финального алгоритма используются только рандомизированные жадные алгоримы с $k$ равными 3 и 10, так как на рассматриваемых графах промежуточное разбиение состоит из недостаточно большого количества сообществ, чтобы использовать адаптивный рандомизированный алгоритм.

Ансамблевая стратегия используется с параметром $s = 10$, а обозначается как $ES(init, final)$, где в первом аргументе указывается начальный алгоритм, а во втором --- финальный. Алгоритмы на графе \emph{cond-mat-2003} запускались 21 раз, на \emph{caidaRouterLevel} и \emph{auto40} --- три раза.

\begin{table}[H]
	\caption{Сравнение модулярности и времени ансамблевой стратегии с $RG(3),\ RG(10)$ и $ARG$ в качестве начальных алгоритмов и $RG(3)$ и $RG(10)$ в качестве финальных алгоритмов}
	\label{tab:es1-q}
	\begin{tabularx}{\textwidth}{Xrrrrrr}	\hline
								& \multicolumn{2}{c}{cond-mat-2003}	& \multicolumn{2}{c}{auto40}	& \multicolumn{2}{c}{caidaRouterLevel} 	\\
								& \multicolumn{1}{c}{Q} & \multicolumn{1}{c}{t} & \multicolumn{1}{c}{Q} & \multicolumn{1}{c}{t} & \multicolumn{1}{c}{Q} & \multicolumn{1}{c}{t} \\\hline
		$ES(RG_3, RG_3)$		& 0.16840	& 2.0	& 0.806277	& 47.0	& 0.84078	& 94.0	\\
		$ES(RG_{10}, RG_3)$		& 0.44934	& 4.7	& 0.806325	& 57.7	& 0.84031	& 104.2	\\
		$ES(ARG, RG_3)$			& 0.42708	& 4.8	& 0.806276	& 65.7	& 0.83671	& 115.1	\\
		$ES(RG_3, RG_{10})$		& 0.71155	& 2.3 	& 0.806445	& 46.6	& 0.85372	& 93.7	\\
		$ES(RG_{10}, RG_{10})$	& 0.74794	& 4.8	& 0.806445	& 57.2	& 0.84448	& 104.3	\\
		$ES(ARG, RG_{10})$		& 0.74872	& 4.9	& 0.806472	& 65.8	& 0.85279	& 118.5	\\\hline
	\end{tabularx}
\end{table}

Граф \emph{cond-mat-2003} плохо разбивается случайными жадными алгоритмами с маленьким $k$, если обратить внимание на таблицу \ref{tab:es1-q} --- в случаях, когда $RG_3$ используется в качестве начального алгоритма --- модулярность ансамблевой стратегии сравнима с модулярностью финального алгоритма на этом графе (можно увидеть в таблице \ref{tab:arg-res-q}). В случаях, когда $RG_3$ используется в качестве финального алгоритма --- модулярность получается сравнительно плохой.

Заранее неизвестно, при каких $k$ алгоритм будет хорошо работать, а при каких плохо, кроме того неизвестно по результату, хорошее ли это разбиение для конкретного графа или нет. Кроме того, ансамблевая стратегия работает заметно дольше, чем рандомизированный жадный алгоритм, поэтому запускать ансамблевую стратегию несколько раз для определения хороших параметров не выгодно. Таким образом, выгодно использовать адаптивный рандомизированный жадный алгоритм и в качестве начального алгоритма, и в качестве финального, если промежуточное разбиение подходит по размерам.


%%%%%%%%%%%%%%%%%%%%%%%%%%%%%%%%%%%%%%%%%%%%%%%%%%%%%%%%%%%%%%%%%%%%%%%%%%%%%%%%%%%%%%%%%%%%%%%%%%%%%%%%%%%%%%%

\subsection{Адаптивное построение промежуточного разбиения}

Так как адаптивный рандомизированный жадный алгоритм не может работать на графах с маленьким количеством узлов или на разбиениях с маленьким количеством сообществ --- была предложена другая схема адаптивной ансамблевой стратегии:

\begin{enumerate}
	\item На вход подаётся граф $G$. Выбор параметров алгоритма: $d \in \mathbb{N}$ --- размер возмущения, $f(Q, k): \mathbb{R} \times \mathbb{N} \rightarrow \mathbb{R}$ --- функция качества, $l \in \mathbb{N}$ --- количество шагов, $r \in (0, 1]$ --- доля лучших начальных разбиений, учавствующих в создании промежуточного разбиения. И наконец $\hat{k}_0 \in \mathbb{N}$ --- начальная центральная точка. Установка счётчика $n = 0$. Установка список начальных разбиений $S = \emptyset$
	\item Увеличение счётчика $n \rightarrow n + 1$
	\item Вычисление следующий параметров рандомизированного жадного алгоритма $k_n^{-} = \max\{1, \hat{k}_{n - 1} - d\}$ и $k_n^{+} = \hat{k}_{n - 1} + d$
	\item Выделение сообществ в графе $G$ рандомизированным жадным алгоритмом с параметром $k = k_n^{-}$, пара из результирующего разбиения $P_n^{-}$ и его модулярности $Q_n^{-}$ записывается в список $S$
	\item Выделение сообществ в графе $G$ рандомизированным жадным алгоритмом с параметром $k = k_n^{+}$, пара из результирующего разбиения $P_n^{+}$ и его модулярности $Q_n^{+}$ записывается в список $S$
	\item Вычисляются функции качества $y_n^{-} = f(Q_n^{-}, k_n^{-}),\ y_n^{+} = f(Q_n^{+}, k_n^{+})$
	\item Вычисляется следующая центральная точка аналогично \eqref{eq:arg-centre}
	\begin{equation}\label{eq:aes-centre}
		\hat{k}_n = \max\left\{1, \left[\hat{k}_{n - 1} - \frac{y_n^{+} - y_n^{-}}{k_n^{+} - k_n^{-}}\right]\right\}
	\end{equation}
	\item Если $n \ne l$ --- переход на второй пункт, иначе --- на следующий
	\item По $\lceil r|S| \rceil = \lceil 2rl \rceil$ лучшим разбиениям (с наибольшей модулярностью) в $S$ создаётся промежуточное множество, которое затем передаётся в финальный алгоритм
\end{enumerate}

При этом граф $G$ может быть иметь сколь угодно маленькое количество узлов.

Далее в работе ансамблевая стратегия с адаптивным построением промежуточного разбиения будет называться \emph{адаптивной ансамблевой стратегией}.


%%%%%%%%%%%%%%%%%%%%%%%%%%%%%%%%%%%%%%%%%%%%%%%%%%%%%%%%%%%%%%%%%%%%%%%%%%%%%%%%%%%%%%%%%%%%%%%%%%%%%%%%%%%%%%%

\subsection{Размер возмущения и чувствительность к перепадам функции качества}

Функцией качества для построения промежуточного разбиения выбрана $f(Q, k) = -\alpha(\ln Q - \beta \ln k)$. Мотивация этого выбора аналогична выбору функции качества для адаптивного рандомизированного алгоритма, описанная в подразделе \ref{subsec:arg-f}. Аналогично, параметр $\alpha$ указывает на чувствительность алгоритма к разностям значений функции качества, а параметр $\beta$ указывает на значимость времени работы алгоритма.

\begin{figure}[H]
	\begin{tikzpicture}
		\begin{semilogxaxis}[
		    table/col sep = semicolon,
		    height = 0.165\paperheight, 
		    width = 0.83\columnwidth,
		    xlabel = {$\alpha$},
		    ylabel = {$Q$},
		    legend pos = outer north east,
		    no marks,
		    title = {celegans\_metabolic},
		    yticklabel style={/pgf/number format/fixed,
                  /pgf/number format/precision=3},
            y label style={at={(axis description cs:0,.5)}, anchor=south},
		]
		\addplot table [x={alpha}, y={q}]{data/es/alphad/celegans_1.csv};
		\addplot table [x={alpha}, y={q}]{data/es/alphad/celegans_2.csv};
		\legend{{$d = 1$}, {$d = 2$}};
		\end{semilogxaxis}
	\end{tikzpicture}
	\caption{Зависимость модулярности от параметра $\alpha$ при $d = 1$ и $d = 2$ в работе адаптивной ансамблевой стратегии на графе \emph{celegans\_metabolic}}
	\label{fig:es-alpha-d1}
\end{figure}
\begin{figure}[H]
	\begin{tikzpicture}
		\begin{semilogxaxis}[
		    table/col sep = semicolon,
		    height = 0.15\paperheight, 
		    width = 0.825\columnwidth,
		    xlabel = {$\alpha$},
		    ylabel = {$Q$},
		    legend pos = outer north east,
		    no marks,
		    title = {jazz},
		    yticklabel style={/pgf/number format/fixed,
                  /pgf/number format/precision=4},
            y label style={at={(axis description cs:-0.005,.5)}, anchor=south},
		]
		\addplot table [x={alpha}, y={q}]{data/es/alphad/jazz_1.csv};
		\addplot table [x={alpha}, y={q}]{data/es/alphad/jazz_2.csv};
		\legend{{$d = 1$}, {$d = 2$}};
		\end{semilogxaxis}
	\end{tikzpicture}
	\begin{tikzpicture}
		\begin{semilogxaxis}[
		    table/col sep = semicolon,
		    height = 0.15\paperheight, 
		    width = 0.83\columnwidth,
		    xlabel = {$\alpha$},
		    ylabel = {$Q$},
		    legend pos = outer north east,
		    no marks,
		    title = {as-22july06},
		    yticklabel style={/pgf/number format/fixed,
                  /pgf/number format/precision=3},
            y label style={at={(axis description cs:0,.5)}, anchor=south},
		]
		\addplot table [x={alpha}, y={q}]{data/es/alphad/july_1.csv};
		\addplot table [x={alpha}, y={q}]{data/es/alphad/july_2.csv};
		\legend{{$d = 1$}, {$d = 2$}};
		\end{semilogxaxis}
	\end{tikzpicture}
	\begin{tikzpicture}
		\begin{semilogxaxis}[
		    table/col sep = semicolon,
		    height = 0.15\paperheight, 
		    width = 0.85\columnwidth,
		    xlabel = {$\alpha$},
		    ylabel = {$Q$},
		    legend pos = outer north east,
		    no marks,
		    title = {netscience}
		]
		\addplot table [x={alpha}, y={q}]{data/es/alphad/netscience_1.csv};
		\addplot table [x={alpha}, y={q}]{data/es/alphad/netscience_2.csv};
		\legend{{$d = 1$}, {$d = 2$}};
		\end{semilogxaxis}
	\end{tikzpicture}
	\caption{Продолжение рисунка \ref{fig:es-alpha-d1}. Зависимость модулярности от параметра $\alpha$ при $d = 1$ и $d = 2$ в работе адаптивной ансамблевой стратегии на графах \emph{jazz}, \emph{as-22july06} и \emph{netscience}}
	\label{fig:es-alpha-d2}
\end{figure}

На рисунках \ref{fig:es-alpha-d1} и \ref{fig:es-alpha-d2} изображены результаты адаптивной стратегии с параметром $\beta = 0$ и остальными параметрами равными $f(Q, k) = -\alpha \ln Q,\ \hat{k}_0 = 5,\ l = 10,\ r = 1$. В качестве финального алгоритма используется $RG_{10}$.

Значения модулярности колеблются в очень небольшом промежутке и на вышеприведённых четырёх графах $\alpha$, примерно равное $10^3$ даёт в среднем лучшие результаты. При этом на графах \emph{celegans\_metabolic}, \emph{jazz} и \emph{netscience} зависимости модулярности от $\alpha$ выглядят одинакого для $d = 1$ и $d = 2$, а на \emph{as-22july06} зависимость при параметре $d = 2$ лежит в среднем выше, чем при параметре $d = 1$. Б\'{о}льшие значения параметра $d$ чреваты тем, что алгоритм будет слишком сильно отклоняться от оптимального значения параметра $k$.

На рисунке \ref{fig:es-alpha-d3} представлено изменение $k^{-}_n$ и $k^{+}_n$ в ходе построения промежуточного разбиения от номера шага при $\alpha = 994$ на графе \emph{as-22july06} и $\alpha = 1020$ на графе \emph{netscience}, в обоих случаях $d = 2$, а остальные параметры те же, что и на рисунках \ref{fig:es-alpha-d1} и \ref{fig:es-alpha-d2}. Отдельно измерено, что наилучшее в среднем значение рандомизированный жадный алгоритм даст при $k = 1$ на графе \emph{as-22july06} и при очень больших $k$ на графе \emph{netscience}.

\begin{figure}[H]
	\begin{tikzpicture}
		\begin{axis}[
		    table/col sep = semicolon,
		    height = 0.16\paperheight, 
		    width = 0.49\columnwidth,
		    xlabel = {$n$},
		    ylabel = {$k^{\pm}_n$},
		    legend pos = north east,
		    no marks,
		    title = {as-22july06}
		]
		\addplot table [x expr = {\thisrowno{0} / 2}, y={k}]{data/es/alphad/_july994.csv};
		\end{axis}
	\end{tikzpicture}
	\hskip 0.01\columnwidth
	\begin{tikzpicture}
		\begin{axis}[
		    table/col sep = semicolon,
		    height = 0.16\paperheight, 
		    width = 0.49\columnwidth,
		    xlabel = {$n$},
		    ylabel = {$k^{\pm}_n$},
		    legend pos = north east,
		    no marks,
		    title = {netscience}
		]
		\addplot table [x expr = {\thisrowno{0} / 2}, y={k}]{data/es/alphad/_netscience1020.csv};
		\end{axis}
	\end{tikzpicture}
	\caption{Изменение $k^{-}_n$ и $k^{+}_n$ с изменением номера шага в работе адаптивной ансамблевой стратегии на графах \emph{as--22july06} и \emph{netscience}}
	\label{fig:es-alpha-d3}
\end{figure}

Заметно, как быстро алгоритмы находят близкие к оптимальным значениям центральные точки $\hat{k}_n$, вокруг которых.


%%%%%%%%%%%%%%%%%%%%%%%%%%%%%%%%%%%%%%%%%%%%%%%%%%%%%%%%%%%%%%%%%%%%%%%%%%%%%%%%%%%%%%%%%%%%%%%%%%%%%%%%%%%%%%%

\subsection{Начальная центральная точка}

Начальная центральная точка $\hat{k}_0$ указывает, рядом с какой точкой будут находиться параметры первых двух запусков рандомизированного жадного алгоритма. Очень большим делать начальную центральную точку не имеет смысла, так как довольно много графов имеет небольшой оптимальный параметр.

\begin{figure}[H]
	\begin{tikzpicture}
		\begin{axis}[
		    table/col sep = semicolon,
		    height = 0.145\paperheight,
		    width = 0.465\columnwidth,
		    xlabel = {$\hat{k}_0$},
		    ylabel = {$Q$},
		    no marks,
		    title = {celegans\_metabolic},
		    y label style={at={(axis description cs:0,.5)}, anchor=south},
		    yticklabel style={/pgf/number format/fixed,
                  /pgf/number format/precision=4},
		]
		\addplot table [x = {k0}, y={q}]{data/es/k0/celegans_k0.csv};
		\end{axis}
	\end{tikzpicture}
	\hskip 10pt
	\begin{tikzpicture}
		\begin{axis}[
		    table/col sep = semicolon,
		    height = 0.145\paperheight,
		    width = 0.465\columnwidth,
		    xlabel = {$\hat{k}_0$},
		    ylabel = {$Q$},
		    no marks,
		    title = {jazz},
		    y label style={at={(axis description cs:0,.5)}, anchor=south},
		    yticklabel style={/pgf/number format/fixed,
                  /pgf/number format/precision=4},
		]
		\addplot table [x = {k0}, y={q}]{data/es/k0/jazz_k0.csv};
		\end{axis}
	\end{tikzpicture}\\ %	\hskip 10pt
	\begin{tikzpicture}
		\begin{axis}[
		    table/col sep = semicolon,
		    height = 0.145\paperheight,
		    width = 0.465\columnwidth,
		    xlabel = {$\hat{k}_0$},
		    ylabel = {$Q$},
		    legend pos = north east,
		    no marks,
		    title = {as-22july06},
		    y label style={at={(axis description cs:0,.5)}, anchor=south},
		    yticklabel style={/pgf/number format/fixed,
                  /pgf/number format/precision=3},
		]
		\addplot table [x = {k0}, y={q}]{data/es/k0/july_k0.csv};
		\end{axis}
	\end{tikzpicture}
	\hskip 10pt
	\begin{tikzpicture}
		\begin{axis}[
		    table/col sep = semicolon,
		    height = 0.145\paperheight,
		    width = 0.465\columnwidth,
		    xlabel = {$\hat{k}_0$},
		    ylabel = {$Q$},
		    legend pos = north east,
		    no marks,
		    title = {netscience},
		    y label style={at={(axis description cs:0,.5)}, anchor=south},
		    yticklabel style={/pgf/number format/fixed,
                  /pgf/number format/precision=4},
		]
		\addplot table [x = {k0}, y={q}]{data/es/k0/netscience_k0.csv};
		\end{axis}
	\end{tikzpicture}
	\caption{Зависимость модулярности от начальной центральной точки $\hat{k}_0$ в работе адаптивной ансамблевой стратегии на четырёх графах}
	\label{fig:es-k0}
\end{figure}

Однако, как показывает рисунок \ref{fig:es-k0}, на самом деле модулярность очень слабо зависит от начальной центральной точки. Для получения рисунка ансамблевая стратегия запускалась с параметрами $d = 2,\ f(Q, k) = -1000 \ln Q,\ l = 10,\ r = 1$, в качестве финального алгоритма используется $RG_{10}$.


%%%%%%%%%%%%%%%%%%%%%%%%%%%%%%%%%%%%%%%%%%%%%%%%%%%%%%%%%%%%%%%%%%%%%%%%%%%%%%%%%%%%%%%%%%%%%%%%%%%%%%%%%%%%%%%

\subsection{Количество шагов и доля хороших разбиений}

Множество $S$ состоит из $2l$ разбиений, из которых затем выбираются $\lceil 2rl \rceil$ лучших разбиений, учавствующих в создании промежуточного разбиения. Предполагается, что при слишком маленьких $l$ алгоритм не успеет найти оптимальную точку $k$, в то время как при $r \rightarrow 1$ в создании промежуточного разбиения примут участие много плохих разбиений, полученных на первых шагах. При $d = 2,\ f(Q, k) = -1000 \ln Q,\ \hat{k}_0 = 5,\ l = 10$ и финальном алгоритме $RG_{10}$ зависимость модулярности от доли хороших разбиений выглядит следующим образом:

\begin{figure}[H]
	\begin{tikzpicture}
		\begin{axis}[
		    table/col sep = semicolon,
		    height = 0.16\paperheight,
		    width = 0.3\columnwidth,
		    xlabel = {$r$},
		    ylabel = {$Q$},
		    legend pos = north east,
		    no marks,
		    title = {celegans\_metabolic},
		    y label style={at={(axis description cs:0.01,.5)}, anchor=south},
		    yticklabel style={/pgf/number format/fixed,
                  /pgf/number format/precision=4},
		]
		\addplot table [x = {r}, y={q}]{data/es/r/celegans_r.csv};
		\end{axis}
	\end{tikzpicture}
	\hskip -0.01\columnwidth
	\begin{tikzpicture}
		\begin{axis}[
		    table/col sep = semicolon,
		    height = 0.16\paperheight,
		    width = 0.3\columnwidth,
		    xlabel = {$r$},
		    ylabel = {$Q$},
		    legend pos = north east,
		    no marks,
		    title = {jazz},
		    y label style={at={(axis description cs:0.01,.5)}, anchor=south},
		    yticklabel style={/pgf/number format/fixed,
                  /pgf/number format/precision=4},
		]
		\addplot table [x = {r}, y={q}]{data/es/r/jazz_r.csv};
		\end{axis}
	\end{tikzpicture}
	\hskip 0\columnwidth
	\begin{tikzpicture}
		\begin{axis}[
		    table/col sep = semicolon,
		    height = 0.16\paperheight,
		    width = 0.3\columnwidth,
		    xlabel = {$r$},
		    ylabel = {$Q$},
		    legend pos = north east,
		    no marks,
		    title = {as-22july06},
		    y label style={at={(axis description cs:0.05,.5)}, anchor=south},
		    yticklabel style={/pgf/number format/fixed,
                  /pgf/number format/precision=3},
		]
		\addplot table [x = {r}, y={q}]{data/es/r/july_r.csv};
		\end{axis}
	\end{tikzpicture}
	\caption{Зависимость модулярности от доли хороших разбиений, учавствующих в создании промежуточного разбиения, $r$ в работе адаптивной ансамблевой стратегии на трёх графах}
	\label{fig:es-r}
\end{figure}

Видно, что маленькие $r$ дают плохие разбиения, что логично, а затем на разных графах алгоритмы ведут себя по разному, хотя и разница между модулярностями при разных значениях $r$ небольшия.

На рисунке \ref{fig:es-l-q} изображена зависимость модулярности от количества шагов $l$ при $r = 0.5$ и $r = 1$. Остальные параметры при этом были равны $d = 2,\ f(Q, k) = -1000 \ln Q,\ \hat{k}_0 = 5$. В качестве финального алгоритма, как и в предыдущих примерах, использовался рандомизированный жадный алгоритм с параметром $k = 10$.

Стоит отметить, что несмотря на то, что модулярность не сильно меняется при росте $l$ --- время работы растёт практически линейно, как можно увидеть на рисунке \ref{fig:es-l-t}. Можно так же заметить, что время работы при $r = 0.5$ меньше времени при $r = 1$, это связано с тем, что некоторое время тратится на создание максимального перекрытия.

Таким образом, нет смысла брать большое $l$, повышая время работы, но не гарантируя улучшения получившихся разбиений.

\begin{figure}[H]
	\begin{tikzpicture}
		\begin{axis}[
		    table/col sep = semicolon,
		    height = 0.16\paperheight,
		    width = 0.4\columnwidth,
		    xlabel = {$l$},
		    ylabel = {$t$},
		    legend pos = north east,
		    no marks,
		    title = {celegans\_metabolic},
		    y label style={at={(axis description cs:0,.5)}, anchor=south},
		    yticklabel style={/pgf/number format/fixed,
                  /pgf/number format/precision=4},
		]
		\addplot table [x = {l}, y={time}]{data/es/l/celegans_l_0.5.csv};
		\addplot table [x = {l}, y={time}]{data/es/l/celegans_l_1.csv};
		\end{axis}
	\end{tikzpicture}
	\begin{tikzpicture}
		\begin{axis}[
		    table/col sep = semicolon,
		    height = 0.16\paperheight,
		    width = 0.4\columnwidth,
		    xlabel = {$l$},
		    ylabel = {$t$},
		    legend style = {
		    	cells = {anchor = west},
		    	legend pos = outer north east
		    },
		    no marks,
		    title = {jazz},
		    y label style={at={(axis description cs:0,.5)}, anchor=south},
		    yticklabel style={/pgf/number format/fixed,
                  /pgf/number format/precision=4},
		]
		\addplot table [x = {l}, y={time}]{data/es/l/jazz_l_0.5.csv};
		\addplot table [x = {l}, y={time}]{data/es/l/jazz_l_1.csv};
		\legend{{$r = 0.5$}, {$r = 1$}};
		\end{axis}
	\end{tikzpicture}
	\caption{Время работы от количества шагов $l$ в работе адаптивной ансамблевой стратегии на четырёх графах при $r = 0.5$ и $r = 1$}
	\label{fig:es-l-t}
\end{figure}

\begin{figure}[H]
	\begin{tikzpicture}
		\begin{axis}[
		    table/col sep = semicolon,
		    height = 0.16\paperheight,
		    width = 0.4\columnwidth,
		    xlabel = {$l$},
		    ylabel = {$Q$},
		    legend pos = north east,
		    no marks,
		    title = {celegans\_metabolic},
		    y label style={at={(axis description cs:0,.5)}, anchor=south},
		    yticklabel style={/pgf/number format/fixed,
                  /pgf/number format/precision=4},
		]
		\addplot table [x = {l}, y={q}]{data/es/l/celegans_l_0.5.csv};
		\addplot table [x = {l}, y={q}]{data/es/l/celegans_l_1.csv};
		\end{axis}
	\end{tikzpicture}
	\begin{tikzpicture}
		\begin{axis}[
		    table/col sep = semicolon,
		    height = 0.16\paperheight,
		    width = 0.4\columnwidth,
		    xlabel = {$l$},
		    ylabel = {$Q$},
		    legend style = {
		    	cells = {anchor = west},
		    	legend pos = outer north east
		    },
		    no marks,
		    title = {jazz},
		    y label style={at={(axis description cs:0,.5)}, anchor=south},
		    yticklabel style={/pgf/number format/fixed,
                  /pgf/number format/precision=4},
		]
		\addplot table [x = {l}, y={q}]{data/es/l/jazz_l_0.5.csv};
		\addplot table [x = {l}, y={q}]{data/es/l/jazz_l_1.csv};
		\legend{{$r = 0.5$}, {$r = 1$}};
		\end{axis}
	\end{tikzpicture}
	\begin{tikzpicture}
		\begin{axis}[
		    table/col sep = semicolon,
		    height = 0.16\paperheight,
		    width = 0.4\columnwidth,
		    xlabel = {$l$},
		    ylabel = {$Q$},
		    legend pos = north east,
		    no marks,
		    title = {as-22july06},
		    y label style={at={(axis description cs:0,.5)}, anchor=south},
		    yticklabel style={/pgf/number format/fixed,
                  /pgf/number format/precision=3},
		]
		\addplot table [x = {l}, y={q}]{data/es/l/july_l_0.5.csv};
		\addplot table [x = {l}, y={q}]{data/es/l/july_l_1.csv};
		\end{axis}
	\end{tikzpicture}
	% \hskip 10pt
	\begin{tikzpicture}
		\begin{axis}[
		    table/col sep = semicolon,
		    height = 0.16\paperheight,
		    width = 0.4\columnwidth,
		    xlabel = {$l$},
		    ylabel = {$Q$},
		    legend style = {
		    	cells = {anchor = west},
		    	legend pos = outer north east
		    },
		    no marks,
		    title = {netscience},
		    y label style={at={(axis description cs:0,.5)}, anchor=south},
		    yticklabel style={/pgf/number format/fixed,
                /pgf/number format/precision=4},
		]
		\addplot table [x = {l}, y={q}]{data/es/l/netscience_l_0.5.csv};
		\addplot table [x = {l}, y={q}]{data/es/l/netscience_l_1.csv};
		\legend{{$r = 0.5$}, {$r = 1$}};
		\end{axis}
	\end{tikzpicture}
	\caption{Зависимость модулярности от количества шагов $l$ в работе адаптивной ансамблевой стратегии на четырёх графах при $r = 0.5$ и $r = 1$}
	\label{fig:es-l-q}
\end{figure}


%%%%%%%%%%%%%%%%%%%%%%%%%%%%%%%%%%%%%%%%%%%%%%%%%%%%%%%%%%%%%%%%%%%%%%%%%%%%%%%%%%%%%%%%%%%%%%%%%%%%%%%%%%%%%%%

\subsection{Время работы}

Одним из способов влиять на время работы является увеличения коэффициента $\beta$ в функции качества $f(Q, k) = -\alpha(\ln Q - \beta \ln k)$. Это обусловлено тем, что при наличии хорошего финального алгоритма, или при использовании итеративной схемы ансамблевой стратегии, время работы начальных алгоритмом очень критично, в то время как увеличение модулярности начальных алгоритмов на сотые доли может не принести большой выгоды глобально. Стоит отметить, что на графах, у которых существует ярко выраженная оптимальная точка $k$, увеличение $\beta$ не сильно повлияет на модулярность, но зато и не будет влиять на время работы. На графах, у которых с увеличением $k$ растёт модулярность, увеличение $\beta$ снижает модулярность и время работы.

\begin{figure}[H]
	\begin{tikzpicture}
		\begin{semilogxaxis}[
		    table/col sep = semicolon,
		    height = 0.16\paperheight,
		    width = 0.47\columnwidth,
		    xlabel = {$\beta$},
		    ylabel = {$Q$},
		    no marks,
		    title style = {xshift = 3.8cm, yshift = 0cm},
		    title = {celegans\_metabolic},
		    y label style={at={(axis description cs:0,.5)}, anchor=south},
		    yticklabel style={/pgf/number format/fixed,
                  /pgf/number format/precision=3},
		]
		\addplot table [x = {beta}, y = {q}]{data/es/beta/celegans_beta.csv};
		\end{semilogxaxis}
	\end{tikzpicture}
	\hskip -2.3cm
	\begin{tikzpicture}
		\begin{semilogxaxis}[
		    table/col sep = semicolon,
		    height = 0.16\paperheight,
		    width = 0.47\columnwidth,
		    xlabel = {$\beta$},
		    ylabel = {$t$},
		    y label style={at={(axis description cs:0.1,.5)}, anchor=south},
		    no marks,
		]
		\addplot[red] table [x = {beta}, y = {time}]{data/es/beta/celegans_beta.csv};
		\end{semilogxaxis}
	\end{tikzpicture}	%%%%%%%%%%%%%%%
	\begin{tikzpicture}
		\begin{semilogxaxis}[
		    table/col sep = semicolon,
		    height = 0.16\paperheight,
		    width = 0.47\columnwidth,
		    xlabel = {$\beta$},
		    ylabel = {$Q$},
		    no marks,
		    title style = {xshift = 3.8cm, yshift = 0cm},
		    title = {jazz},
		    y label style={at={(axis description cs:0,.5)}, anchor=south},
		    yticklabel style={/pgf/number format/fixed,
                  /pgf/number format/precision=3},
		]
		\addplot table [x = {beta}, y = {q}]{data/es/beta/jazz_beta.csv};
		\end{semilogxaxis}
	\end{tikzpicture}
	\hskip -2.3cm
	\begin{tikzpicture}
		\begin{semilogxaxis}[
		    table/col sep = semicolon,
		    height = 0.16\paperheight,
		    width = 0.47\columnwidth,
		    xlabel = {$\beta$},
		    ylabel = {$t$},
		    y label style={at={(axis description cs:0.1,.5)}, anchor=south},
		    no marks,
		]
		\addplot[red] table [x = {beta}, y = {time}]{data/es/beta/jazz_beta.csv};
		\end{semilogxaxis}
	\end{tikzpicture}	%%%%%%%%%%%%%%%
	\begin{tikzpicture}
		\begin{semilogxaxis}[
		    table/col sep = semicolon,
		    height = 0.16\paperheight,
		    width = 0.47\columnwidth,
		    xlabel = {$\beta$},
		    ylabel = {$Q$},
		    no marks,
		    title style = {xshift = 3.8cm, yshift = -0.1cm},
		    title = {as-22july06},
		    y label style={at={(axis description cs:0,.5)}, anchor=south},
		    yticklabel style={/pgf/number format/fixed,
                  /pgf/number format/precision=3},
		]
		\addplot table [x = {beta}, y = {q}]{data/es/beta/july_beta.csv};
		\end{semilogxaxis}
	\end{tikzpicture}
	\hskip -2.2cm
	\begin{tikzpicture}
		\begin{semilogxaxis}[
		    table/col sep = semicolon,
		    height = 0.16\paperheight,
		    width = 0.47\columnwidth,
		    xlabel = {$\beta$},
		    ylabel = {$t$},
		    y label style={at={(axis description cs:0.03,.5)}, anchor=south},
		    no marks,
		]
		\addplot[red] table [x = {beta}, y = {time}]{data/es/beta/july_beta.csv};
		\end{semilogxaxis}
	\end{tikzpicture}	%%%%%%%%%%%%%%%
	\begin{tikzpicture}
		\begin{semilogxaxis}[
		    table/col sep = semicolon,
		    height = 0.16\paperheight,
		    width = 0.47\columnwidth,
		    xlabel = {$\beta$},
		    ylabel = {$Q$},
		    no marks,
		    title style = {xshift = 3.8cm, yshift = 0cm},
		    title = {netscience},
		    y label style={at={(axis description cs:0,.5)}, anchor=south},
		    yticklabel style={/pgf/number format/fixed,
                  /pgf/number format/precision=3},
		]
		\addplot table [x = {beta}, y = {q}]{data/es/beta/netscience_beta.csv};
		\end{semilogxaxis}
	\end{tikzpicture}
	\hskip -1.4cm
	\begin{tikzpicture}
		\begin{semilogxaxis}[
		    table/col sep = semicolon,
		    height = 0.16\paperheight,
		    width = 0.47\columnwidth,
		    xlabel = {$\beta$},
		    ylabel = {$t$},
		    y label style={at={(axis description cs:0.05,.5)}, anchor=south},
		    no marks,
		]
		\addplot[red] table [x = {beta}, y = {time}]{data/es/beta/netscience_beta.csv};
		\end{semilogxaxis}
	\end{tikzpicture}
	\caption{Зависимость модулярности и времени работы алгоритма от параметра $\beta$ в работе адаптивной ансамблевой стратегии на четырёх графах}
	\label{fig:es-beta1}
\end{figure}

Рисунок \ref{fig:es-beta1} был получен при применении адаптивной ансамблевой стратегии с параметрами $d = 2,\ f(Q, k) = -1000(\ln Q - \beta \ln k),\ \hat{k}_0 = 5,\ l = 8,\ r = 0.5$. Видно, что на графах \emph{celegans\_metabolic}, \emph{netscience} и \emph{jazz} увеличение $\beta$ очень сильно влияло на время работы, при этом на первом графе модулярность росла, а на втором и третьем вела себя непредсказуемо. На графе \emph{as-22july06} время и модулярность оставались на одном уровне с ростом $\beta$ (это объясняется тем, что оптимальным $k$ является единица).

Другим подходом к ограничению времени является установка максимального возможного значения $k_n^{+}$ и $\hat{k}_n$ (как $k_n^{-}$ и $\hat{k}_n$ ограничены снизу единицей).