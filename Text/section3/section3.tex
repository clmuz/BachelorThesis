%%%%%%%%%%%%%%%%%%%%%%%%%%%%%%%%%%%%%%%%%%%%%%%%%%%%%%%%%%%%%%%%%%%%%%%%%%%%%%%%%%%%%%%%%%%%%%%%%%%%%%%%%%%%%%%

\section{Ансамблевая стратегия}


%%%%%%%%%%%%%%%%%%%%%%%%%%%%%%%%%%%%%%%%%%%%%%%%%%%%%%%%%%%%%%%%%%%%%%%%%%%%%%%%%%%%%%%%%%%%%%%%%%%%%%%%%%%%%%%

\subsection{Адаптивный рандомизированный жадный алгоритм в ансамблевой стратегии}

Ансамблевая стратегия была более подробно описана в подразделе \ref{subsec:ens}, однако общая схема работы заключается в том, что сначала $s$ начальных алгоритмов выделяют  сообщества, а те узлы, в которых они разошлись по мнению, затем разбивает на сообщества финальный алгоритм. Трудоёмкость таким образом складывается из $s$ трудоёмкостей начальных алгоритмов и трудоёмкости финального алгоритма. Таким образом имеет смысл в качестве начальных алгоритмов брать алгоритмы, работающие быстро. Однако в случае, если начальные алгоритмы очень плохие, то есть дают разбиения не лучше случайного --- модулярность ансамблевой стратегии будет приблизительно равна модулярности финального алгоритма. 

Далее разбиение, полученное начальными алгоритмами называется \emph{начальным} разбиением, полученное финальным алгоритмом --- \emph{финальным} разбиением. Также  разбиение, равное максимальному перекрытию начальных разбиений далее называется \emph{промежуточным}.

Если промежуточное разбиение состоит из достаточно большого количества сообществ (как минимум в несколько раз больше $2\sigma$) --- на нём разумно использовать в качестве финального алгоритма адаптивную версию рандомизированного жадного алгоритма, чтобы получать стабильные результаты. Однако на небольших графах адаптивный алгоритм не имеет смысла.

Так же, если граф, подающийся на вход, достаточно большой --- в качестве начального алгоритма на нём имеет смысл использовать адаптивный алгоритм.

В таблице \ref{tab:es1-q} сравниваются результаты работы ансамблевой стратегии с разными начальными и финальными алгоритмами. В качестве начального алгоритма используюся два рандомизированных жадных алгоритма с параметрами $k = 3$ (обозначается $RG_3$) и $k = 10$ (обозначается $RG_{10}$), и адаптивный рандомизированный жадный алгоритм с параметрами $d = 5,\ f(\mu, k) = -10(\ln \mu - 0.02 \ln k),\ \sigma = 1000,\ \hat{k}_0 = 8$ (обозначается $ARG$). А в качестве финального алгоритма используются только рандомизированные жадные алгоримы с $k$ равными 3 и 10, так как на рассматриваемых графах промежуточное разбиение состоит из недостаточно большого количества сообществ, чтобы использовать адаптивный рандомизированный алгоритм.

Ансамблевая стратегия используется с параметром $s = 10$, а обозначается как $ES(init, final)$, где в первом аргументе указывается начальный алгоритм, а во втором --- финальный. Алгоритмы на графе \emph{cond-mat-2003} запускались 21 раз, на \emph{caidaRouterLevel} и \emph{auto40} --- три раза.

\begin{table}[H]
	\caption{Сравнение модулярности и времени ансамблевой стратегии с $RG(3),\ RG(10)$ и $ARG$ в качестве начальных алгоритмов и $RG(3)$ и $RG(10)$ в качестве финальных алгоритмов}
	\label{tab:es1-q}
	\begin{tabularx}{\textwidth}{Xrrrrrr}	\hline
								& \multicolumn{2}{c}{cond-mat-2003}	& \multicolumn{2}{c}{auto40}	& \multicolumn{2}{c}{caidaRouterLevel} 	\\
								& \multicolumn{1}{c}{Q} & \multicolumn{1}{c}{t} & \multicolumn{1}{c}{Q} & \multicolumn{1}{c}{t} & \multicolumn{1}{c}{Q} & \multicolumn{1}{c}{t} \\\hline
		$ES(RG_3, RG_3)$		& 0.16840	& 2.0	& 0.806277	& 47.0	& 0.84078	& 94.0	\\
		$ES(RG_{10}, RG_3)$		& 0.44934	& 4.7	& 0.806325	& 57.7	& 0.84031	& 104.2	\\
		$ES(ARG, RG_3)$			& 0.42708	& 4.8	& 0.806276	& 65.7	& 0.83671	& 115.1	\\
		$ES(RG_3, RG_{10})$		& 0.71155	& 2.3 	& 0.806445	& 46.6	& 0.85372	& 93.7	\\
		$ES(RG_{10}, RG_{10})$	& 0.74794	& 4.8	& 0.806445	& 57.2	& 0.84448	& 104.3	\\
		$ES(ARG, RG_{10})$		& 0.74872	& 4.9	& 0.806472	& 65.8	& 0.85279	& 118.5	\\\hline
	\end{tabularx}
\end{table}

Граф \emph{cond-mat-2003} плохо разбивается случайными жадными алгоритмами с маленьким $k$, если обратить внимание на таблицу \ref{tab:es1-q} --- в случаях, когда $RG_3$ используется в качестве начального алгоритма --- модулярность ансамблевой стратегии сравнима с модулярностью финального алгоритма на этом графе (можно увидеть в таблице \ref{tab:arg-res-q}). В случаях, когда $RG_3$ используется в качестве финального алгоритма --- модулярность получается сравнительно плохой.

Заранее неизвестно, при каких $k$ алгоритм будет хорошо работать, а при каких плохо, кроме того неизвестно по результату, хорошее ли это разбиение для конкретного графа или нет. Кроме того, ансамблевая стратегия работает заметно дольше, чем рандомизированный жадный алгоритм, поэтому запускать ансамблевую стратегию несколько раз для определения хороших параметров не выгодно. Таким образом, выгодно использовать адаптивный рандомизированный жадный алгоритм и в качестве начального алгоритма, и в качестве финального, если промежуточное разбиение подходит по размерам.


%%%%%%%%%%%%%%%%%%%%%%%%%%%%%%%%%%%%%%%%%%%%%%%%%%%%%%%%%%%%%%%%%%%%%%%%%%%%%%%%%%%%%%%%%%%%%%%%%%%%%%%%%%%%%%%

\subsection{Адаптивное построение промежуточного разбиения}

Так как адаптивный рандомизированный жадный алгоритм не может работать на графах с маленьким количеством узлов или на разбиениях с маленьким количеством сообществ --- была предложена другая схема адаптивной ансамблевой стратегии:

\begin{enumerate}
	\item На вход подаётся граф $G$. Выбор параметров алгоритма: $d \in \mathbb{N}$ --- размер возмущения, $f(Q, k): \mathbb{R} \times \mathbb{N} \rightarrow \mathbb{R}$ --- функция качества, $l_1 \in \mathbb{N}$ --- количество шагов, $l_2 \le l_1 \in \mathbb{N}$ --- количество лучших начальных разбиений, учавствующих в создании промежуточного разбиения. И наконец $\hat{k}_0 \in \mathbb{N}$ --- начальная центральная точка. Установка счётчика $n = 0$. Установка список начальных разбиений $S = \emptyset$
	\item Увеличение счётчика $n \rightarrow n + 1$
	\item Вычисление следующий параметров рандомизированного жадного алгоритма $k_n^{-} = \max\{1, \hat{k}_{n - 1} - d\}$ и $k_n^{+} = \hat{k}_{n - 1} + d$
	\item Выделение сообществ в графе $G$ рандомизированным жадным алгоритмом с параметром $k = k_n^{-}$, пара из результирующего разбиения $P_n^{-}$ и его модулярности $Q_n^{-}$ записывается в список $S$
	\item Выделение сообществ в графе $G$ рандомизированным жадным алгоритмом с параметром $k = k_n^{+}$, пара из результирующего разбиения $P_n^{+}$ и его модулярности $Q_n^{+}$ записывается в список $S$
	\item Вычисляются функции качества $y_n^{-} = f(Q_n^{-}, k_n^{-}),\ y_n^{+} = f(Q_n^{+}, k_n^{+})$
	\item Вычисляется следующая центральная точка аналогично \eqref{eq:arg-centre}
	\begin{equation}\label{eq:aes-centre}
		\hat{k}_n = \max\left\{1, \left[\hat{k}_{n - 1} - \frac{y_n^{+} - y_n^{-}}{k_n^{+} - k_n^{-}}\right]\right\}
	\end{equation}
	\item Если $n \ne l_1$ --- переход на второй пункт, иначе --- на следующий
	\item По $l_2$ лучшим разбиениям (с наибольшей модулярностью) в $S$ создаётся промежуточное множество, которое затем передаётся в финальный алгоритм
\end{enumerate}

При этом граф $G$ может быть иметь сколь угодно маленькое количество узлов.


%%%%%%%%%%%%%%%%%%%%%%%%%%%%%%%%%%%%%%%%%%%%%%%%%%%%%%%%%%%%%%%%%%%%%%%%%%%%%%%%%%%%%%%%%%%%%%%%%%%%%%%%%%%%%%%

\subsection{Количество начальных разбиений}

